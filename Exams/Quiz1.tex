%%%%%%%%%%%%%%%%%%%%%%%%%%%%%%%%%%%%%%%%%%%%%%%%%%%%%%%%%%%%%%%%%%%%%%%%%%%%%%%%%%%%%%%%%%%%%%%%%%%%%%%%%%%%%%%%%%%%%%%%%%%%%%%%%%%%%%%%%%%%%%%%%%%%%%%%%%%%%%%%%%%
% Written By Michael Brodskiy
% Class: Fundamentals of Linear Systems
% Professor: I. Salama
%%%%%%%%%%%%%%%%%%%%%%%%%%%%%%%%%%%%%%%%%%%%%%%%%%%%%%%%%%%%%%%%%%%%%%%%%%%%%%%%%%%%%%%%%%%%%%%%%%%%%%%%%%%%%%%%%%%%%%%%%%%%%%%%%%%%%%%%%%%%%%%%%%%%%%%%%%%%%%%%%%%

\include{Includes.tex}

\title{Quiz 1}
\date{\today}
\author{Michael Brodskiy\\ \small Professor: I. Salama}

\begin{document}

\maketitle

\begin{enumerate}[label=\textbf{\Alph*}.]

  \item

    We may begin by finding the energy:

    $$E_{\infty}=\lim_{T\to\infty}\int_{-T}^T |x(t)|^2\,dt$$

    Since the signal is finite, we have a nonzero value only for $0\leq t\leq 1$. Therefore, we may write:

    $$E_{\infty}=\int_{0}^1 |e^t|^2\,dt$$
    $$E_{\infty}=\int_{0}^1 e^{2t}\,dt$$
    $$E_{\infty}=\frac{1}{2}e^{2t}\Big_0^1$$
    $$\boxed{E_{\infty}=\frac{1}{2}e^{2}-\frac{1}{2}}$$

    We then find the power:

    $$P_{\infty}=\lim_{T\to\infty}\frac{E_{\infty}}{2T}$$
    $$\boxed{P_{\infty}=0}$$

    Therefore, because energy is finite and power is 0, then \underline{this is an energy signal}

  \item

    \begin{enumerate}[label=(\roman*)]

      \item For $x(t+2)$:

        \vspace{-30pt}

        \begin{figure}[H]
          \centering
          \tikzset{every picture/.style={line width=0.75pt}} %set default line width to 0.75pt        

\begin{tikzpicture}[x=0.75pt,y=0.75pt,yscale=-1,xscale=1]
%uncomment if require: \path (0,423); %set diagram left start at 0, and has height of 423

%Shape: Axis 2D [id:dp7787745858917079] 
\draw  (314,315.6) -- (618,315.6)(344.4,42) -- (344.4,346) (611,310.6) -- (618,315.6) -- (611,320.6) (339.4,49) -- (344.4,42) -- (349.4,49)  ;
%Shape: Axis 2D [id:dp2833357045793793] 
\draw  (374.8,315.6) -- (70.8,315.6)(344.4,42) -- (344.4,346) (77.8,310.6) -- (70.8,315.6) -- (77.8,320.6) (349.4,49) -- (344.4,42) -- (339.4,49)  ;
%Shape: Grid [id:dp29542181740302487] 
\draw  [draw opacity=0][dash pattern={on 4.5pt off 4.5pt}] (104,76) -- (344.4,76) -- (344.4,314.7) -- (104,314.7) -- cycle ; \draw  [color={rgb, 255:red, 155; green, 155; blue, 155 }  ,draw opacity=1 ][dash pattern={on 4.5pt off 4.5pt}] (104,76) -- (104,314.7)(164,76) -- (164,314.7)(224,76) -- (224,314.7)(284,76) -- (284,314.7)(344,76) -- (344,314.7) ; \draw  [color={rgb, 255:red, 155; green, 155; blue, 155 }  ,draw opacity=1 ][dash pattern={on 4.5pt off 4.5pt}] (104,76) -- (344.4,76)(104,136) -- (344.4,136)(104,196) -- (344.4,196)(104,256) -- (344.4,256) ; \draw  [color={rgb, 255:red, 155; green, 155; blue, 155 }  ,draw opacity=1 ][dash pattern={on 4.5pt off 4.5pt}]  ;
%Shape: Grid [id:dp6629033718892944] 
\draw  [draw opacity=0][dash pattern={on 4.5pt off 4.5pt}] (584,76) -- (343.6,76) -- (343.6,314) -- (584,314) -- cycle ; \draw  [color={rgb, 255:red, 155; green, 155; blue, 155 }  ,draw opacity=1 ][dash pattern={on 4.5pt off 4.5pt}] (584,76) -- (584,314)(524,76) -- (524,314)(464,76) -- (464,314)(404,76) -- (404,314)(344,76) -- (344,314) ; \draw  [color={rgb, 255:red, 155; green, 155; blue, 155 }  ,draw opacity=1 ][dash pattern={on 4.5pt off 4.5pt}] (584,76) -- (343.6,76)(584,136) -- (343.6,136)(584,196) -- (343.6,196)(584,256) -- (343.6,256) ; \draw  [color={rgb, 255:red, 155; green, 155; blue, 155 }  ,draw opacity=1 ][dash pattern={on 4.5pt off 4.5pt}]  ;
%Straight Lines [id:da7162297428339012] 
\draw [color={rgb, 255:red, 208; green, 2; blue, 27 }  ,draw opacity=1 ][line width=1.5]    (104.4,315.6) -- (224,315) ;
%Straight Lines [id:da6489949529763962] 
\draw [color={rgb, 255:red, 208; green, 2; blue, 27 }  ,draw opacity=1 ][line width=1.5]    (224,315) -- (284,196) ;
%Straight Lines [id:da23798784359213232] 
\draw [color={rgb, 255:red, 208; green, 2; blue, 27 }  ,draw opacity=1 ][line width=1.5]    (284.4,196.6) -- (344,196) ;
%Straight Lines [id:da3244245846200695] 
\draw [color={rgb, 255:red, 208; green, 2; blue, 27 }  ,draw opacity=1 ][line width=1.5]    (344.4,315.6) -- (464,315) ;
%Straight Lines [id:da5286072330399609] 
\draw [color={rgb, 255:red, 208; green, 2; blue, 27 }  ,draw opacity=1 ][line width=1.5]  [dash pattern={on 5.63pt off 4.5pt}]  (343.82,196.2) -- (344.18,315.8) ;

% Text Node
\draw (344.23,349.4) node [anchor=north] [inner sep=0.75pt]    {$0$};
% Text Node
\draw (404.23,318.4) node [anchor=north] [inner sep=0.75pt]    {$1$};
% Text Node
\draw (464.23,318.4) node [anchor=north] [inner sep=0.75pt]    {$2$};
% Text Node
\draw (524.23,318.4) node [anchor=north] [inner sep=0.75pt]    {$3$};
% Text Node
\draw (584.23,318.4) node [anchor=north] [inner sep=0.75pt]    {$4$};
% Text Node
\draw (105.23,318.4) node [anchor=north] [inner sep=0.75pt]    {$-4$};
% Text Node
\draw (165.23,318.4) node [anchor=north] [inner sep=0.75pt]    {$-3$};
% Text Node
\draw (225.23,318.4) node [anchor=north] [inner sep=0.75pt]    {$-2$};
% Text Node
\draw (285.23,318.4) node [anchor=north] [inner sep=0.75pt]    {$-1$};
% Text Node
\draw (342,259.4) node [anchor=north east] [inner sep=0.75pt]    {$1$};
% Text Node
\draw (342,199.4) node [anchor=north east] [inner sep=0.75pt]    {$2$};
% Text Node
\draw (342,139.4) node [anchor=north east] [inner sep=0.75pt]    {$3$};
% Text Node
\draw (342,79.4) node [anchor=north east] [inner sep=0.75pt]    {$4$};
% Text Node
\draw (319,38) node [anchor=south west] [inner sep=0.75pt]    {$x( t+2)$};
% Text Node
\draw (623,309.4) node [anchor=north west][inner sep=0.75pt]    {$t$};


\end{tikzpicture}

          \caption{Plot for $x(t+2)$}
          \label{fig:1}
        \end{figure}

      \item For $x\left( -\frac{t}{2}+2 \right)$:

        \vspace{-20pt}

        \begin{figure}[H]
          \centering
          \tikzset{every picture/.style={line width=0.75pt}} %set default line width to 0.75pt        

\begin{tikzpicture}[x=0.75pt,y=0.75pt,yscale=-1,xscale=1]
%uncomment if require: \path (0,423); %set diagram left start at 0, and has height of 423

%Shape: Axis 2D [id:dp7787745858917079] 
\draw  (314,335.6) -- (618,335.6)(344.4,62) -- (344.4,366) (611,330.6) -- (618,335.6) -- (611,340.6) (339.4,69) -- (344.4,62) -- (349.4,69)  ;
%Shape: Axis 2D [id:dp2833357045793793] 
\draw  (374.8,335.6) -- (70.8,335.6)(344.4,62) -- (344.4,366) (77.8,330.6) -- (70.8,335.6) -- (77.8,340.6) (349.4,69) -- (344.4,62) -- (339.4,69)  ;
%Shape: Grid [id:dp29542181740302487] 
\draw  [draw opacity=0][dash pattern={on 4.5pt off 4.5pt}] (104,96) -- (344.4,96) -- (344.4,334.7) -- (104,334.7) -- cycle ; \draw  [color={rgb, 255:red, 155; green, 155; blue, 155 }  ,draw opacity=1 ][dash pattern={on 4.5pt off 4.5pt}] (104,96) -- (104,334.7)(164,96) -- (164,334.7)(224,96) -- (224,334.7)(284,96) -- (284,334.7)(344,96) -- (344,334.7) ; \draw  [color={rgb, 255:red, 155; green, 155; blue, 155 }  ,draw opacity=1 ][dash pattern={on 4.5pt off 4.5pt}] (104,96) -- (344.4,96)(104,156) -- (344.4,156)(104,216) -- (344.4,216)(104,276) -- (344.4,276) ; \draw  [color={rgb, 255:red, 155; green, 155; blue, 155 }  ,draw opacity=1 ][dash pattern={on 4.5pt off 4.5pt}]  ;
%Shape: Grid [id:dp6629033718892944] 
\draw  [draw opacity=0][dash pattern={on 4.5pt off 4.5pt}] (584,96) -- (343.6,96) -- (343.6,334) -- (584,334) -- cycle ; \draw  [color={rgb, 255:red, 155; green, 155; blue, 155 }  ,draw opacity=1 ][dash pattern={on 4.5pt off 4.5pt}] (584,96) -- (584,334)(524,96) -- (524,334)(464,96) -- (464,334)(404,96) -- (404,334)(344,96) -- (344,334) ; \draw  [color={rgb, 255:red, 155; green, 155; blue, 155 }  ,draw opacity=1 ][dash pattern={on 4.5pt off 4.5pt}] (584,96) -- (343.6,96)(584,156) -- (343.6,156)(584,216) -- (343.6,216)(584,276) -- (343.6,276) ; \draw  [color={rgb, 255:red, 155; green, 155; blue, 155 }  ,draw opacity=1 ][dash pattern={on 4.5pt off 4.5pt}]  ;
%Straight Lines [id:da7162297428339012] 
\draw [color={rgb, 255:red, 208; green, 2; blue, 27 }  ,draw opacity=1 ][line width=1.5]    (105,336) -- (344.4,335.6) ;
%Straight Lines [id:da6489949529763962] 
\draw [color={rgb, 255:red, 208; green, 2; blue, 27 }  ,draw opacity=1 ][line width=1.5]    (583,336) -- (464,216) ;
%Straight Lines [id:da23798784359213232] 
\draw [color={rgb, 255:red, 208; green, 2; blue, 27 }  ,draw opacity=1 ][line width=1.5]    (344,216) -- (463.6,215.4) ;
%Straight Lines [id:da3244245846200695] 
\draw [color={rgb, 255:red, 208; green, 2; blue, 27 }  ,draw opacity=1 ][line width=1.5]    (583,336) -- (613,336) ;
%Straight Lines [id:da5286072330399609] 
\draw [color={rgb, 255:red, 208; green, 2; blue, 27 }  ,draw opacity=1 ][line width=1.5]  [dash pattern={on 5.63pt off 4.5pt}]  (343.82,216.2) -- (344.18,335.8) ;

% Text Node
\draw (344.23,369.4) node [anchor=north] [inner sep=0.75pt]    {$0$};
% Text Node
\draw (404.23,338.4) node [anchor=north] [inner sep=0.75pt]    {$1$};
% Text Node
\draw (464.23,338.4) node [anchor=north] [inner sep=0.75pt]    {$2$};
% Text Node
\draw (524.23,338.4) node [anchor=north] [inner sep=0.75pt]    {$3$};
% Text Node
\draw (584.23,338.4) node [anchor=north] [inner sep=0.75pt]    {$4$};
% Text Node
\draw (105.23,338.4) node [anchor=north] [inner sep=0.75pt]    {$-4$};
% Text Node
\draw (165.23,338.4) node [anchor=north] [inner sep=0.75pt]    {$-3$};
% Text Node
\draw (225.23,338.4) node [anchor=north] [inner sep=0.75pt]    {$-2$};
% Text Node
\draw (285.23,338.4) node [anchor=north] [inner sep=0.75pt]    {$-1$};
% Text Node
\draw (342,279.4) node [anchor=north east] [inner sep=0.75pt]    {$1$};
% Text Node
\draw (342,219.4) node [anchor=north east] [inner sep=0.75pt]    {$2$};
% Text Node
\draw (342,159.4) node [anchor=north east] [inner sep=0.75pt]    {$3$};
% Text Node
\draw (342,99.4) node [anchor=north east] [inner sep=0.75pt]    {$4$};
% Text Node
\draw (308,58) node [anchor=south west] [inner sep=0.75pt]    {$x\left( -\frac{t}{2} +2\right)$};
% Text Node
\draw (623,329.4) node [anchor=north west][inner sep=0.75pt]    {$t$};


\end{tikzpicture}

          \caption{Plot for $x\left( -\frac{t}{2}+2 \right)$}
          \label{fig:2}
        \end{figure}

    \end{enumerate}

  \item

    \begin{enumerate}[label=\textbf{\Roman*}.]

      \item We begin by finding whether the individual sinusoids are periodic:

        $$N_1\to \frac{2\pi}{\frac{4\pi}{5}}\to \frac{5}{2}m\to 5$$
        $$N_2\to \frac{2\pi}{\frac{5\pi}{7}}\to \frac{14}{5}m\to 14$$

        Thus, we see both are, individually, periodic.

        We then find the least common factor between the two signals, which comes out to be:

        $$N_o=aN_1+bN_2\to N_1N_2$$
        $$\boxed{N_o=70}$$

      \item From the signal, we see that the leading term is $\left( \frac{1}{2} \right)^n$. Since this is a geometric, decaying series, the signal must converge. Therefore, \underline{the signal is \textbf{not} periodic}

    \end{enumerate}

\end{enumerate}

\end{document}

