%%%%%%%%%%%%%%%%%%%%%%%%%%%%%%%%%%%%%%%%%%%%%%%%%%%%%%%%%%%%%%%%%%%%%%%%%%%%%%%%%%%%%%%%%%%%%%%%%%%%%%%%%%%%%%%%%%%%%%%%%%%%%%%%%%%%%%%%%%%%%%%%%%%%%%%%%%%%%%%%%%%
% Written By Michael Brodskiy
% Class: Fundamentals of Linear Systems
% Professor: I. Salama
%%%%%%%%%%%%%%%%%%%%%%%%%%%%%%%%%%%%%%%%%%%%%%%%%%%%%%%%%%%%%%%%%%%%%%%%%%%%%%%%%%%%%%%%%%%%%%%%%%%%%%%%%%%%%%%%%%%%%%%%%%%%%%%%%%%%%%%%%%%%%%%%%%%%%%%%%%%%%%%%%%%

\include{Includes.tex}

\title{Quiz 2}
\date{\today}
\author{Michael Brodskiy\\ \small Professor: I. Salama}

\begin{document}

\maketitle

\begin{enumerate}[label=\textbf{\Roman*}.]

  \item

    \begin{itemize}

      \item Linearity: \textbf{Linear}

        We may test for linearity by checking that:

        $$ax_1(t)+bx_2(t)=ay_1(t)+by_2(t)$$

        First, we find:

        $$ax_1(t)+bx_2(t)=ae^{2t}x_1(t-1)\cos\left( \frac{\pi}{4}t \right)+be^{2t}x_2(t-1)\cos\left( \frac{\pi}{4}t \right)$$

        And then we find:

        $$ay_1(t)+by_2(t)=ae^{2t}x_1(t-1)\cos\left( \frac{\pi}{4}t \right)+be^{2t}x_2(t-1)\cos\left( \frac{\pi}{4}t \right)$$

        Therefore, because the condition from above is met, \underline{the system is linear}

      \item Time-Variance: \textbf{Time-Varying}

        We may text for time-variance by checking:

        $$x(t-t_o)=y(t-t_o)$$

        First, we find:

        $$x(t-t_o)=e^{2t}x(t-t_o-1)\cos\left( \frac{\pi}{4}t \right)$$

        Then, we find:

        $$y(t-t_o)=e^{2(t-t_o)}x(t-t_o-1)\cos\left( \frac{\pi}{4}(t-t_o) \right)$$

        Therefore, because $x(t-t_o)\neq y(t-t_o)$, \underline{the system is time-varying}

      \item Causality: \textbf{Causal} — By observation, we may see that the function depends on only past or present values of $t$, and, therefore, \underline{it is causal}

      \item Stability: \textbf{Unstable} — We may find whether the function is stable by integrating over its entire range and determining whether the value is finite. By observation, the positive exponential, $e^{2t}$, would diverge, and, therefore, the integrate would evaluate to infinity. Therefore, \underline{the system is unstable}

      \item Invertibility: \textbf{Non-invertible} — We know that a system is non-invertible if two values of $t$ produce the same response. Since the sinusoid oscillates, we may observe that there are many repeated values. Most evidently, we may see that $y(t)$ would be zero for every $t=4n+2$. Therefore, \underline{the system is not invertible}

    \end{itemize}

  \item

    \begin{enumerate}[label=\Alph*.]

      \item We may find $y_1[n]$ by taking $y_1[n]=x_1[n]*h[n]$. Given that $x_1[n]$ consists only of a delta function, we may use the sifting property to evaluate:

        $$x[n]*\delta[n-n_o]=x[n-n_o]$$

        We may write our case as:

        $$y_1[n]=h[n]*(5\delta[n-2])$$
        $$y_1[n]=5h[n-2]$$

        Since we know the system is linear and time invariant, we may write:

        $$\boxed{y_1[n]=5\left( \frac{1}{4} \right)^{n-2}u[n-2]}$$

      \item To simplify analysis, we may define $x_2[n]=4\left( \delta[n+1]+\delta[n]+\delta[n-1] \right)$. Using the same sifting property as part (A), we write:

        $$y_2[n]=h[n]*x_2[n]$$
        $$y_2[n]=4h[n]*(\delta[n+1]+\delta[n]+\delta[n-1])$$
        $$y_2[n]=4h[n+1]+4h[n]+4h[n-1]$$

        We may write this as:

        $$y_2[n]=4\left( \frac{1}{4} \right)^{n+1}u[n+1]+4\left( \frac{1}{4} \right)^{n}u[n]+4\left( \frac{1}{4} \right)^{n-1}u[n-1]$$

        Given that $4=\frac{1}{4}^{-1}$, we may simplify this as:

        $$\boxed{y_2[n]=\left( \frac{1}{4} \right)^{n}u[n+1]+\left( \frac{1}{4} \right)^{n-1}u[n]+\left( \frac{1}{4} \right)^{n-2}u[n-1]}$$

    \end{enumerate}

\end{enumerate}

\end{document}

