

% This document was generated by the publish-function
% from GNU Octave 6.2.0



\documentclass[10pt]{article}
\usepackage{listings}
\usepackage{float}
\usepackage{mathtools}
\usepackage{amssymb}
\usepackage{graphicx}
\usepackage{hyperref}
\usepackage{xcolor}
\usepackage{titlesec}
\usepackage[utf8]{inputenc}
\usepackage[T1]{fontenc}
\usepackage{lmodern}


\lstset{
language=Octave,
numbers=none,
frame=single,
tabsize=2,
showstringspaces=false,
breaklines=true}


\titleformat*{\section}{\Huge\bfseries}
\titleformat*{\subsection}{\large\bfseries}
\renewcommand{\contentsname}{\Large\bfseries Contents}
\setlength{\parindent}{0pt}

\begin{document}

{\Huge\section*{ComputationalHW1}}

\tableofcontents
\vspace*{4em}

\newpage

\begin{lstlisting}
clear; % Clear workspace
\end{lstlisting}


\phantomsection
\addcontentsline{toc}{section}{Question 1}
\subsection*{Question 1}

\begin{lstlisting}
delta = @(n) (n == 0); % Define the unit impulse function, $\delta[n]$
u = @(n) ((n >= 0) & (rem(n,1) == 0)); % Define the unit step function, $u[n]$
a = .5; % Define exponential value
halfExp = @(n) (a).^(n-2) .* u(n) - (a).^(n-2) .* u(n-4); % Define the exponential function

n=-20:25; % Range of times to evaluate
x = @(n) halfExp(n) + delta(n-4) + delta(n-5) + delta(n-6); % Define $x[n]$ per the problem specifications
\end{lstlisting}


\phantomsection
\addcontentsline{toc}{section}{Part A, Plotting $x[n]$}
\subsection*{Part A, Plotting $x[n]$}

\begin{lstlisting}
subplot(2,2,1); % Create subplots
stem(n,x(n),'fill'); % Plot figure
xlabel('n'); % Define $x$-axis title
ylabel('x[n]'); % Define $y$-axis title
ylim([-2 5]); % Establish $y$ limits
xlim([-2 8]); % Establish $x$ limits
\end{lstlisting}
\begin{figure}[H]
\includegraphics[width=\textwidth]{ComputationalHW1-1.eps}
\end{figure}


\phantomsection
\addcontentsline{toc}{section}{Part B, Plotting $x[n-2]$}
\subsection*{Part B, Plotting $x[n-2]$}

\begin{lstlisting}
subplot(2,2,2);
stem(n,x(n-2),'fill');
xlabel('n');
ylabel('x[n-2]');
ylim([-2 5]);
xlim([0 10]);
\end{lstlisting}
\begin{figure}[H]
\includegraphics[width=\textwidth]{ComputationalHW1-2.eps}
\end{figure}


\phantomsection
\addcontentsline{toc}{section}{Part C, Plotting $x[-2n-2]$}
\subsection*{Part C, Plotting $x[-2n-2]$}

\begin{lstlisting}
subplot(2,2,3);
stem(n,x(-2*n-2),'fill');
xlabel('n');
ylabel('x[-2n-2]');
ylim([-2 5]);
xlim([-5 5]);
\end{lstlisting}
\begin{figure}[H]
\includegraphics[width=\textwidth]{ComputationalHW1-3.eps}
\end{figure}


\phantomsection
\addcontentsline{toc}{section}{Part D, Plotting $x\left[\frac{n}{3}-2\right]$}
\subsection*{Part D, Plotting $x\left[\frac{n}{3}-2\right]$}

\begin{lstlisting}
subplot(2,2,4);
stem(n,x(n/3-2),'fill');
xlabel('n');
ylabel('x[(n/3)-2]');
ylim([-2 5]);
xlim([5 25]);
axes('visible', 'off', 'title', 'x[n] and variations' );

pause; % Wait for input before continuing to next question

clear all; % Clear the workspace
\end{lstlisting}
\begin{figure}[H]
\includegraphics[width=\textwidth]{ComputationalHW1-4.eps}
\end{figure}


\phantomsection
\addcontentsline{toc}{section}{Question 2}
\subsection*{Question 2}

\begin{lstlisting}
N = 4; % Define $N$ value
range = 0:4*N; % Define plotting range
x1 = @(m) 2*sin( (2*pi*m) / N) + cos( (6*pi*m) / N); % Define function 1
x2 = @(l) 2*sin( (6*l) / N) + cos( (18*l) / N); % Define function 2

% We can see from the functions that $x_1$ is periodic, since the fundamental period of the first function is $N=4$ and $$(N/3)m=4$$ for the second one, thus the least common multiple of the frequencies is 4
% We can see from the second function that $x_2$ is not periodic, since neither sinusoid contains $\pi$ in its period

subplot(2,1,1);
stem(range, x1(range), 'fill'); % Use stem to plot figure
ylim([-3 3]); % Set $y$-axis bounds
xlim([-1 4*N+1]); % Set $x$-axis bounds
ylabel('Function 1'); % Label $y$-axis

subplot(2,1,2);
stem(range, x2(range), 'fill');
ylim([-3 3]); % Set $y$-axis bounds
xlim([-1 4*N+1]); % Set $x$-axis bounds
xlabel('Question 2 Functions'); % Label $x$-axis
ylabel('Function 2');

% We can see that, indeed, no value in function 2 repeats (at least within this range); on the other hand $x_1$ repeats every 4 samples

pause; % Wait for input before continuing to next question

clear all; % Clear the workspace
\end{lstlisting}
\begin{figure}[H]
\includegraphics[width=\textwidth]{ComputationalHW1-5.eps}
\end{figure}


\phantomsection
\addcontentsline{toc}{section}{Question 3}
\subsection*{Question 3}

\begin{lstlisting}
q3Func = @(a,T,phi,t) (2 .* e.^(a * t) .* cos( (2 * pi * t) / T + phi));
\end{lstlisting}


\phantomsection
\addcontentsline{toc}{section}{Part A, a=0, T\_o=10, \ensuremath{\backslash}phi=0}
\subsection*{Part A, a=0, T\_o=10, \ensuremath{\backslash}phi=0}

\begin{lstlisting}
subplot(3,1,1); % Create subplots for each plot
a = 0;
T = 10;
phi = 0;
t = 0:(5*T); % Define parameter variables for the function ($a, T_o, \phi, t$)
plot(t, q3Func(a,T,phi,t)); % Plot the function
title('Question 3 Functions'); % Title Graphs
ylabel('Function 1'); % Label y axis
\end{lstlisting}
\begin{figure}[H]
\includegraphics[width=\textwidth]{ComputationalHW1-6.eps}
\end{figure}


\phantomsection
\addcontentsline{toc}{section}{Part B, a=0, T\_o=10, \ensuremath{\backslash}phi=-\ensuremath{\backslash}frac\{\ensuremath{\backslash}pi\}\{4\}}
\subsection*{Part B, a=0, T\_o=10, \ensuremath{\backslash}phi=-\ensuremath{\backslash}frac\{\ensuremath{\backslash}pi\}\{4\}}

\begin{lstlisting}
subplot(3,1,2); % Create subplots for each plot
a = 0;
T = 10;
phi = -(pi/4);
t = 0:(5*T); % Define parameter variables for the function ($a, T_o, \phi, t$)
plot(t, q3Func(a,T,phi,t)); % Plot the function
ylabel('Function 2'); % Label y axis

% We see that $\phi$ causes a phase shift of the signal. Because, in this case, it is negative, the signal shifts to the right by $\pi/4$
\end{lstlisting}
\begin{figure}[H]
\includegraphics[width=\textwidth]{ComputationalHW1-7.eps}
\end{figure}


\phantomsection
\addcontentsline{toc}{section}{Part C, a=-.05, T\_o=10, \ensuremath{\backslash}phi=0}
\subsection*{Part C, a=-.05, T\_o=10, \ensuremath{\backslash}phi=0}

\begin{lstlisting}
subplot(3,1,3); % Create subplots for each plot
a = -.05;
T = 10;
phi = 0;
t = 0:(5*T); % Define parameter variables for the function ($a, T_o, \phi, t$)
plot(t, q3Func(a,T,phi,t)); % Plot the function
ylabel('Function 3'); % Label y axis

% We see that $a$ is the attentuation factor. Since it is negative in this case, the signal attenuates as it goes on (gets weaker)

pause; % Wait for input before continuing to next question

clear all; % Clear the workspace
\end{lstlisting}
\begin{figure}[H]
\includegraphics[width=\textwidth]{ComputationalHW1-8.eps}
\end{figure}


\phantomsection
\addcontentsline{toc}{section}{Question 4}
\subsection*{Question 4}

\begin{lstlisting}
No = 3; % Define fundamental period
q4Func = @(n) 5 .* sin((2 * pi / No) * n + (pi / 4)); % Define function without $k$
q4FuncK = @(n,k) 5 .* sin((2 * pi * k / No) * n + (pi / 4)); % Define function with $k$
xlimits = 0:2*No; % Set $x$ limits

% Plotting the graphs
subplot(3,2,1);
stem(xlimits, q4Func(xlimits), 'fill');
ylabel('Function 1');
subplot(3,2,2);
stem(xlimits, q4FuncK(xlimits,1), 'fill');
ylabel('Function 2 (k=1)');
subplot(3,2,3);
stem(xlimits, q4FuncK(xlimits,2), 'fill');
ylabel('Function 3 (k=2)');
subplot(3,2,4);
stem(xlimits, q4FuncK(xlimits,3), 'fill');
ylabel('Function 4 (k=3)');
subplot(3,2,5);
stem(xlimits, q4FuncK(xlimits,4), 'fill');
ylabel('Function 5 (k=4)');
subplot(3,2,6);
stem(xlimits, q4FuncK(xlimits,5), 'fill');
ylabel('Function 6 (k=5)');

% We see that, for all values of $k$ that are not an integer multiple of $N_o$, the same plot is generated

% When $k$ is an integer multiple of $N_o$, the fundamental frequency becomes 1
\end{lstlisting}
\begin{figure}[H]
\includegraphics[width=\textwidth]{ComputationalHW1-9.eps}
\end{figure}


\end{document}
