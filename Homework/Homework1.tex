%%%%%%%%%%%%%%%%%%%%%%%%%%%%%%%%%%%%%%%%%%%%%%%%%%%%%%%%%%%%%%%%%%%%%%%%%%%%%%%%%%%%%%%%%%%%%%%%%%%%%%%%%%%%%%%%%%%%%%%%%%%%%%%%%%%%%%%%%%%%%%%%%%%%%%%%%%%%%%%%%%%
% Written By Michael Brodskiy
% Class: Fundamentals of Linear Systems
% Professor: I. Salama
%%%%%%%%%%%%%%%%%%%%%%%%%%%%%%%%%%%%%%%%%%%%%%%%%%%%%%%%%%%%%%%%%%%%%%%%%%%%%%%%%%%%%%%%%%%%%%%%%%%%%%%%%%%%%%%%%%%%%%%%%%%%%%%%%%%%%%%%%%%%%%%%%%%%%%%%%%%%%%%%%%%

\documentclass[12pt]{article} 
\usepackage{alphalph}
\usepackage[utf8]{inputenc}
\usepackage[russian,english]{babel}
\usepackage{titling}
\usepackage{amsmath}
\usepackage{graphicx}
\usepackage{enumitem}
\usepackage{amssymb}
\usepackage[super]{nth}
\usepackage{everysel}
\usepackage{ragged2e}
\usepackage{geometry}
\usepackage{multicol}
\usepackage{fancyhdr}
\usepackage{cancel}
\usepackage{siunitx}
\usepackage{physics}
\usepackage{tikz}
\usepackage{mathdots}
\usepackage{yhmath}
\usepackage{cancel}
\usepackage{color}
\usepackage{array}
\usepackage{multirow}
\usepackage{gensymb}
\usepackage{tabularx}
\usepackage{extarrows}
\usepackage{booktabs}
\usepackage{lastpage}
\usetikzlibrary{fadings}
\usetikzlibrary{patterns}
\usetikzlibrary{shadows.blur}
\usetikzlibrary{shapes}

\geometry{top=1.0in,bottom=1.0in,left=1.0in,right=1.0in}
\newcommand{\subtitle}[1]{%
  \posttitle{%
    \par\end{center}
    \begin{center}\large#1\end{center}
    \vskip0.5em}%

}
\usepackage{hyperref}
\hypersetup{
colorlinks=true,
linkcolor=blue,
filecolor=magenta,      
urlcolor=blue,
citecolor=blue,
}


\title{Homework 1}
\date{\today}
\author{Michael Brodskiy\\ \small Professor: I. Salama}

\begin{document}

\maketitle

\begin{enumerate}

  \item Express each of the following complex numbers in polar form and plot them

    \begin{enumerate}

      \item $8$

        $$r=\sqrt{8^2+0^2}=8$$
        $$\theta=0$$
        $$z(r,\theta)=r(\cos(\theta)+j\sin(\theta))$$
        $$z(8,0)=8(\cos(0)+j\sin(0))$$
        $$\therefore \text{ In polar: } \boxed{z=8}$$

        \begin{figure}[H]
          \centering
          \tikzset{every picture/.style={line width=0.75pt}} %set default line width to 0.75pt        

\begin{tikzpicture}[x=0.75pt,y=0.75pt,yscale=-1,xscale=1]
%uncomment if require: \path (0,300); %set diagram left start at 0, and has height of 300

%Shape: Axis 2D [id:dp5907524483347102] 
\draw  (267,209.7) -- (469,209.7)(287.2,36) -- (287.2,229) (462,204.7) -- (469,209.7) -- (462,214.7) (282.2,43) -- (287.2,36) -- (292.2,43)  ;
%Shape: Grid [id:dp6199898397920625] 
\draw  [draw opacity=0] (288,49) -- (449,49) -- (449,209) -- (288,209) -- cycle ; \draw  [color={rgb, 255:red, 155; green, 155; blue, 155 }  ,draw opacity=1 ] (288,49) -- (288,209)(308,49) -- (308,209)(328,49) -- (328,209)(348,49) -- (348,209)(368,49) -- (368,209)(388,49) -- (388,209)(408,49) -- (408,209)(428,49) -- (428,209)(448,49) -- (448,209) ; \draw  [color={rgb, 255:red, 155; green, 155; blue, 155 }  ,draw opacity=1 ] (288,49) -- (449,49)(288,69) -- (449,69)(288,89) -- (449,89)(288,109) -- (449,109)(288,129) -- (449,129)(288,149) -- (449,149)(288,169) -- (449,169)(288,189) -- (449,189) ; \draw  [color={rgb, 255:red, 155; green, 155; blue, 155 }  ,draw opacity=1 ]  ;
%Straight Lines [id:da9323494469903814] 
\draw [color={rgb, 255:red, 208; green, 2; blue, 27 }  ,draw opacity=1 ]   (287.2,209.7) -- (446.65,209.01) ;
\draw [shift={(449,209)}, rotate = 359.75] [color={rgb, 255:red, 208; green, 2; blue, 27 }  ,draw opacity=1 ][line width=0.75]      (0, 0) circle [x radius= 3.35, y radius= 3.35]   ;

% Text Node
\draw (287.23,33) node [anchor=south] [inner sep=0.75pt]    {$i$};
% Text Node
\draw (471,200.4) node [anchor=north west][inner sep=0.75pt]    {$r$};

\end{tikzpicture}

          \caption{$z=8$ Plotted on the Imaginary Plane}
          \label{fig:1}
        \end{figure}

      \item $-5$

        $$r=\sqrt{(-5)^2+0^2}=5$$
        $$\theta=\pi$$
        $$z(r,\theta)=r(\cos(\theta)+j\sin(\theta))$$
        $$z(5,\pi)=5(\cos(\pi)+j\sin(\pi))$$
        $$\therefore \text{ In polar: } \boxed{z=-5=5e^{\pi j}}$$

        \begin{figure}[H]
          \centering
          \tikzset{every picture/.style={line width=0.75pt}} %set default line width to 0.75pt        

\begin{tikzpicture}[x=0.75pt,y=0.75pt,yscale=-1,xscale=1]
%uncomment if require: \path (0,300); %set diagram left start at 0, and has height of 300

%Shape: Axis 2D [id:dp5907524483347102] 
\draw  (469,209.7) -- (267,209.7)(448.8,36) -- (448.8,229) (274,204.7) -- (267,209.7) -- (274,214.7) (453.8,43) -- (448.8,36) -- (443.8,43)  ;
%Shape: Grid [id:dp6199898397920625] 
\draw  [draw opacity=0] (448,49) -- (287,49) -- (287,209) -- (448,209) -- cycle ; \draw  [color={rgb, 255:red, 155; green, 155; blue, 155 }  ,draw opacity=1 ] (448,49) -- (448,209)(428,49) -- (428,209)(408,49) -- (408,209)(388,49) -- (388,209)(368,49) -- (368,209)(348,49) -- (348,209)(328,49) -- (328,209)(308,49) -- (308,209)(288,49) -- (288,209) ; \draw  [color={rgb, 255:red, 155; green, 155; blue, 155 }  ,draw opacity=1 ] (448,49) -- (287,49)(448,69) -- (287,69)(448,89) -- (287,89)(448,109) -- (287,109)(448,129) -- (287,129)(448,149) -- (287,149)(448,169) -- (287,169)(448,189) -- (287,189) ; \draw  [color={rgb, 255:red, 155; green, 155; blue, 155 }  ,draw opacity=1 ]  ;
%Straight Lines [id:da9323494469903814] 
\draw [color={rgb, 255:red, 208; green, 2; blue, 27 }  ,draw opacity=1 ]   (448.8,209.7) -- (350.35,209.99) ;
\draw [shift={(348,210)}, rotate = 179.83] [color={rgb, 255:red, 208; green, 2; blue, 27 }  ,draw opacity=1 ][line width=0.75]      (0, 0) circle [x radius= 3.35, y radius= 3.35]   ;

% Text Node
\draw (450.23,33) node [anchor=south] [inner sep=0.75pt]    {$i$};
% Text Node
\draw (246,199.4) node [anchor=north west][inner sep=0.75pt]    {$-r$};

\end{tikzpicture}

          \caption{$z=-5$ Plotted on the Imaginary Plane}
          \label{fig:2}
        \end{figure}

      \item $2j$

        $$r=\sqrt{0^2+(2)^2}=2$$
        $$\theta=\frac{\pi}{2}$$
        $$z(r,\theta)=r(\cos(\theta)+j\sin(\theta))$$
        $$z(2,.5\pi)=2j$$
        $$\therefore \text{ In polar: } \boxed{z=2j=2e^{.5\pi j}}$$

        \begin{figure}[H]
          \centering
          \tikzset{every picture/.style={line width=0.75pt}} %set default line width to 0.75pt        

\begin{tikzpicture}[x=0.75pt,y=0.75pt,yscale=-1,xscale=1]
%uncomment if require: \path (0,300); %set diagram left start at 0, and has height of 300

%Shape: Axis 2D [id:dp5907524483347102] 
\draw  (267,209.7) -- (469,209.7)(287.2,36) -- (287.2,229) (462,204.7) -- (469,209.7) -- (462,214.7) (282.2,43) -- (287.2,36) -- (292.2,43)  ;
%Shape: Grid [id:dp6199898397920625] 
\draw  [draw opacity=0] (288,49) -- (449,49) -- (449,209) -- (288,209) -- cycle ; \draw  [color={rgb, 255:red, 155; green, 155; blue, 155 }  ,draw opacity=1 ] (288,49) -- (288,209)(308,49) -- (308,209)(328,49) -- (328,209)(348,49) -- (348,209)(368,49) -- (368,209)(388,49) -- (388,209)(408,49) -- (408,209)(428,49) -- (428,209)(448,49) -- (448,209) ; \draw  [color={rgb, 255:red, 155; green, 155; blue, 155 }  ,draw opacity=1 ] (288,49) -- (449,49)(288,69) -- (449,69)(288,89) -- (449,89)(288,109) -- (449,109)(288,129) -- (449,129)(288,149) -- (449,149)(288,169) -- (449,169)(288,189) -- (449,189) ; \draw  [color={rgb, 255:red, 155; green, 155; blue, 155 }  ,draw opacity=1 ]  ;
%Straight Lines [id:da9323494469903814] 
\draw [color={rgb, 255:red, 208; green, 2; blue, 27 }  ,draw opacity=1 ]   (287.2,209.7) -- (287.95,171.35) ;
\draw [shift={(288,169)}, rotate = 271.13] [color={rgb, 255:red, 208; green, 2; blue, 27 }  ,draw opacity=1 ][line width=0.75]      (0, 0) circle [x radius= 3.35, y radius= 3.35]   ;

% Text Node
\draw (287.23,33) node [anchor=south] [inner sep=0.75pt]    {$i$};
% Text Node
\draw (471,200.4) node [anchor=north west][inner sep=0.75pt]    {$r$};

\end{tikzpicture}

          \caption{$z=2j$ Plotted on the Imaginary Plane}
          \label{fig:3}
        \end{figure}

      \item $\frac{1}{4}(1-j)^5$

        $$.25(1-j)^2(1-j)^3$$
        $$.25(-2j)(1-j)(1-j)^2$$
        $$.25(-2-2j)(-2j)$$
        $$z=j-1$$
        $$r=\sqrt{1^2+(-1)^2}=\sqrt{2}$$
        $$\theta=\frac{3\pi}{4}$$
        $$z(r,\theta)=r(\cos(\theta)+j\sin(\theta))$$
        $$z(\sqrt{2},.75\pi)=j-1$$
        $$\therefore \text{ In polar: } \boxed{z=j-1=\sqrt{2}e^{.75\pi j}}$$

        \begin{figure}[H]
          \centering
          \tikzset{every picture/.style={line width=0.75pt}} %set default line width to 0.75pt        

\begin{tikzpicture}[x=0.75pt,y=0.75pt,yscale=-1,xscale=1]
%uncomment if require: \path (0,300); %set diagram left start at 0, and has height of 300

%Shape: Axis 2D [id:dp5907524483347102] 
\draw  (469,209.7) -- (267,209.7)(448.8,36) -- (448.8,229) (274,204.7) -- (267,209.7) -- (274,214.7) (453.8,43) -- (448.8,36) -- (443.8,43)  ;
%Shape: Grid [id:dp6199898397920625] 
\draw  [draw opacity=0] (448,49) -- (287,49) -- (287,209) -- (448,209) -- cycle ; \draw  [color={rgb, 255:red, 155; green, 155; blue, 155 }  ,draw opacity=1 ] (448,49) -- (448,209)(428,49) -- (428,209)(408,49) -- (408,209)(388,49) -- (388,209)(368,49) -- (368,209)(348,49) -- (348,209)(328,49) -- (328,209)(308,49) -- (308,209)(288,49) -- (288,209) ; \draw  [color={rgb, 255:red, 155; green, 155; blue, 155 }  ,draw opacity=1 ] (448,49) -- (287,49)(448,69) -- (287,69)(448,89) -- (287,89)(448,109) -- (287,109)(448,129) -- (287,129)(448,149) -- (287,149)(448,169) -- (287,169)(448,189) -- (287,189) ; \draw  [color={rgb, 255:red, 155; green, 155; blue, 155 }  ,draw opacity=1 ]  ;
%Straight Lines [id:da9323494469903814] 
\draw [color={rgb, 255:red, 208; green, 2; blue, 27 }  ,draw opacity=1 ]   (448.8,209.7) -- (429.67,190.66) ;
\draw [shift={(428,189)}, rotate = 224.86] [color={rgb, 255:red, 208; green, 2; blue, 27 }  ,draw opacity=1 ][line width=0.75]      (0, 0) circle [x radius= 3.35, y radius= 3.35]   ;

% Text Node
\draw (450.23,34) node [anchor=south] [inner sep=0.75pt]    {$i$};
% Text Node
\draw (243,200.4) node [anchor=north west][inner sep=0.75pt]    {$-r$};

\end{tikzpicture}

          \caption{$z=\frac{1}{4}(1-j)^5$ Plotted on the Imaginary Axis}
          \label{fig:4}
        \end{figure}

      \item $\frac{(1+j)}{j}e^{\frac{j\pi}{3}}$

        $$\frac{(1+j)}{j}\cdot\frac{-j}{-j}=1-j$$
        $$\tan\left( \frac{\pi}{3} \right)=\frac{b}{a}$$
        $$\frac{b}{a}=\sqrt{3}$$
        $$b=a\sqrt{3}$$
        $$\sqrt{(a\sqrt{3})^2+a^2}=1$$
        $$4a^2=\pm1$$
        $$a=\frac{1}{2}$$
        $$b=\frac{\sqrt{3}}{2}$$
        $$\frac{1}{2}(1-j)(1+\sqrt{3}j)\to\frac{1}{2}((\sqrt{3}+1)+(\sqrt{3}-1)j)$$
        $$r=\frac{1}{2}\sqrt{(\sqrt{3}+1)^2+(\sqrt{3}-1)^2}=\sqrt{(4+2\sqrt{3})+(4-2\sqrt{3})}$$
        $$r=\sqrt{2}$$
        $$\theta=\tan^{-1}\left( \frac{\sqrt{3}-1}{\sqrt{3}+1} \right)=.26179$$
        $$z(r,\theta)=r(\cos(\theta)+j\sin(\theta))$$
        $$z(\sqrt{2},.26179)=\frac{1}{2}(\sqrt{3}+1)+(\sqrt{3}-1)j$$
        $$\therefore \text{ In polar: } \boxed{z=\frac{1}{2}(\sqrt{3}+1)+(\sqrt{3}-1)j=\sqrt{2}e^{.26179j}}$$

        \begin{figure}[H]
          \centering
          \tikzset{every picture/.style={line width=0.75pt}} %set default line width to 0.75pt        

\begin{tikzpicture}[x=0.75pt,y=0.75pt,yscale=-1,xscale=1]
%uncomment if require: \path (0,300); %set diagram left start at 0, and has height of 300

%Shape: Axis 2D [id:dp5907524483347102] 
\draw  (267,209.7) -- (469,209.7)(287.2,36) -- (287.2,229) (462,204.7) -- (469,209.7) -- (462,214.7) (282.2,43) -- (287.2,36) -- (292.2,43)  ;
%Shape: Grid [id:dp6199898397920625] 
\draw  [draw opacity=0] (288,49) -- (449,49) -- (449,209) -- (288,209) -- cycle ; \draw  [color={rgb, 255:red, 155; green, 155; blue, 155 }  ,draw opacity=1 ] (288,49) -- (288,209)(308,49) -- (308,209)(328,49) -- (328,209)(348,49) -- (348,209)(368,49) -- (368,209)(388,49) -- (388,209)(408,49) -- (408,209)(428,49) -- (428,209)(448,49) -- (448,209) ; \draw  [color={rgb, 255:red, 155; green, 155; blue, 155 }  ,draw opacity=1 ] (288,49) -- (449,49)(288,69) -- (449,69)(288,89) -- (449,89)(288,109) -- (449,109)(288,129) -- (449,129)(288,149) -- (449,149)(288,169) -- (449,169)(288,189) -- (449,189) ; \draw  [color={rgb, 255:red, 155; green, 155; blue, 155 }  ,draw opacity=1 ]  ;
%Straight Lines [id:da9323494469903814] 
\draw [color={rgb, 255:red, 208; green, 2; blue, 27 }  ,draw opacity=1 ]   (287.2,209.7) -- (316.77,199.75) ;
\draw [shift={(319,199)}, rotate = 341.4] [color={rgb, 255:red, 208; green, 2; blue, 27 }  ,draw opacity=1 ][line width=0.75]      (0, 0) circle [x radius= 3.35, y radius= 3.35]   ;

% Text Node
\draw (287.23,33) node [anchor=south] [inner sep=0.75pt]    {$i$};
% Text Node
\draw (471,200.4) node [anchor=north west][inner sep=0.75pt]    {$r$};

\end{tikzpicture}

          \caption{$z=\frac{(1+j)}{j}e^{\frac{j\pi}{3}}$ Plotted on the Imaginary Axis}
          \label{fig:5}
        \end{figure}

      \item $(\sqrt{3}-j^5)(1+j)$

        $$j^5=j\to (\sqrt{3}-j)(1+j)=(\sqrt{3}+(\sqrt{3}-1)j+1)$$
        $$(\sqrt{3}+1)+(\sqrt{3}-1)j$$
        $$r=\sqrt{(\sqrt{3}+1)^2+(\sqrt{3}-1)^2}=\sqrt{(4+2\sqrt{3})+(4-2\sqrt{3})}$$
        $$r=\sqrt{8}=2\sqrt{2}$$
        $$\theta=\tan^{-1}\left( \frac{\sqrt{3}-1}{\sqrt{3}+1} \right)=.26179$$
        $$z(r,\theta)=r(\cos(\theta)+j\sin(\theta))$$
        $$z(2\sqrt{2},.26179)=(\sqrt{3}+1)+(\sqrt{3}-1)j$$
        $$\therefore \text{ In polar: } \boxed{z=(\sqrt{3}+1)+(\sqrt{3}-1)j=2\sqrt{2}e^{.26179j}}$$

        \begin{figure}[H]
          \centering
          \tikzset{every picture/.style={line width=0.75pt}} %set default line width to 0.75pt        

\begin{tikzpicture}[x=0.75pt,y=0.75pt,yscale=-1,xscale=1]
%uncomment if require: \path (0,300); %set diagram left start at 0, and has height of 300

%Shape: Axis 2D [id:dp5907524483347102] 
\draw  (267,209.7) -- (469,209.7)(287.2,36) -- (287.2,229) (462,204.7) -- (469,209.7) -- (462,214.7) (282.2,43) -- (287.2,36) -- (292.2,43)  ;
%Shape: Grid [id:dp6199898397920625] 
\draw  [draw opacity=0] (288,49) -- (449,49) -- (449,209) -- (288,209) -- cycle ; \draw  [color={rgb, 255:red, 155; green, 155; blue, 155 }  ,draw opacity=1 ] (288,49) -- (288,209)(308,49) -- (308,209)(328,49) -- (328,209)(348,49) -- (348,209)(368,49) -- (368,209)(388,49) -- (388,209)(408,49) -- (408,209)(428,49) -- (428,209)(448,49) -- (448,209) ; \draw  [color={rgb, 255:red, 155; green, 155; blue, 155 }  ,draw opacity=1 ] (288,49) -- (449,49)(288,69) -- (449,69)(288,89) -- (449,89)(288,109) -- (449,109)(288,129) -- (449,129)(288,149) -- (449,149)(288,169) -- (449,169)(288,189) -- (449,189) ; \draw  [color={rgb, 255:red, 155; green, 155; blue, 155 }  ,draw opacity=1 ]  ;
%Straight Lines [id:da9323494469903814] 
\draw [color={rgb, 255:red, 208; green, 2; blue, 27 }  ,draw opacity=1 ]   (287.2,209.7) -- (338.76,193.7) ;
\draw [shift={(341,193)}, rotate = 342.76] [color={rgb, 255:red, 208; green, 2; blue, 27 }  ,draw opacity=1 ][line width=0.75]      (0, 0) circle [x radius= 3.35, y radius= 3.35]   ;

% Text Node
\draw (287.23,33) node [anchor=south] [inner sep=0.75pt]    {$i$};
% Text Node
\draw (471,200.4) node [anchor=north west][inner sep=0.75pt]    {$r$};

\end{tikzpicture}

          \caption{$z=(\sqrt{3}-j^5)(1+j)$ Plotted on the Imaginary Axis}
          \label{fig:6}
        \end{figure}

      \item $\frac{2(\sqrt{3}-j)}{1+j\sqrt{3}}$

        $$\frac{2\sqrt{3}-2j}{1+j\sqrt{3}}\cdot\frac{1-j\sqrt{3}}{1-j\sqrt{3}}=-2j$$
        $$r=\sqrt{0^2+(-2)^2}=2$$
        $$\theta=\frac{3\pi}{2}$$
        $$z(r,\theta)=r(\cos(\theta)+j\sin(\theta))$$
        $$z(2,1.5\pi)=-2j$$
        $$\therefore \text{ In polar: } \boxed{z=-2j=2e^{1.5\pi j}}$$

        \begin{figure}[H]
          \centering
          \tikzset{every picture/.style={line width=0.75pt}} %set default line width to 0.75pt        

\begin{tikzpicture}[x=0.75pt,y=0.75pt,yscale=-1,xscale=1]
%uncomment if require: \path (0,300); %set diagram left start at 0, and has height of 300

%Shape: Axis 2D [id:dp5907524483347102] 
\draw  (267,55.3) -- (469,55.3)(287.2,229) -- (287.2,36) (462,60.3) -- (469,55.3) -- (462,50.3) (282.2,222) -- (287.2,229) -- (292.2,222)  ;
%Shape: Grid [id:dp6199898397920625] 
\draw  [draw opacity=0] (288,216) -- (449,216) -- (449,56) -- (288,56) -- cycle ; \draw  [color={rgb, 255:red, 155; green, 155; blue, 155 }  ,draw opacity=1 ] (288,216) -- (288,56)(308,216) -- (308,56)(328,216) -- (328,56)(348,216) -- (348,56)(368,216) -- (368,56)(388,216) -- (388,56)(408,216) -- (408,56)(428,216) -- (428,56)(448,216) -- (448,56) ; \draw  [color={rgb, 255:red, 155; green, 155; blue, 155 }  ,draw opacity=1 ] (288,216) -- (449,216)(288,196) -- (449,196)(288,176) -- (449,176)(288,156) -- (449,156)(288,136) -- (449,136)(288,116) -- (449,116)(288,96) -- (449,96)(288,76) -- (449,76) ; \draw  [color={rgb, 255:red, 155; green, 155; blue, 155 }  ,draw opacity=1 ]  ;
%Straight Lines [id:da9323494469903814] 
\draw [color={rgb, 255:red, 208; green, 2; blue, 27 }  ,draw opacity=1 ]   (287.2,55.3) -- (287.95,93.65) ;
\draw [shift={(288,96)}, rotate = 88.87] [color={rgb, 255:red, 208; green, 2; blue, 27 }  ,draw opacity=1 ][line width=0.75]      (0, 0) circle [x radius= 3.35, y radius= 3.35]   ;

% Text Node
\draw (289.23,248) node [anchor=south] [inner sep=0.75pt]    {$-i$};
% Text Node
\draw (471,46.4) node [anchor=north west][inner sep=0.75pt]    {$r$};

\end{tikzpicture}

          \caption{$z=\frac{2(\sqrt{3}-j)}{1+j\sqrt{3}}$ Plotted on the Imaginary Axis}
          \label{fig:7}
        \end{figure}

    \end{enumerate}

  \item Determine the value of $E_{\infty}$ and $P_{\infty}$ for each of the following signals and indicate whether the signal is a power or energy signal or neither.

    \begin{enumerate}

      \item $x_1(t)=\left\{\begin{array}{l r} 5e^{j(4t+\pi/3)},\, & t\geq 2\\ 0,\, & \text{Otherwise}\end{array}$

          $$E_{\infty}=\int_2^{\infty} 25\left[ \cos\left( 4t+\frac{\pi}{3} \right)+j\sin\left( 4t+\frac{\pi}{3} \right) \right]^2\,dt$$
          \begin{center}
            Period of the sinusoids is $\pi/2$
          \end{center}
          $$E_{\infty}=\frac{50}{\pi}\int_2^{2+\frac{\pi}{2}} \left[ \cos\left( 4t+\frac{\pi}{3} \right)+j\sin\left( 4t+\frac{\pi}{3} \right) \right]^2\,dt$$

      \item $x_2(t)=\left\{\begin{array}{l r} 2+2\cos(t),\, & 0<t< 2\pi\\ 0,\, & \text{Otherwise}\end{array}$

          $$P_{\infty}=\lim_{T\to\infty}\frac{1}{2T}\int_{-T}^{T}(2+2\cos(t))^2\,dt$$
          $$P_{\infty}=6$$
          $$\therefore\text{ Power is finite}$$

          $$E_{\infty}=\lim_{T\to\infty}\int_{-T}^T(2+2\cos(t))^2\,dt$$
          $$P_{\infty}=\infty$$
          $$\therefore\text{ Energy is infinite}$$

          \begin{center}
            $$\boxed{\text{Since power is finite and energy is infinite, this is a power signal}}$$
          \end{center}

        \item $x_3[n]=\left\{\begin{array}{l r} (.5)^n,\, & n\geq0\\ 0,\, & \text{Otherwise}\end{array}$

            $$E_{\infty}=\lim_{N\to\infty}\sum_{n=0}^N \left( .25 \right)^n$$
            \begin{center}
              A geometric series must be finite:
            \end{center}
            $$E_{\infty}\approx \left( \frac{1}{1-.25} \right)\approx \frac{4}{3}$$

            $$P_{\infty}=\lim_{N\to\infty}\frac{4/3}{2N+1}\approx 0$$

            \begin{center}
              $\boxed{\text{As such, because energy is finite and average power is 0, this is an energy signal}}$
            \end{center}

    \end{enumerate}

  \item For the discrete time signal shown in Figure P1.3, sketch, and carefully label each of the following.

    \begin{enumerate}

      \item $x[n-4]$

        \begin{figure}[H]
          \centering
          \tikzset{every picture/.style={line width=0.75pt}} %set default line width to 0.75pt        

\begin{tikzpicture}[x=0.75pt,y=0.75pt,yscale=-1,xscale=1]
%uncomment if require: \path (0,364); %set diagram left start at 0, and has height of 364

%Straight Lines [id:da06072638558515686] 
\draw    (378.71,168) -- (79.29,168) ;
%Straight Lines [id:da5547538617459793] 
\draw [color={rgb, 255:red, 208; green, 2; blue, 27 }  ,draw opacity=1 ][fill={rgb, 255:red, 208; green, 2; blue, 27 }  ,fill opacity=1 ][line width=1.5]    (229,96.58) -- (229,168) ;
\draw [shift={(229,96.58)}, rotate = 90] [color={rgb, 255:red, 208; green, 2; blue, 27 }  ,draw opacity=1 ][fill={rgb, 255:red, 208; green, 2; blue, 27 }  ,fill opacity=1 ][line width=1.5]      (0, 0) circle [x radius= 4.36, y radius= 4.36]   ;
%Straight Lines [id:da08967735559368817] 
\draw [color={rgb, 255:red, 208; green, 2; blue, 27 }  ,draw opacity=1 ][fill={rgb, 255:red, 208; green, 2; blue, 27 }  ,fill opacity=1 ][line width=1.5]    (249.42,96.58) -- (249.42,168) ;
\draw [shift={(249.42,96.58)}, rotate = 90] [color={rgb, 255:red, 208; green, 2; blue, 27 }  ,draw opacity=1 ][fill={rgb, 255:red, 208; green, 2; blue, 27 }  ,fill opacity=1 ][line width=1.5]      (0, 0) circle [x radius= 4.36, y radius= 4.36]   ;
%Straight Lines [id:da11193473546126165] 
\draw [color={rgb, 255:red, 208; green, 2; blue, 27 }  ,draw opacity=1 ][fill={rgb, 255:red, 208; green, 2; blue, 27 }  ,fill opacity=1 ][line width=1.5]    (208.58,96.58) -- (208.58,168) ;
\draw [shift={(208.58,96.58)}, rotate = 90] [color={rgb, 255:red, 208; green, 2; blue, 27 }  ,draw opacity=1 ][fill={rgb, 255:red, 208; green, 2; blue, 27 }  ,fill opacity=1 ][line width=1.5]      (0, 0) circle [x radius= 4.36, y radius= 4.36]   ;
%Straight Lines [id:da8545689410720038] 
\draw [color={rgb, 255:red, 208; green, 2; blue, 27 }  ,draw opacity=1 ][fill={rgb, 255:red, 208; green, 2; blue, 27 }  ,fill opacity=1 ][line width=1.5]    (269.84,132.29) -- (269.84,167.71) ;
\draw [shift={(269.84,132.29)}, rotate = 90] [color={rgb, 255:red, 208; green, 2; blue, 27 }  ,draw opacity=1 ][fill={rgb, 255:red, 208; green, 2; blue, 27 }  ,fill opacity=1 ][line width=1.5]      (0, 0) circle [x radius= 4.36, y radius= 4.36]   ;
%Straight Lines [id:da9437858155461635] 
\draw [color={rgb, 255:red, 208; green, 2; blue, 27 }  ,draw opacity=1 ][fill={rgb, 255:red, 208; green, 2; blue, 27 }  ,fill opacity=1 ][line width=1.5]    (188.16,132.29) -- (188.16,167.71) ;
\draw [shift={(188.16,132.29)}, rotate = 90] [color={rgb, 255:red, 208; green, 2; blue, 27 }  ,draw opacity=1 ][fill={rgb, 255:red, 208; green, 2; blue, 27 }  ,fill opacity=1 ][line width=1.5]      (0, 0) circle [x radius= 4.36, y radius= 4.36]   ;
%Shape: Circle [id:dp8415745106048833] 
\draw  [color={rgb, 255:red, 208; green, 2; blue, 27 }  ,draw opacity=1 ][fill={rgb, 255:red, 208; green, 2; blue, 27 }  ,fill opacity=1 ] (163.24,167.42) .. controls (163.24,164.94) and (165.25,162.92) .. (167.74,162.92) .. controls (170.22,162.92) and (172.24,164.94) .. (172.24,167.42) .. controls (172.24,169.91) and (170.22,171.92) .. (167.74,171.92) .. controls (165.25,171.92) and (163.24,169.91) .. (163.24,167.42) -- cycle ;
%Shape: Circle [id:dp1924487381459985] 
\draw  [color={rgb, 255:red, 208; green, 2; blue, 27 }  ,draw opacity=1 ][fill={rgb, 255:red, 208; green, 2; blue, 27 }  ,fill opacity=1 ] (285.76,168) .. controls (285.76,165.51) and (287.78,163.5) .. (290.26,163.5) .. controls (292.75,163.5) and (294.76,165.51) .. (294.76,168) .. controls (294.76,170.49) and (292.75,172.5) .. (290.26,172.5) .. controls (287.78,172.5) and (285.76,170.49) .. (285.76,168) -- cycle ;
%Shape: Circle [id:dp9724739272897065] 
\draw  [color={rgb, 255:red, 208; green, 2; blue, 27 }  ,draw opacity=1 ][fill={rgb, 255:red, 208; green, 2; blue, 27 }  ,fill opacity=1 ] (146.24,149.42) .. controls (146.24,146.94) and (148.25,144.92) .. (150.74,144.92) .. controls (153.22,144.92) and (155.24,146.94) .. (155.24,149.42) .. controls (155.24,151.91) and (153.22,153.92) .. (150.74,153.92) .. controls (148.25,153.92) and (146.24,151.91) .. (146.24,149.42) -- cycle ;
%Shape: Circle [id:dp37116477968769124] 
\draw  [color={rgb, 255:red, 208; green, 2; blue, 27 }  ,draw opacity=1 ][fill={rgb, 255:red, 208; green, 2; blue, 27 }  ,fill opacity=1 ] (128.24,149.42) .. controls (128.24,146.94) and (130.25,144.92) .. (132.74,144.92) .. controls (135.22,144.92) and (137.24,146.94) .. (137.24,149.42) .. controls (137.24,151.91) and (135.22,153.92) .. (132.74,153.92) .. controls (130.25,153.92) and (128.24,151.91) .. (128.24,149.42) -- cycle ;
%Shape: Circle [id:dp0005769558109711692] 
\draw  [color={rgb, 255:red, 208; green, 2; blue, 27 }  ,draw opacity=1 ][fill={rgb, 255:red, 208; green, 2; blue, 27 }  ,fill opacity=1 ] (109.24,149.42) .. controls (109.24,146.94) and (111.25,144.92) .. (113.74,144.92) .. controls (116.22,144.92) and (118.24,146.94) .. (118.24,149.42) .. controls (118.24,151.91) and (116.22,153.92) .. (113.74,153.92) .. controls (111.25,153.92) and (109.24,151.91) .. (109.24,149.42) -- cycle ;
%Shape: Circle [id:dp5765900568262574] 
\draw  [color={rgb, 255:red, 208; green, 2; blue, 27 }  ,draw opacity=1 ][fill={rgb, 255:red, 208; green, 2; blue, 27 }  ,fill opacity=1 ] (336.24,147.42) .. controls (336.24,144.94) and (338.25,142.92) .. (340.74,142.92) .. controls (343.22,142.92) and (345.24,144.94) .. (345.24,147.42) .. controls (345.24,149.91) and (343.22,151.92) .. (340.74,151.92) .. controls (338.25,151.92) and (336.24,149.91) .. (336.24,147.42) -- cycle ;
%Shape: Circle [id:dp3702216225275915] 
\draw  [color={rgb, 255:red, 208; green, 2; blue, 27 }  ,draw opacity=1 ][fill={rgb, 255:red, 208; green, 2; blue, 27 }  ,fill opacity=1 ] (318.24,147.42) .. controls (318.24,144.94) and (320.25,142.92) .. (322.74,142.92) .. controls (325.22,142.92) and (327.24,144.94) .. (327.24,147.42) .. controls (327.24,149.91) and (325.22,151.92) .. (322.74,151.92) .. controls (320.25,151.92) and (318.24,149.91) .. (318.24,147.42) -- cycle ;
%Shape: Circle [id:dp5762492676349926] 
\draw  [color={rgb, 255:red, 208; green, 2; blue, 27 }  ,draw opacity=1 ][fill={rgb, 255:red, 208; green, 2; blue, 27 }  ,fill opacity=1 ] (299.24,147.42) .. controls (299.24,144.94) and (301.25,142.92) .. (303.74,142.92) .. controls (306.22,142.92) and (308.24,144.94) .. (308.24,147.42) .. controls (308.24,149.91) and (306.22,151.92) .. (303.74,151.92) .. controls (301.25,151.92) and (299.24,149.91) .. (299.24,147.42) -- cycle ;

% Text Node
\draw (380.71,171.4) node [anchor=north west][inner sep=0.75pt]    {$n$};
% Text Node
\draw (181.16,132.29) node [anchor=east] [inner sep=0.75pt]  [font=\footnotesize]  {$\frac{1}{2}$};
% Text Node
\draw (275.84,132.29) node [anchor=west] [inner sep=0.75pt]  [font=\footnotesize]  {$\frac{1}{2}$};
% Text Node
\draw (200.58,94.58) node [anchor=east] [inner sep=0.75pt]  [font=\footnotesize]  {$1$};
% Text Node
\draw (188.16,171.11) node [anchor=north] [inner sep=0.75pt]  [font=\footnotesize]  {$2$};
% Text Node
\draw (208.58,171.4) node [anchor=north] [inner sep=0.75pt]  [font=\footnotesize]  {$3$};
% Text Node
\draw (229,171.4) node [anchor=north] [inner sep=0.75pt]  [font=\footnotesize]  {$4$};
% Text Node
\draw (249.42,171.4) node [anchor=north] [inner sep=0.75pt]  [font=\footnotesize]  {$5$};
% Text Node
\draw (269.84,171.11) node [anchor=north] [inner sep=0.75pt]  [font=\footnotesize]  {$6$};
% Text Node
\draw (297,81.4) node [anchor=north west][inner sep=0.75pt]    {$x[ n-4]$};


\end{tikzpicture}

          \label{fig:8}
        \end{figure}

      \item $x[2n+2]$

        \begin{figure}[H]
          \centering
          \tikzset{every picture/.style={line width=0.75pt}} %set default line width to 0.75pt        

\begin{tikzpicture}[x=0.75pt,y=0.75pt,yscale=-1,xscale=1]
%uncomment if require: \path (0,364); %set diagram left start at 0, and has height of 364

%Straight Lines [id:da06072638558515686] 
\draw    (378.71,168) -- (79.29,168) ;
%Straight Lines [id:da5547538617459793] 
\draw [color={rgb, 255:red, 208; green, 2; blue, 27 }  ,draw opacity=1 ][fill={rgb, 255:red, 208; green, 2; blue, 27 }  ,fill opacity=1 ][line width=1.5]    (229,96.58) -- (229,168) ;
\draw [shift={(229,96.58)}, rotate = 90] [color={rgb, 255:red, 208; green, 2; blue, 27 }  ,draw opacity=1 ][fill={rgb, 255:red, 208; green, 2; blue, 27 }  ,fill opacity=1 ][line width=1.5]      (0, 0) circle [x radius= 4.36, y radius= 4.36]   ;
%Straight Lines [id:da8545689410720038] 
\draw [color={rgb, 255:red, 208; green, 2; blue, 27 }  ,draw opacity=1 ][fill={rgb, 255:red, 208; green, 2; blue, 27 }  ,fill opacity=1 ][line width=1.5]    (249.42,132.29) -- (249.42,167.71) ;
\draw [shift={(249.42,132.29)}, rotate = 90] [color={rgb, 255:red, 208; green, 2; blue, 27 }  ,draw opacity=1 ][fill={rgb, 255:red, 208; green, 2; blue, 27 }  ,fill opacity=1 ][line width=1.5]      (0, 0) circle [x radius= 4.36, y radius= 4.36]   ;
%Straight Lines [id:da9437858155461635] 
\draw [color={rgb, 255:red, 208; green, 2; blue, 27 }  ,draw opacity=1 ][fill={rgb, 255:red, 208; green, 2; blue, 27 }  ,fill opacity=1 ][line width=1.5]    (212.16,132.29) -- (212.16,167.71) ;
\draw [shift={(212.16,132.29)}, rotate = 90] [color={rgb, 255:red, 208; green, 2; blue, 27 }  ,draw opacity=1 ][fill={rgb, 255:red, 208; green, 2; blue, 27 }  ,fill opacity=1 ][line width=1.5]      (0, 0) circle [x radius= 4.36, y radius= 4.36]   ;
%Shape: Circle [id:dp8415745106048833] 
\draw  [color={rgb, 255:red, 208; green, 2; blue, 27 }  ,draw opacity=1 ][fill={rgb, 255:red, 208; green, 2; blue, 27 }  ,fill opacity=1 ] (190.24,167.42) .. controls (190.24,164.94) and (192.25,162.92) .. (194.74,162.92) .. controls (197.22,162.92) and (199.24,164.94) .. (199.24,167.42) .. controls (199.24,169.91) and (197.22,171.92) .. (194.74,171.92) .. controls (192.25,171.92) and (190.24,169.91) .. (190.24,167.42) -- cycle ;
%Shape: Circle [id:dp1924487381459985] 
\draw  [color={rgb, 255:red, 208; green, 2; blue, 27 }  ,draw opacity=1 ][fill={rgb, 255:red, 208; green, 2; blue, 27 }  ,fill opacity=1 ] (265.76,168) .. controls (265.76,165.51) and (267.78,163.5) .. (270.26,163.5) .. controls (272.75,163.5) and (274.76,165.51) .. (274.76,168) .. controls (274.76,170.49) and (272.75,172.5) .. (270.26,172.5) .. controls (267.78,172.5) and (265.76,170.49) .. (265.76,168) -- cycle ;
%Shape: Circle [id:dp9724739272897065] 
\draw  [color={rgb, 255:red, 208; green, 2; blue, 27 }  ,draw opacity=1 ][fill={rgb, 255:red, 208; green, 2; blue, 27 }  ,fill opacity=1 ] (173.24,149.42) .. controls (173.24,146.94) and (175.25,144.92) .. (177.74,144.92) .. controls (180.22,144.92) and (182.24,146.94) .. (182.24,149.42) .. controls (182.24,151.91) and (180.22,153.92) .. (177.74,153.92) .. controls (175.25,153.92) and (173.24,151.91) .. (173.24,149.42) -- cycle ;
%Shape: Circle [id:dp37116477968769124] 
\draw  [color={rgb, 255:red, 208; green, 2; blue, 27 }  ,draw opacity=1 ][fill={rgb, 255:red, 208; green, 2; blue, 27 }  ,fill opacity=1 ] (155.24,149.42) .. controls (155.24,146.94) and (157.25,144.92) .. (159.74,144.92) .. controls (162.22,144.92) and (164.24,146.94) .. (164.24,149.42) .. controls (164.24,151.91) and (162.22,153.92) .. (159.74,153.92) .. controls (157.25,153.92) and (155.24,151.91) .. (155.24,149.42) -- cycle ;
%Shape: Circle [id:dp0005769558109711692] 
\draw  [color={rgb, 255:red, 208; green, 2; blue, 27 }  ,draw opacity=1 ][fill={rgb, 255:red, 208; green, 2; blue, 27 }  ,fill opacity=1 ] (136.24,149.42) .. controls (136.24,146.94) and (138.25,144.92) .. (140.74,144.92) .. controls (143.22,144.92) and (145.24,146.94) .. (145.24,149.42) .. controls (145.24,151.91) and (143.22,153.92) .. (140.74,153.92) .. controls (138.25,153.92) and (136.24,151.91) .. (136.24,149.42) -- cycle ;
%Shape: Circle [id:dp5765900568262574] 
\draw  [color={rgb, 255:red, 208; green, 2; blue, 27 }  ,draw opacity=1 ][fill={rgb, 255:red, 208; green, 2; blue, 27 }  ,fill opacity=1 ] (319.24,147.42) .. controls (319.24,144.94) and (321.25,142.92) .. (323.74,142.92) .. controls (326.22,142.92) and (328.24,144.94) .. (328.24,147.42) .. controls (328.24,149.91) and (326.22,151.92) .. (323.74,151.92) .. controls (321.25,151.92) and (319.24,149.91) .. (319.24,147.42) -- cycle ;
%Shape: Circle [id:dp3702216225275915] 
\draw  [color={rgb, 255:red, 208; green, 2; blue, 27 }  ,draw opacity=1 ][fill={rgb, 255:red, 208; green, 2; blue, 27 }  ,fill opacity=1 ] (301.24,147.42) .. controls (301.24,144.94) and (303.25,142.92) .. (305.74,142.92) .. controls (308.22,142.92) and (310.24,144.94) .. (310.24,147.42) .. controls (310.24,149.91) and (308.22,151.92) .. (305.74,151.92) .. controls (303.25,151.92) and (301.24,149.91) .. (301.24,147.42) -- cycle ;
%Shape: Circle [id:dp5762492676349926] 
\draw  [color={rgb, 255:red, 208; green, 2; blue, 27 }  ,draw opacity=1 ][fill={rgb, 255:red, 208; green, 2; blue, 27 }  ,fill opacity=1 ] (282.24,147.42) .. controls (282.24,144.94) and (284.25,142.92) .. (286.74,142.92) .. controls (289.22,142.92) and (291.24,144.94) .. (291.24,147.42) .. controls (291.24,149.91) and (289.22,151.92) .. (286.74,151.92) .. controls (284.25,151.92) and (282.24,149.91) .. (282.24,147.42) -- cycle ;

% Text Node
\draw (380.71,171.4) node [anchor=north west][inner sep=0.75pt]    {$n$};
% Text Node
\draw (206.16,132.29) node [anchor=east] [inner sep=0.75pt]  [font=\footnotesize]  {$\frac{1}{2}$};
% Text Node
\draw (256.42,132.29) node [anchor=west] [inner sep=0.75pt]  [font=\footnotesize]  {$\frac{1}{2}$};
% Text Node
\draw (225,96.58) node [anchor=east] [inner sep=0.75pt]  [font=\footnotesize]  {$1$};
% Text Node
\draw (208.58,171.4) node [anchor=north] [inner sep=0.75pt]  [font=\footnotesize]  {$-2$};
% Text Node
\draw (229,171.4) node [anchor=north] [inner sep=0.75pt]  [font=\footnotesize]  {$-1$};
% Text Node
\draw (249.42,171.4) node [anchor=north] [inner sep=0.75pt]  [font=\footnotesize]  {$0$};
% Text Node
\draw (297,81.4) node [anchor=north west][inner sep=0.75pt]    {$x[ 2n+2]$};


\end{tikzpicture}

          \label{fig:9}
        \end{figure}

    \end{enumerate}

  \item For the continuous time signal shown in Figure P1.4, sketch, and carefully label each of the following.

    \begin{enumerate}

      \item $x(t+3)$

        \begin{figure}[H]
          \centering
          \tikzset{every picture/.style={line width=0.75pt}} %set default line width to 0.75pt        

\begin{tikzpicture}[x=0.75pt,y=0.75pt,yscale=-1,xscale=1]
%uncomment if require: \path (0,364); %set diagram left start at 0, and has height of 364

%Straight Lines [id:da2894505135819435] 
\draw    (94.29,251) -- (452.71,251) ;
\draw [shift={(454.71,251)}, rotate = 180] [color={rgb, 255:red, 0; green, 0; blue, 0 }  ][line width=0.75]    (10.93,-3.29) .. controls (6.95,-1.4) and (3.31,-0.3) .. (0,0) .. controls (3.31,0.3) and (6.95,1.4) .. (10.93,3.29)   ;
%Straight Lines [id:da5856960685004501] 
\draw    (399.5,251) -- (399.5,78.58) ;
\draw [shift={(399.5,76.58)}, rotate = 90] [color={rgb, 255:red, 0; green, 0; blue, 0 }  ][line width=0.75]    (10.93,-3.29) .. controls (6.95,-1.4) and (3.31,-0.3) .. (0,0) .. controls (3.31,0.3) and (6.95,1.4) .. (10.93,3.29)   ;
%Straight Lines [id:da3025731174452795] 
\draw [color={rgb, 255:red, 208; green, 2; blue, 27 }  ,draw opacity=1 ][line width=1.5]    (127.5,182.29) -- (127.5,251.71) ;
%Straight Lines [id:da3962505467479802] 
\draw [color={rgb, 255:red, 208; green, 2; blue, 27 }  ,draw opacity=1 ][line width=1.5]    (192.92,182.29) -- (127.5,182.29) ;
%Straight Lines [id:da1537591064735193] 
\draw [color={rgb, 255:red, 208; green, 2; blue, 27 }  ,draw opacity=1 ][line width=1.5]    (192.92,112.87) -- (192.92,182.29) ;
%Straight Lines [id:da5723293546791752] 
\draw [color={rgb, 255:red, 208; green, 2; blue, 27 }  ,draw opacity=1 ][line width=1.5]    (258.34,112.87) -- (258.34,182.29) ;
%Straight Lines [id:da8669489295807551] 
\draw [color={rgb, 255:red, 208; green, 2; blue, 27 }  ,draw opacity=1 ][line width=1.5]    (258.34,182.29) -- (334.43,250.38) ;
%Straight Lines [id:da2726645254117428] 
\draw [color={rgb, 255:red, 208; green, 2; blue, 27 }  ,draw opacity=1 ][line width=1.5]    (258.34,112.87) -- (192.92,112.87) ;

% Text Node
\draw (456.71,254.4) node [anchor=north west][inner sep=0.75pt]    {$t$};
% Text Node
\draw (401.5,73.18) node [anchor=south west] [inner sep=0.75pt]    {$x( t)$};
% Text Node
\draw (330.08,254.4) node [anchor=north] [inner sep=0.75pt]    {$-1$};
% Text Node
\draw (260.66,254.4) node [anchor=north] [inner sep=0.75pt]    {$-2$};
% Text Node
\draw (191.24,254.4) node [anchor=north] [inner sep=0.75pt]    {$-3$};
% Text Node
\draw (121.81,254.4) node [anchor=north] [inner sep=0.75pt]    {$-4$};
% Text Node
\draw (401.5,179.79) node [anchor=west] [inner sep=0.75pt]    {$1$};
% Text Node
\draw (401.5,108.58) node [anchor=west] [inner sep=0.75pt]    {$2$};


\end{tikzpicture}

          \caption{Figure Showing Transformation $x(t)\to x(t+3)$}
          \label{fig:10}
        \end{figure}

      \item $x\left( 3-\frac{2}{3}t \right)$

        \begin{figure}[H]
          \centering
          \tikzset{every picture/.style={line width=0.75pt}} %set default line width to 0.75pt        

\begin{tikzpicture}[x=0.75pt,y=0.75pt,yscale=-1,xscale=1]
%uncomment if require: \path (0,364); %set diagram left start at 0, and has height of 364

%Straight Lines [id:da2894505135819435] 
\draw    (194.5,251) -- (621.71,251) ;
\draw [shift={(623.71,251)}, rotate = 180] [color={rgb, 255:red, 0; green, 0; blue, 0 }  ][line width=0.75]    (10.93,-3.29) .. controls (6.95,-1.4) and (3.31,-0.3) .. (0,0) .. controls (3.31,0.3) and (6.95,1.4) .. (10.93,3.29)   ;
%Straight Lines [id:da5856960685004501] 
\draw    (194.5,251) -- (194.5,78.58) ;
\draw [shift={(194.5,76.58)}, rotate = 90] [color={rgb, 255:red, 0; green, 0; blue, 0 }  ][line width=0.75]    (10.93,-3.29) .. controls (6.95,-1.4) and (3.31,-0.3) .. (0,0) .. controls (3.31,0.3) and (6.95,1.4) .. (10.93,3.29)   ;
%Straight Lines [id:da3025731174452795] 
\draw [color={rgb, 255:red, 208; green, 2; blue, 27 }  ,draw opacity=1 ][line width=1.5]    (587.03,181.58) -- (587.03,251) ;
%Straight Lines [id:da3962505467479802] 
\draw [color={rgb, 255:red, 208; green, 2; blue, 27 }  ,draw opacity=1 ][line width=1.5]    (587.03,181.58) -- (488.9,181.58) ;
%Straight Lines [id:da1537591064735193] 
\draw [color={rgb, 255:red, 208; green, 2; blue, 27 }  ,draw opacity=1 ][line width=1.5]    (390.76,112.16) -- (390.76,181.58) ;
%Straight Lines [id:da5723293546791752] 
\draw [color={rgb, 255:red, 208; green, 2; blue, 27 }  ,draw opacity=1 ][line width=1.5]    (488.9,112.16) -- (488.9,181.58) ;
%Straight Lines [id:da8669489295807551] 
\draw [color={rgb, 255:red, 208; green, 2; blue, 27 }  ,draw opacity=1 ][line width=1.5]    (390.76,181.58) -- (292.63,251) ;
%Straight Lines [id:da8123069025892542] 
\draw [color={rgb, 255:red, 208; green, 2; blue, 27 }  ,draw opacity=1 ][line width=1.5]    (488.9,112.16) -- (390.76,112.16) ;

% Text Node
\draw (625.71,254.4) node [anchor=north west][inner sep=0.75pt]    {$t$};
% Text Node
\draw (196.5,73.18) node [anchor=south west] [inner sep=0.75pt]    {$x( t)$};
% Text Node
\draw (259.92,254.4) node [anchor=north] [inner sep=0.75pt]    {$1$};
% Text Node
\draw (325.34,254.4) node [anchor=north] [inner sep=0.75pt]    {$2$};
% Text Node
\draw (390.76,254.4) node [anchor=north] [inner sep=0.75pt]    {$3$};
% Text Node
\draw (456.19,254.4) node [anchor=north] [inner sep=0.75pt]    {$4$};
% Text Node
\draw (196.5,181.58) node [anchor=west] [inner sep=0.75pt]    {$1$};
% Text Node
\draw (196.5,112.16) node [anchor=west] [inner sep=0.75pt]    {$2$};
% Text Node
\draw (521.61,254.4) node [anchor=north] [inner sep=0.75pt]    {$5$};
% Text Node
\draw (587.03,254.4) node [anchor=north] [inner sep=0.75pt]    {$6$};


\end{tikzpicture}

          \caption{Figure Showing Transformation $x(t)\to x\left( 3-\frac{2}{3}t \right)$}
          \label{fig:11}
        \end{figure}

    \end{enumerate}

  \item Determine and sketch the even and odd parts of the signals depicted in Figure P1.5. Label your sketches carefully.

    \begin{enumerate}

      \item 

      \item 

    \end{enumerate}

  \item Determine and sketch the even and odd parts of the signal depicted in Figure P1.6. Label your sketches carefully.

  \item Express the real part of each of the following signals in the form $Ae^{-at}\cos(\omega t + \phi)$ where $A$, $a$, $\omega$ and $\phi$ are real numbers with $A > 0$ and $-\pi < \phi \leq \pi$.

    \begin{enumerate}

      \item $x_1(t)=4e^{-2t}\sin\left( 10t+\frac{3\pi}{4}\right)\cos\left( 10t+\frac{3\pi}{4} \right)$

        \begin{center}
          Per trig identities, we can rewrite this as:
        \end{center}
        $$2e^{-2t}\sin\left( 20t+\frac{3\pi}{2}\right) \right)$$
        \begin{center}
          Per another identity, we can convert $\sin\to\cos$:
        \end{center}
        $$2e^{-2t}\cos\left( \frac{\pi}{2}-20t-\frac{3\pi}{2}\right) \right)$$
        $$x_1(t)=2e^{-2t}\cos\left( -20t-\pi\right) \right)$$
        \begin{center}
          Since $\cos(x)=\cos(-x)$, we finally write:
        \end{center}
        $$\boxed{x_1(t)=2e^{-2t}\cos\left( 20t+\pi\right) \right)}$$

      \item $x_2(t)=j(1-j)e^{(-5+j\pi)t}$

        We can rewrite this in terms of exponentials:

        $$x_2(t)=e^{\frac{\pi}{2}j}\left( \sqrt{2}e^{-\frac{\pi}{4}j} \right)\left( e^{(-5+j\pi)t} \right)$$
        $$x_2(t)=\sqrt{2}e^{-5t}\left( e^{j\pi t+\frac{\pi}{2}j-\frac{\pi}{4}j} \right)$$
        $$x_2(t)=\sqrt{2}e^{-5t}\left( e^{j(\pi t+\frac{\pi}{4})} \right)$$
        $$\boxed{x_2(t)=\sqrt{2}e^{-5t}\cos\left(\pi t+\frac{\pi}{4}\right)}$$

    \end{enumerate}

  \item Determine whether each of the following continuous time signals is periodic. If the signal is periodic, determine its fundamental period.

    \begin{enumerate}

      \item $x(t)=5\cos\left( 400\pi t+\frac{\pi}{4} \right)$

        The function is periodic, with angular frequency $\omega=400\pi\left[ \dfrac{\text{rad}}{\si{\second}} \right]$

        This gives us the fundamental period:

        $$\boxed{T=\frac{2\pi}{400\pi}=.005[\si{\second}]}$$

      \item $x(t)=20e^{j(\pi t-2)}$

        The function is periodic, with angular frequency $\omega=\pi\left[ \dfrac{\text{rad}}{\si{\second}} \right]$

        This gives us the fundamental period:

        $$\boxed{T=\frac{2\pi}{\pi}=2[\si{\second}]}$$

      \item $x(t)=2\left[ \sin\left( 50\pi t - \frac{\pi}{3} \right) \right]^2$

        The function is periodic, with angular frequency $\omega=50\pi\left[ \dfrac{\text{rad}}{\si{\second}} \right]$

        This gives us the fundamental period:

        $$\boxed{T=\frac{2\pi}{50\pi}=.04[\si{\second}]}$$

      \item $x(t)=\left\{\begin{array}{l r} 2\sin(5\pi t),\, & t\geq 0\\ -2\sin(-5\pi t),\, &t<0\end{array}$

        Per trigonometric identities, we know that:

        $$\sin(t)=-\sin(-t)$$

        Thus, the function presented is simply:

        $$x(t)=2\sin(5\pi t)$$

        This function is periodic, with angular frequency $\omega=5\pi\left[ \dfrac{\text{rad}}{\si{\second}} \right]$

        This gives us the fundamental period:

        $$\boxed{T=\frac{2\pi}{5\pi}=.4[\si{\second}]}$$

    \end{enumerate}

  \item Determine whether each of the following discrete time signals is periodic. If the signal is periodic, determine its fundamental period.

    \begin{enumerate}

      \item $x[n]=2\cos\left( \frac{7}{11}n+\frac{\pi}{2} \right)$

        To be periodic, $(2\pi/\Omega_o)m$ must be rational. We see:

        $$\Omega_o=\frac{7}{11}\to\frac{22\pi m}{7}$$

        As a result of the $\pi$, this is never rational and therefore not periodic.

      \item $x[n]=\cos(\pi n)+4\sin\left( \frac{\pi}{4}n^2 \right)$

        To be periodic, both sinusoids must have $2\pi/\Omega_o$ be rational. We see:

        $$\Omega_1=\pi\to\frac{2\pi}{\pi}=2\text{ and }\Omega_2=\pi/4\to\frac{2\pi}{\pi/4}=8$$

        Thus, the function is periodic. The period is the smallest number such that the two periods are a common divisor of the integer. Since 8 is divisible by 2, \underline{the fundamental period is 8}.

      \item $x[n]=3\sin\left( \frac{\pi}{3}n \right)+\cos\left( \frac{\pi}{4}n \right)-3\cos\left( \frac{\pi}{6}n+\frac{\pi}{3} \right)$ 

        Once again, each of the sinusoids must be periodic:

        $$\Omega_1=\pi/3\to\frac{2\pi}{\pi/3}=6\text{ and }\Omega_2=\pi/4\to\frac{2\pi}{\pi/4}=8\text{ and }\Omega_3=\pi/6\to\frac{2\pi}{\pi/6}=12$$

        Thus, we see these functions are all periodic. The smallest integer which is divisible by all of these values is 24, and, thus, \underline{the fundamental period is 24}.

    \end{enumerate}

\end{enumerate}

\end{document}

