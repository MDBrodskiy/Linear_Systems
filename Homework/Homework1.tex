%%%%%%%%%%%%%%%%%%%%%%%%%%%%%%%%%%%%%%%%%%%%%%%%%%%%%%%%%%%%%%%%%%%%%%%%%%%%%%%%%%%%%%%%%%%%%%%%%%%%%%%%%%%%%%%%%%%%%%%%%%%%%%%%%%%%%%%%%%%%%%%%%%%%%%%%%%%%%%%%%%%
% Written By Michael Brodskiy
% Class: Fundamentals of Linear Systems
% Professor: I. Salama
%%%%%%%%%%%%%%%%%%%%%%%%%%%%%%%%%%%%%%%%%%%%%%%%%%%%%%%%%%%%%%%%%%%%%%%%%%%%%%%%%%%%%%%%%%%%%%%%%%%%%%%%%%%%%%%%%%%%%%%%%%%%%%%%%%%%%%%%%%%%%%%%%%%%%%%%%%%%%%%%%%%

\include{Includes.tex}

\title{Homework 1}
\date{\today}
\author{Michael Brodskiy\\ \small Professor: I. Salama}

\begin{document}

\maketitle

\begin{enumerate}

  \item Express each of the following complex numbers in polar form and plot them

    \begin{enumerate}

      \item $8$

        $$r=\sqrt{8^2+0^2}=8$$
        $$\theta=0$$
        $$z(r,\theta)=r(\cos(\theta)+j\sin(\theta))$$
        $$z(8,0)=8(\cos(0)+j\sin(0))$$
        $$\therefore \text{ In polar: } \boxed{z=8}$$

      \item $-5$

        $$r=\sqrt{(-5)^2+0^2}=5$$
        $$\theta=\pi$$
        $$z(r,\theta)=r(\cos(\theta)+j\sin(\theta))$$
        $$z(5,\pi)=5(\cos(\pi)+j\sin(\pi))$$
        $$\therefore \text{ In polar: } \boxed{z=-5=5e^{\pi j}}$$

      \item $2j$

        $$r=\sqrt{0^2+(2)^2}=2$$
        $$\theta=\frac{\pi}{2}$$
        $$z(r,\theta)=r(\cos(\theta)+j\sin(\theta))$$
        $$z(2,.5\pi)=2j$$
        $$\therefore \text{ In polar: } \boxed{z=2j=2e^{.5\pi j}}$$

      \item $\frac{1}{4}(1-j)^5$

        $$.25(1-j)^2(1-j)^3$$
        $$.25(2-2j)(1-j)(1-j)^2$$
        $$.25(4-4j)(2-2j)$$
        $$z=4-4j$$
        $$r=\sqrt{4^2+(-4)^2}=4\sqrt{2}$$
        $$\theta=-\frac{\pi}{4}$$
        $$z(r,\theta)=r(\cos(\theta)+j\sin(\theta))$$
        $$z(4\sqrt{2},-.25\pi)=4-4j$$
        $$\therefore \text{ In polar: } \boxed{z=4-4j=4\sqrt{2}e^{-.25\pi j}}$$

      \item $\frac{(1+j)}{j}e^{\frac{j\pi}{3}}$

        $$\frac{(1+j)}{j}\cdot\frac{-j}{-j}=1-j$$
        $$\tan\left( \frac{\pi}{3} \right)=\frac{b}{a}$$
        $$\frac{b}{a}=\sqrt{3}$$
        $$b=a\sqrt{3}$$
        $$\sqrt{(a\sqrt{3})^2+a^2}=1$$
        $$4a^2=\pm1$$
        $$a=\frac{1}{2}$$
        $$b=\frac{\sqrt{3}}{2}$$
        $$\frac{1}{2}(1-j)(1+\sqrt{3}j)\to\frac{1}{2}((\sqrt{3}+1)+(\sqrt{3}-1)j)$$
        $$r=\frac{1}{2}\sqrt{(\sqrt{3}+1)^2+(\sqrt{3}-1)^2}=\sqrt{(4+2\sqrt{3})+(4-2\sqrt{3})}$$
        $$r=\sqrt{2}$$
        $$\theta=\tan^{-1}\left( \frac{\sqrt{3}-1}{\sqrt{3}+1} \right)=.26179$$
        $$z(r,\theta)=r(\cos(\theta)+j\sin(\theta))$$
        $$z(\sqrt{2},.26179)=\frac{1}{2}(\sqrt{3}+1)+(\sqrt{3}-1)j$$
        $$\therefore \text{ In polar: } \boxed{z=\frac{1}{2}(\sqrt{3}+1)+(\sqrt{3}-1)j=\sqrt{2}e^{.26179j}}$$

      \item $(\sqrt{3}-j^5)(1+j)$

        $$j^5=j\to (\sqrt{3}-j)(1+j)=(\sqrt{3}+(\sqrt{3}-1)j+1)$$
        $$(\sqrt{3}+1)+(\sqrt{3}-1)j$$
        $$r=\sqrt{(\sqrt{3}+1)^2+(\sqrt{3}-1)^2}=\sqrt{(4+2\sqrt{3})+(4-2\sqrt{3})}$$
        $$r=\sqrt{8}=2\sqrt{2}$$
        $$\theta=\tan^{-1}\left( \frac{\sqrt{3}-1}{\sqrt{3}+1} \right)=.26179$$
        $$z(r,\theta)=r(\cos(\theta)+j\sin(\theta))$$
        $$z(2\sqrt{2},.26179)=(\sqrt{3}+1)+(\sqrt{3}-1)j$$
        $$\therefore \text{ In polar: } \boxed{z=(\sqrt{3}+1)+(\sqrt{3}-1)j=2\sqrt{2}e^{.26179j}}$$

      \item $\frac{2(\sqrt{3}-j)}{1+j\sqrt{3}}$

        $$\frac{2\sqrt{3}-2j}{1+j\sqrt{3}}\cdot\frac{1-j\sqrt{3}}{1-j\sqrt{3}}=-2j$$
        $$r=\sqrt{0^2+(-2)^2}=2$$
        $$\theta=\frac{3\pi}{2}$$
        $$z(r,\theta)=r(\cos(\theta)+j\sin(\theta))$$
        $$z(2,1.5\pi)=-2j$$
        $$\therefore \text{ In polar: } \boxed{z=-2j=2e^{1.5\pi j}}$$

    \end{enumerate}

  \item Determine the value of $E_{\infty}$ and $P_{\infty}$ for each of the following signals and indicate whether the signal is a power or energy signal or neither.

    \begin{enumerate}

      \item $x_1(t)=\left\{\begin{array}{l r} 5e^{j(4t+\pi/3)},\, & t\geq 2\\ 0,\, & \text{Otherwise}\end{array}$

      \item $x_2(t)=\left\{\begin{array}{l r} 2+2\cos(t),\, & 0<t< 2\pi\\ 0,\, & \text{Otherwise}\end{array}$

        \item $x_3[n]=\left\{\begin{array}{l r} (.5)^n,\, & n\geq0\\ 0,\, & \text{Otherwise}\end{array}$

    \end{enumerate}

  \item For the discrete time signal shown in Figure P1.3, sketch, and carefully label each of the following.

    \begin{enumerate}

      \item $x[n-4]$

      \item $x[2n+2]$

    \end{enumerate}

  \item For the continuous time signal shown in Figure P1.4, sketch, and carefully label each of the following.

    \begin{enumerate}

      \item $x(t+3)$

      \item $x\left( 3-\frac{2}{3}t \right)$

    \end{enumerate}

  \item Determine and sketch the even and odd parts of the signals depicted in Figure P1.5. Label your sketches carefully.

    \begin{enumerate}

      \item 

      \item 

    \end{enumerate}

  \item Determine and sketch the even and odd parts of the signal depicted in Figure P1.6. Label your sketches carefully.

  \item Express the real part of each of the following signals in the form $Ae^{-at}\cos(\omega t + \phi)$ where $A$, $a$, $\omega$ and $\phi$ are real numbers with $A > 0$ and $-\pi < \phi \leq \pi$.

    \begin{enumerate}

      \item $x_1(t)=4e^{-2t}\sin\left( 10t+\frac{3\pi}{4}\right)\cos\left( 10t+\frac{3\pi}{4} \right)$

      \item $x_2(t)=j(1-j)e^{(-5+j\pi)t}$

    \end{enumerate}

  \item Determine whether each of the following continuous time signals is periodic. If the signal is periodic, determine its fundamental period.

    \begin{enumerate}

      \item $x(t)=5\cos\left( 400\pi t+\frac{\pi}{4} \right)$

      \item $x(t)=20e^{j(\pi t-2)}$

      \item $x(t)=2\left[ \sin\left( 50\pi t - \frac{\pi}{3} \right) \right]^2$

      \item $x(t)=\left\{\begin{array}{l r} 2\sin(5\pi t),\, & t\geq 0\\ -2\sin(-5\pi t),\, &t<0\end{array}$

    \end{enumerate}

  \item Determine whether each of the following discrete time signals is periodic. If the signal is periodic, determine its fundamental period.

    \begin{enumerate}

      \item $x[n]=2\cos\left( \frac{7}{11}n+\frac{\pi}{2} \right)$

      \item $x[n]=\cos(\pi n)+4\sin\left( \frac{\pi}{4}n^2 \right)$

      \item $x[n]=3\sin\left( \frac{\pi}{3}n \right)+\cos\left( \frac{\pi}{4}n \right)-3\cos\left( \frac{\pi}{6}n+\frac{\pi}{3} \right)$ 

    \end{enumerate}

\end{enumerate}

\end{document}

