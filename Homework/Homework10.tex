%%%%%%%%%%%%%%%%%%%%%%%%%%%%%%%%%%%%%%%%%%%%%%%%%%%%%%%%%%%%%%%%%%%%%%%%%%%%%%%%%%%%%%%%%%%%%%%%%%%%%%%%%%%%%%%%%%%%%%%%%%%%%%%%%%%%%%%%%%%%%%%%%%%%%%%%%%%%%%%%%%%
% Written By Michael Brodskiy
% Class: Fundamentals of Linear Systems
% Professor: I. Salama
%%%%%%%%%%%%%%%%%%%%%%%%%%%%%%%%%%%%%%%%%%%%%%%%%%%%%%%%%%%%%%%%%%%%%%%%%%%%%%%%%%%%%%%%%%%%%%%%%%%%%%%%%%%%%%%%%%%%%%%%%%%%%%%%%%%%%%%%%%%%%%%%%%%%%%%%%%%%%%%%%%%

\include{Includes.tex}

\title{Homework 10}
\date{\today}
\author{Michael Brodskiy\\ \small Professor: I. Salama}

\begin{document}

\maketitle

\begin{enumerate}

  \item We first determine the sampling frequency to be:

    $$f_s=\frac{1}{10^{-4}}=10[\si{\kilo\hertz}]$$
    $$\omega_s=2\pi\cdot10^4\left[\si{rad\over\sec}\right]$$

    \begin{enumerate}

      \item Since the signal band is limited to $5000\pi[\si{rad\per\second}]$, the Nyquist frequency is:

        $$\omega_N=2\omega$$
        $$\omega_N=10000\pi$$

        Since the sampling frequency is greater than the Nyquist frequency, \underline{the signal} \underline{can be fully recovered}
        
      \item Since the signal band is limited to $12000\pi[\si{rad\per\second}]$, the Nyquist frequency is:

        $$\omega_N=24000\pi$$

        Since the sampling frequency is less than the Nyquist frequency, \underline{the signal can not} \underline{be fully recovered}

      \item Since we only know the behavior of the real part of the signal, we do not know if the signal is band-limited. Thus, the sampling theorem \underline{can not guarantee exact recovery}
        
      \item Since the signal is real, $X(j\omega)$ must be symmetrical, and, therefore, the signal is band-limited to $5000\pi[\si{rad\per\second}]$. This means that, for the same reasoning as (a), the signal \underline{can be fully recovered}
        
      \item Similarly, since the signal is real, we know that $X(j\omega)$ must be symmetrical, and that, per the same reasoning as (b), it \underline{can not be exactly recovered}
        
      \item We know that, if $X(j\omega)=0$ for $|\omega|=\omega_1$, then $X(j\omega)*X(j\omega)=0$ for $|\omega|>2\omega$. Thus, we may write:

        $$X(j\omega)=0\text{  for  }|\omega|>6000\pi$$

        This means that the Nyquist frequency is:

        $$\omega_N=12000\pi$$

        Since the Nyquist frequency is less than the sampling frequency, we know that the signal \underline{can be fully recovered}
        
      \item From this, we may determine that, because:

        $$|X(j\omega)|=0\text{  for  }|\omega|>6000\pi$$

        The signal \underline{can be fully recovered}, for the same reasoning as part (f).
        
    \end{enumerate}

  \item

    \begin{enumerate}

      \item The maximum frequency of the signal, due to the multiplication in the frequency domain, becomes the minimum of the two bandwidths, such that:

        $$\omega=2000\pi$$

        Thus, the Nyquist frequency becomes:

        $$\omega_N=4000\pi$$

        This means that, to fully recover the signal, the period must be:

        $$T\leq \frac{2\pi}{4000\pi}$$
        $$T\leq \frac{1}{2000}$$
        $$\boxed{T\leq .5[\si{\milli\second}]}$$

      \item Multiplication leads to convolution in the frequency domain, which means that the bandwidth now limits the frequency to a maximum of:

        $$\omega=2000\pi+4000\pi$$
        $$\omega=6000\pi$$

        The Nyquist frequency becomes:

        $$\omega_N=12000\pi$$

        This gives us a period range of:

        $$T\leq\frac{2\pi}{12000\pi}$$
        $$T\leq\frac{1}{6000}$$
        $$\boxed{T\leq.166\bar{6}[\si{\milli\second}]}$$

    \end{enumerate}

  \item Per the Nyquist criterion for bandpass signals, we know that the sampling frequency must be twice the bandwidth of the signal. Thus, we may find:

    $$\omega_N=2(\omega_2-\omega_1)$$

    This gives us a maximum period of:

    $$T\leq\frac{2\pi}{2(\omega_2-\omega_1)}$$
    $$\boxed{T\leq\frac{\pi}{\omega_2-\omega_1}}$$

    To accurately recover the signal, we should take the parameters as:

    $$\boxed{\left\{\begin{array}{ll}\omega_a &= \omega_1\\\omega_b&= \omega_2 \end{array}}$$

    This allows us to accurately obtain the initial signal. Furthermore, since we want the original signal, we want a gain of unity, or:

    $$\boxed{A=1}$$

  \item

    \begin{enumerate}

      \item We may sketch $X_p(j\omega)$ as:

        \begin{figure}[H]
          \centering
          \tikzset{every picture/.style={line width=0.75pt}} %set default line width to 0.75pt        

\begin{tikzpicture}[x=0.75pt,y=0.75pt,yscale=-1,xscale=1]
%uncomment if require: \path (0,393); %set diagram left start at 0, and has height of 393

%Shape: Axis 2D [id:dp1655481678492966] 
\draw  (297,257.1) -- (631,257.1)(330.4,33) -- (330.4,282) (624,252.1) -- (631,257.1) -- (624,262.1) (325.4,40) -- (330.4,33) -- (335.4,40)  ;
%Shape: Axis 2D [id:dp38625059588045074] 
\draw  (363.8,257.1) -- (29.8,257.1)(330.4,33) -- (330.4,282) (36.8,252.1) -- (29.8,257.1) -- (36.8,262.1) (335.4,40) -- (330.4,33) -- (325.4,40)  ;
%Shape: Rectangle [id:dp7545526571720483] 
\draw   (330.4,199.1) -- (417,199.1) -- (417,257.1) -- (330.4,257.1) -- cycle ;
%Shape: Rectangle [id:dp1161158803287109] 
\draw   (496.4,199.1) -- (583,199.1) -- (583,257.1) -- (496.4,257.1) -- cycle ;
%Shape: Rectangle [id:dp18357778438310013] 
\draw   (77.8,199.1) -- (164.4,199.1) -- (164.4,257.1) -- (77.8,257.1) -- cycle ;

% Text Node
\draw (631,257.4) node [anchor=north west][inner sep=0.75pt]    {$\omega $};
% Text Node
\draw (306,12.4) node [anchor=north west][inner sep=0.75pt]    {$X_{p}( j\omega )$};
% Text Node
\draw (332.4,260.5) node [anchor=north west][inner sep=0.75pt]    {$0$};
% Text Node
\draw (417,260.5) node [anchor=north] [inner sep=0.75pt]    {$\frac{\omega _{2} -\omega _{1}}{2}$};
% Text Node
\draw (496.4,260.5) node [anchor=north] [inner sep=0.75pt]    {$\frac{\omega _{2} +\omega _{1}}{2}$};
% Text Node
\draw (583,260.5) node [anchor=north] [inner sep=0.75pt]    {$\omega _{2}$};
% Text Node
\draw (164.4,260.5) node [anchor=north] [inner sep=0.75pt]    {$\frac{-\omega _{2} -\omega _{1}}{2}$};
% Text Node
\draw (77.8,260.5) node [anchor=north] [inner sep=0.75pt]    {$-\omega _{2}$};


\end{tikzpicture}

          \caption{Sketch of $X_p(j\omega)$}
          \label{fig:1}
        \end{figure}

      \item Given that the signal is limited to $\omega_2$, we can take the minimum sampling frequency as:

        $$\boxed{\omega_s\geq 2\omega_2}$$

      \item Given that $X(j\omega)$ is not present within $X_p(j\omega)$, the original signal can not be recovered using any kind of filter. Thus, this is a loss of information.

    \end{enumerate}

  \item

    \begin{enumerate}

      \item We may sketch the given signals as:

        \begin{figure}[H]
          \centering
          \tikzset{every picture/.style={line width=0.75pt}} %set default line width to 0.75pt        

\begin{tikzpicture}[x=0.75pt,y=0.75pt,yscale=-1,xscale=1]
%uncomment if require: \path (0,393); %set diagram left start at 0, and has height of 393

%Shape: Axis 2D [id:dp1655481678492966] 
\draw  (281,257.1) -- (615,257.1)(314.4,33) -- (314.4,282) (608,252.1) -- (615,257.1) -- (608,262.1) (309.4,40) -- (314.4,33) -- (319.4,40)  ;
%Shape: Axis 2D [id:dp38625059588045074] 
\draw  (347.8,257.1) -- (13.8,257.1)(314.4,33) -- (314.4,282) (20.8,252.1) -- (13.8,257.1) -- (20.8,262.1) (319.4,40) -- (314.4,33) -- (309.4,40)  ;
%Straight Lines [id:da9294018403143072] 
\draw    (245.98,257.1) -- (314.8,59) ;
%Straight Lines [id:da9740337393831212] 
\draw    (387.07,255.87) -- (314.8,59) ;
%Straight Lines [id:da9384743114749254] 
\draw    (587.29,257.1) -- (515.02,60.23) ;
%Straight Lines [id:da5891124578320841] 
\draw    (446.2,258.33) -- (515.02,60.23) ;
%Straight Lines [id:da1820199963240804] 
\draw    (184.29,256.1) -- (112.02,59.23) ;
%Straight Lines [id:da7845075875180415] 
\draw    (43.2,257.33) -- (112.02,59.23) ;
%Straight Lines [id:da45856677787947864] 
\draw [color={rgb, 255:red, 155; green, 155; blue, 155 }  ,draw opacity=1 ] [dash pattern={on 4.5pt off 4.5pt}]  (112.02,59.23) -- (515.02,60.23) ;

% Text Node
\draw (606,229.4) node [anchor=north west][inner sep=0.75pt]    {$\omega \cdot 10^{4}$};
% Text Node
\draw (290,12.4) node [anchor=north west][inner sep=0.75pt]    {$X_{p}( j\omega )$};
% Text Node
\draw (316.4,260.5) node [anchor=north west][inner sep=0.75pt]    {$0$};
% Text Node
\draw (245.98,260.5) node [anchor=north] [inner sep=0.75pt]    {$-2\pi $};
% Text Node
\draw (387.07,259.27) node [anchor=north] [inner sep=0.75pt]    {$2\pi $};
% Text Node
\draw (451.24,260.5) node [anchor=north] [inner sep=0.75pt]    {$4\pi $};
% Text Node
\draw (587.29,260.5) node [anchor=north] [inner sep=0.75pt]    {$8\pi $};
% Text Node
\draw (177.56,260.5) node [anchor=north] [inner sep=0.75pt]    {$-4\pi $};
% Text Node
\draw (41.51,260.5) node [anchor=north] [inner sep=0.75pt]    {$-8\pi $};
% Text Node
\draw (315,43.4) node [anchor=north west][inner sep=0.75pt]    {$30k$};


\end{tikzpicture}

          \caption{Sketch of $X_p(j\omega)$}
          \label{fig:2}
        \end{figure}

        We may see that the signal is constrained by:

        $$x(n)=x_c(nT)$$

        We can continue to sketch $X(e^{j\Omega})$, which looks quite similar; however, we first convert from analog to digital frequency:

        $$\Omega=\frac{\omega}{f_s}$$
        $$\Omega=\frac{\omega}{3\cdot10^4}$$

        This gives us:
        
        \begin{figure}[H]
          \centering
          \include{Figures/HW10-5a2}
          \caption{Sketch of $X(e^{j\Omega})$}
          \label{fig:3}
        \end{figure}

        From here, we may obtain $Y(e^{j\Omega})$ as:

        \begin{figure}[H]
          \centering
          \include{Figures/HW10-5a3}
          \caption{Sketch of $Y(e^{j\Omega})$}
          \label{fig:4}
        \end{figure}

        We convert back to digital frequency to get:

        \begin{figure}[H]
          \centering
          \tikzset{every picture/.style={line width=0.75pt}} %set default line width to 0.75pt        

\begin{tikzpicture}[x=0.75pt,y=0.75pt,yscale=-1,xscale=1]
%uncomment if require: \path (0,393); %set diagram left start at 0, and has height of 393

%Shape: Axis 2D [id:dp1655481678492966] 
\draw  (281,257.1) -- (615,257.1)(314.4,33) -- (314.4,282) (608,252.1) -- (615,257.1) -- (608,262.1) (309.4,40) -- (314.4,33) -- (319.4,40)  ;
%Shape: Axis 2D [id:dp38625059588045074] 
\draw  (347.8,257.1) -- (13.8,257.1)(314.4,33) -- (314.4,282) (20.8,252.1) -- (13.8,257.1) -- (20.8,262.1) (319.4,40) -- (314.4,33) -- (309.4,40)  ;
%Straight Lines [id:da9384743114749254] 
\draw    (552.1,158.55) -- (515.02,60.23) ;
%Straight Lines [id:da45856677787947864] 
\draw [color={rgb, 255:red, 155; green, 155; blue, 155 }  ,draw opacity=1 ] [dash pattern={on 4.5pt off 4.5pt}]  (115.18,59.23) -- (515.02,60.23) ;
%Straight Lines [id:da7336512866059715] 
\draw    (314.8,59) -- (314.4,257.1) ;
%Straight Lines [id:da5107354205792817] 
\draw [color={rgb, 255:red, 155; green, 155; blue, 155 }  ,draw opacity=1 ] [dash pattern={on 4.5pt off 4.5pt}]  (78.1,157.55) -- (552.1,158.55) ;
%Straight Lines [id:da7473212892139033] 
\draw    (477.93,158.55) -- (515.02,60.23) ;
%Straight Lines [id:da18274923539856058] 
\draw    (276.43,158.05) -- (313.52,59.73) ;
%Straight Lines [id:da08038296278804791] 
\draw    (351.88,157.32) -- (314.8,59) ;
%Straight Lines [id:da5042495825394699] 
\draw    (78.1,157.55) -- (115.18,59.23) ;
%Straight Lines [id:da3695599935945484] 
\draw    (152.27,157.55) -- (115.18,59.23) ;
%Straight Lines [id:da6958486433791145] 
\draw [color={rgb, 255:red, 155; green, 155; blue, 155 }  ,draw opacity=1 ] [dash pattern={on 4.5pt off 4.5pt}]  (78.1,157.55) -- (78.1,257.97) ;
%Straight Lines [id:da023465556705617896] 
\draw [color={rgb, 255:red, 155; green, 155; blue, 155 }  ,draw opacity=1 ] [dash pattern={on 4.5pt off 4.5pt}]  (152.27,157.55) -- (152.27,257.97) ;
%Straight Lines [id:da3405622250433783] 
\draw [color={rgb, 255:red, 155; green, 155; blue, 155 }  ,draw opacity=1 ] [dash pattern={on 4.5pt off 4.5pt}]  (276.43,158.05) -- (276.43,258.47) ;
%Straight Lines [id:da850803385131168] 
\draw [color={rgb, 255:red, 155; green, 155; blue, 155 }  ,draw opacity=1 ] [dash pattern={on 4.5pt off 4.5pt}]  (351.88,157.32) -- (351.88,257.74) ;
%Straight Lines [id:da8565069393910767] 
\draw [color={rgb, 255:red, 155; green, 155; blue, 155 }  ,draw opacity=1 ] [dash pattern={on 4.5pt off 4.5pt}]  (477.93,158.55) -- (477.93,258.97) ;
%Straight Lines [id:da6530573331663155] 
\draw [color={rgb, 255:red, 155; green, 155; blue, 155 }  ,draw opacity=1 ] [dash pattern={on 4.5pt off 4.5pt}]  (552.1,158.55) -- (552.1,258.97) ;

% Text Node
\draw (592,232.4) node [anchor=north west][inner sep=0.75pt]    {$\omega \cdot 10^{4}$};
% Text Node
\draw (290,12.4) node [anchor=north west][inner sep=0.75pt]    {$Y_{p}( j\omega )$};
% Text Node
\draw (316.4,260.5) node [anchor=north west][inner sep=0.75pt]    {$0$};
% Text Node
\draw (487,41.4) node [anchor=north west][inner sep=0.75pt]    {$30k$};
% Text Node
\draw (276.43,261.87) node [anchor=north] [inner sep=0.75pt]    {$\frac{-3\pi }{4}$};
% Text Node
\draw (554.1,161.95) node [anchor=north west][inner sep=0.75pt]    {$18.75k$};
% Text Node
\draw (351.88,261.14) node [anchor=north] [inner sep=0.75pt]    {$\frac{3\pi }{4}$};
% Text Node
\draw (78.1,261.37) node [anchor=north] [inner sep=0.75pt]    {$\frac{-27\pi }{4}$};
% Text Node
\draw (152.27,261.37) node [anchor=north] [inner sep=0.75pt]    {$\frac{-21\pi }{4}$};
% Text Node
\draw (477.93,262.37) node [anchor=north] [inner sep=0.75pt]    {$\frac{21\pi }{4}$};
% Text Node
\draw (552.1,262.37) node [anchor=north] [inner sep=0.75pt]    {$\frac{27\pi }{4}$};


\end{tikzpicture}

          \caption{Sketch of $Y_p(j\omega)$}
          \label{fig:5}
        \end{figure}

        Passed through the filter, our response becomes:

        \begin{figure}[H]
          \centering
          \tikzset{every picture/.style={line width=0.75pt}} %set default line width to 0.75pt        

\begin{tikzpicture}[x=0.75pt,y=0.75pt,yscale=-1,xscale=1]
%uncomment if require: \path (0,393); %set diagram left start at 0, and has height of 393

%Shape: Axis 2D [id:dp1655481678492966] 
\draw  (281,257.1) -- (615,257.1)(314.4,33) -- (314.4,282) (608,252.1) -- (615,257.1) -- (608,262.1) (309.4,40) -- (314.4,33) -- (319.4,40)  ;
%Shape: Axis 2D [id:dp38625059588045074] 
\draw  (347.8,257.1) -- (13.8,257.1)(314.4,33) -- (314.4,282) (20.8,252.1) -- (13.8,257.1) -- (20.8,262.1) (319.4,40) -- (314.4,33) -- (309.4,40)  ;
%Straight Lines [id:da7336512866059715] 
\draw    (314.8,59) -- (314.4,257.1) ;
%Straight Lines [id:da18274923539856058] 
\draw    (276.43,158.05) -- (313.52,59.73) ;
%Straight Lines [id:da08038296278804791] 
\draw    (351.88,157.32) -- (314.8,59) ;
%Straight Lines [id:da3405622250433783] 
\draw [color={rgb, 255:red, 155; green, 155; blue, 155 }  ,draw opacity=1 ] [dash pattern={on 4.5pt off 4.5pt}]  (276.43,158.05) -- (276.43,258.47) ;
%Straight Lines [id:da850803385131168] 
\draw [color={rgb, 255:red, 155; green, 155; blue, 155 }  ,draw opacity=1 ] [dash pattern={on 4.5pt off 4.5pt}]  (351.88,157.32) -- (351.88,257.74) ;

% Text Node
\draw (592,232.4) node [anchor=north west][inner sep=0.75pt]    {$\omega \cdot 10^{4}$};
% Text Node
\draw (290,12.4) node [anchor=north west][inner sep=0.75pt]    {$Y_{c}( j\omega )$};
% Text Node
\draw (316.4,260.5) node [anchor=north west][inner sep=0.75pt]    {$0$};
% Text Node
\draw (276.43,261.87) node [anchor=north] [inner sep=0.75pt]    {$\frac{-3\pi }{4}$};
% Text Node
\draw (351.88,261.14) node [anchor=north] [inner sep=0.75pt]    {$\frac{3\pi }{4}$};
% Text Node
\draw (315.52,56.33) node [anchor=south west] [inner sep=0.75pt]    {$1$};
% Text Node
\draw (353.88,153.92) node [anchor=south west] [inner sep=0.75pt]    {$.625$};


\end{tikzpicture}

          \caption{Sketch of $Y_c(j\omega)$}
          \label{fig:6}
        \end{figure}

        The frequency response of the continuous time equivalent would be:

        $$\omega=\Omega f_s$$
        $$\omega=\left( \frac{\pi}{4} \right)\left( 3\cdot10^4 \right)$$
        $$\boxed{\omega=7500\pi}$$

        This gives us:

        \begin{figure}[H]
          \centering
          \include{Figures/HW10-5a6}
          \caption{Continuous Time Response}
          \label{fig:7}
        \end{figure}

      \item Only the first sinusoid is within the bandpass of the filter, which lets us determine:

        $$\boxed{y(t)=2\cos(4500\pi t+\pi/2)}$$
        
        When converted to a discrete sequence, we may observe that the sine term becomes:

        $$\Omega_N=6\pi\cdot10^4$$
        $$6\pi\cdot10^4-54000\pi=6000\pi$$

        This becomes aliased, which results in a discrete time result of:

        $$\boxed{y(t)=2\cos(4500\pi t+ \pi/2)+\sin(6000\pi t)}$$

        Thus, aliasing gives us a different result in discrete time.

    \end{enumerate}

\end{enumerate}

\end{document}

