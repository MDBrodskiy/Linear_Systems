%%%%%%%%%%%%%%%%%%%%%%%%%%%%%%%%%%%%%%%%%%%%%%%%%%%%%%%%%%%%%%%%%%%%%%%%%%%%%%%%%%%%%%%%%%%%%%%%%%%%%%%%%%%%%%%%%%%%%%%%%%%%%%%%%%%%%%%%%%%%%%%%%%%%%%%%%%%%%%%%%%%
% Written By Michael Brodskiy
% Class: Fundamentals of Linear Systems
% Professor: I. Salama
%%%%%%%%%%%%%%%%%%%%%%%%%%%%%%%%%%%%%%%%%%%%%%%%%%%%%%%%%%%%%%%%%%%%%%%%%%%%%%%%%%%%%%%%%%%%%%%%%%%%%%%%%%%%%%%%%%%%%%%%%%%%%%%%%%%%%%%%%%%%%%%%%%%%%%%%%%%%%%%%%%%

\documentclass[12pt]{article} 
\usepackage{alphalph}
\usepackage[utf8]{inputenc}
\usepackage[russian,english]{babel}
\usepackage{titling}
\usepackage{amsmath}
\usepackage{graphicx}
\usepackage{enumitem}
\usepackage{amssymb}
\usepackage[super]{nth}
\usepackage{everysel}
\usepackage{ragged2e}
\usepackage{geometry}
\usepackage{multicol}
\usepackage{fancyhdr}
\usepackage{cancel}
\usepackage{siunitx}
\usepackage{physics}
\usepackage{tikz}
\usepackage{mathdots}
\usepackage{yhmath}
\usepackage{cancel}
\usepackage{color}
\usepackage{array}
\usepackage{multirow}
\usepackage{gensymb}
\usepackage{tabularx}
\usepackage{extarrows}
\usepackage{booktabs}
\usepackage{lastpage}
\usetikzlibrary{fadings}
\usetikzlibrary{patterns}
\usetikzlibrary{shadows.blur}
\usetikzlibrary{shapes}

\geometry{top=1.0in,bottom=1.0in,left=1.0in,right=1.0in}
\newcommand{\subtitle}[1]{%
  \posttitle{%
    \par\end{center}
    \begin{center}\large#1\end{center}
    \vskip0.5em}%

}
\usepackage{hyperref}
\hypersetup{
colorlinks=true,
linkcolor=blue,
filecolor=magenta,      
urlcolor=blue,
citecolor=blue,
}


\title{Homework 4}
\date{\today}
\author{Michael Brodskiy\\ \small Professor: I. Salama}

\begin{document}

\maketitle

\begin{enumerate}

  \item

    \begin{enumerate}

      \item We begin by taking the Laplace transform to get:

        $$H(s)=-\frac{1}{s-1}\quad\text{ and }\quad X(s)=\frac{2}{s}\left[  e^{-s}-e^{-2s}\right]$$

        We then multiply the two to get:

        $$Y(s)=X(s)H(s)$$
        $$Y(s)=-\frac{2}{s(s-1)}\left[ e^{-s}-e^{-2s} \right]$$

        Using partial fraction decomposition, we may write the equivalent such that:

        $$-\frac{2}{s(s-1)}=\frac{A}{s-1}+\frac{B}{s}=As+B(s-1)$$

        We use $s=0,1$ to get:

        $$A=-2,\,B=2\to -\frac{2}{s-1}+\frac{2}{s}$$

        We now distribute this in the above case to get:

        $$Y(s)=-\frac{2}{s-1}\left[ e^{-s}-e^{-2s} \right]+\frac{2}{s}\left[ e^{-s}-e^{-2s} \right]$$
        $$Y(s)=-\frac{2e^{-s}}{s-1}+\frac{2e^{-2s}}{s-1}+\frac{2e^{-s}}{s}-\frac{2e^{-2s}}{s}$$

        Finally, we take the inverse transform to get:

        $$\boxed{y(t)=-2u(1-t)+2e^{t-1}u(1-t)+2u(2-t)-2e^{t-2}u(2-t)}$$

      \item 

        Differentiating one of the inputs is the same as differentiating the output. Thus, we may say:

        $$g(t)=\frac{d}{dt}[y(t)]$$
        $$\boxed{g(t)=2e^{t-1}u(1-t)-2e^{t-2}u(2-t)}$$

      \item As stated in (b) \underline{$g(t)=(d/dt)[y(t)]$}

      \item $z(t)$ is the same as $g(t)$. Since taking the differential is a linear operation, it does not matter if this is done to the impulse response or to $x(t)$. Therefore, we get:

      \item By linearity of the transform, we may say:

        $$y_1(t)=2y(t-1)$$

        Therefore, we may obtain:

        $$\boxed{y_1(t)=-4u(2-t)+4e^{t-2}u(2-t)+4u(3-t)-4e^{t-3}u(3-t)}$$

    \end{enumerate}

  \item

    \begin{enumerate}

      \item 

      \item 

      \item 

      \item 

    \end{enumerate}

  \item

    \begin{enumerate}

      \item 

      \item 

      \item 

    \end{enumerate}

  \item

    \begin{enumerate}

      \item 

      \item 

      \item 

      \item 

    \end{enumerate}

  \item

    \begin{enumerate}

      \item 

      \item 

      \item 

      \item 

    \end{enumerate}

  \item

    \begin{enumerate}

      \item 

      \item 

    \end{enumerate}

  \item

    \begin{enumerate}

      \item 

      \item 

    \end{enumerate}

  \item

\end{enumerate}

\end{document}

