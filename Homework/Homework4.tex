%%%%%%%%%%%%%%%%%%%%%%%%%%%%%%%%%%%%%%%%%%%%%%%%%%%%%%%%%%%%%%%%%%%%%%%%%%%%%%%%%%%%%%%%%%%%%%%%%%%%%%%%%%%%%%%%%%%%%%%%%%%%%%%%%%%%%%%%%%%%%%%%%%%%%%%%%%%%%%%%%%%
% Written By Michael Brodskiy
% Class: Fundamentals of Linear Systems
% Professor: I. Salama
%%%%%%%%%%%%%%%%%%%%%%%%%%%%%%%%%%%%%%%%%%%%%%%%%%%%%%%%%%%%%%%%%%%%%%%%%%%%%%%%%%%%%%%%%%%%%%%%%%%%%%%%%%%%%%%%%%%%%%%%%%%%%%%%%%%%%%%%%%%%%%%%%%%%%%%%%%%%%%%%%%%

\include{Includes.tex}

\title{Homework 4}
\date{\today}
\author{Michael Brodskiy\\ \small Professor: I. Salama}

\begin{document}

\maketitle

\begin{enumerate}

  \item

    \begin{enumerate}

      \item We begin by taking the Laplace transform to get:

        $$H(s)=-\frac{1}{s-1}\quad\text{ and }\quad X(s)=\frac{2}{s}\left[  e^{-s}-e^{-2s}\right]$$

        We then multiply the two to get:

        $$Y(s)=X(s)H(s)$$
        $$Y(s)=-\frac{2}{s(s-1)}\left[ e^{-s}-e^{-2s} \right]$$

        Using partial fraction decomposition, we may write the equivalent such that:

        $$-\frac{2}{s(s-1)}=\frac{A}{s-1}+\frac{B}{s}=As+B(s-1)$$

        We use $s=0,1$ to get:

        $$A=-2,\,B=2\to -\frac{2}{s-1}+\frac{2}{s}$$

        We now distribute this in the above case to get:

        $$Y(s)=-\frac{2}{s-1}\left[ e^{-s}-e^{-2s} \right]+\frac{2}{s}\left[ e^{-s}-e^{-2s} \right]$$
        $$Y(s)=-\frac{2e^{-s}}{s-1}+\frac{2e^{-2s}}{s-1}+\frac{2e^{-s}}{s}-\frac{2e^{-2s}}{s}$$

        Finally, we take the inverse transform to get:

        $$\boxed{y(t)=-2u(1-t)+2e^{t-1}u(1-t)+2u(2-t)-2e^{t-2}u(2-t)}$$

      \item 

        Differentiating one of the inputs is the same as differentiating the output. Thus, we may say:

        $$g(t)=\frac{d}{dt}[y(t)]$$
        $$\boxed{g(t)=2e^{t-1}u(1-t)-2e^{t-2}u(2-t)}$$

      \item As stated in (b) \underline{$g(t)=(d/dt)[y(t)]$}

      \item $z(t)$ is the same as $g(t)$. Since taking the differential is a linear operation, it does not matter if this is done to the impulse response or to $x(t)$. Therefore, we get:

      \item By linearity of the transform, we may say:

        $$y_1(t)=2y(t-1)$$

        Therefore, we may obtain:

        $$\boxed{y_1(t)=-4u(2-t)+4e^{t-2}u(2-t)+4u(3-t)-4e^{t-3}u(3-t)}$$

    \end{enumerate}

  \item

    \begin{enumerate}

      \item 

      \item 

      \item 

      \item 

    \end{enumerate}

  \item

    \begin{enumerate}

      \item 

        Given the set up, we may write:

        $$\left( \frac{1}{4} \right)^nu[n]-A\left( \frac{1}{4} \right)^{n-1}u[n-1]=\delta[n]$$

        We may redefine the delta as:

        $$\left( \frac{1}{4} \right)^nu[n]-A\left( \frac{1}{4} \right)^{n-1}u[n-1]=u[n]-u[n-1]$$

        Thus, we see that we need the exponential term to cancel. We can do this by simply taking:

        $$\left( \frac{1}{4} \right)^n=A\left( \frac{1}{4} \right)^{n-1}$$

        Dividing the exponential from one side to the other, we see:

        $$A=\left( \frac{1}{4} \right)^1$$
        $$\boxed{A=\frac{1}{4}}$$

      \item 

        By definition, with $h[n]$ and $g[n]=h_{inv}[n]$, we know:

        $$h[n]*g[n]=\delta[n]$$

        Using the equation from part (a), we know:

        $$h[n]-Ah[n-1]=\delta[n]$$

        By the properties of convolution, we know that:

        $$x[n]*\delta[n-n_o]=x[n-n_o]$$

        Thus, we may expand to write:

        $$h[n]*\delta[n]-Ah[n]*\delta[n-1]=\delta[n]$$
        $$h[n]*(\delta[n]-A\delta[n-1])=\delta[n]$$

        Thus, combining this with the definition of inverse, we may write:

        $$\boxed{g[n]=\delta[n]-\frac{1}{4}\delta[n-1]}$$

      \item 

    \end{enumerate}

  \item

    \begin{enumerate}

      \item 

      \item 

      \item 

      \item 

    \end{enumerate}

  \item

    \begin{enumerate}

      \item For the given system, we may see that, for $t<0$, the response may be non-zero (more precisely, it is non-zero for $-\infty< t<5$); therefore, the system is \underline{not causal}. We check for stability below:

        $$\int_{-\infty}^{5}e^{-3t}\,dt$$
        $$-\frac{e^{-3t}}{3}\Big|_{-\infty}^5=\infty$$

        Therefore, the system is \underline{not stable}

      \item We may see that, for the given system, for $t<0$, the response may be non-zero (more precisely, it is non-zero for $t>-10$); therefore, the system is \underline{not causal}. We check for stability below:

        $$\int_{-10}^{\infty}e^{-4t}\,dt$$
        $$-\frac{e^{-4t}}{4}\Big_{-10}^{\infty}=\frac{e^{40}}{4}<\infty$$

        Thus, we see that the system is \underline{stable}

      \item 

        We may rewrite the function as:

        $$x(t)=\left\{\begin{array}{ll} e^{-2t}, & t\geq0\\ e^{2t}, & t<0\end{array}$$

        Because the value of the function is non-zero when $t<0$, we can see that it is \underline{not causal}

        We may check for stability below:

        $$\int_{-\infty}^{0} e^{2t}\,dt + \int_0^{\infty} e^{-2t}\,dt$$
        $$\frac{e^{2t}}{2}\Big|_{-\infty}^0-\frac{e^{-2t}}{2}\Big|_0^{\infty}=1$$

        Therefore, we may see that the system is \underline{stable}

      \item We may see that, because the system is zero for $t\leq 0$, it \underline{is causal}. We now check for stability:

        $$\int_2^{\infty} 3e^{-2t}-e^{-.05t+5}\,dt$$
        $$3e^{-2t}-e^{-.05t+5}\Big|_2^{\infty}=e^{4.9}-\frac{3}{e^4}<\infty$$

        Therefore, we may see that the system is \underline{stable}

    \end{enumerate}

  \item

    \begin{enumerate}

      \item 

      \item 

    \end{enumerate}

  \item

    \begin{enumerate}

      \item Given the form of $x(t)$, we know $y(t)$ is of the form:

        $$y(t)=Ae^{(-1+2j)t}$$

        Plugging this into the given equation, we get:

        $$A(-1+2j)e^{(-1+2j)t}+3Ae^{(-1+2j)t}=e^{(-1+2j)t}$$

        This simplifies to:

        $$A(-1+2j)+3A=1$$
        $$A=\frac{1}{2j+2}$$

        And gives us the particular equation:

        $$y_p(t)=\frac{e^{(-1+2j)t}}{2j+2}$$

        We can now find the homogenous solution, with general form of $y(t)$:

        $$y_h(t)=Ae^{\lambda t}$$

        Using our equation, we may obtain:

        $$A\lambda e^{\lambda t}+3Ae^{\lambda t}=0$$
        $$\lambda=-3$$

        Combing the two we get:

        $$y(t)=Ae^{-3 t}+\frac{e^{(-1+2j)t}}{2j+2}$$

        Applying the initial rest condition, we get:

        $$0=A+\frac{1}{2j+2}$$
        $$A=-\frac{1}{2j+2}$$

        Thus, ensuring that there is a response only for $t>0$, we finally get:

        $$\boxed{y(t)=\frac{1}{2+2j}\left[ e^{(-1+2j)t}-e^{-3t} \right]u(t)}$$

      \item 

        We can expand our answer from (a):

        $$y(t)=\frac{1}{2+2j}\left[ e^{-t}\cos(2t)+je^{-t}\sin(2t)-e^{-3t} \right]u(t)$$

        Multiplying by the conjugate, we get:

        $$y(t)=(.25-.25j)\left[ e^{-t}\cos(2t)+je^{-t}\sin(2t)-e^{-3t} \right]u(t)$$
        $$y(t)=\left[ .25e^{-t}\cos(2t)+.25e^{-t}\sin(2t)-.25e^{-3t} \right]u(t)$$

        Therefore, our output becomes:

        $$\boxed{y(t)=\left[ .25e^{-t}\sin(2t)+.25e^{-t}\cos(2t)-.25e^{-3t}\right]u(t)}$$

    \end{enumerate}

  \item

    \begin{itemize}

      \item We know that the equations governing time response of an inductor and capacitor are (respectively):

        $$y(t)=L\frac{di(t)}{dt}\quad\text{ and }i(t)=C\frac{dV_c(t)}{dt}$$

        We know that the voltage across the capacitor will be the difference between the voltage supplied and the voltage across the inductor; thus, we may write:

        $$i(t)=C\frac{d}{dt}\left[ x(t)-y(t) \right]$$

        Inserting this into the inductor equation, we get:

        $$y(t)=L\frac{d}{dt}\left[ C\frac{d}{dt}\left[ x(t)-y(t) \right] \right]$$
        $$y(t)=LC\frac{d^2}{dt^2}\left[x(t)-y(t) \right]$$

        Putting similar terms to one side, we may write:

        $$\frac{d^2y(t)}{dt^2}+\frac{1}{LC}y(t)=\frac{d^2x(t)}{dt^2}$$

        Inserting known values:

        $$\boxed{\frac{d^2y(t)}{dt^2}+25y(t)=\frac{d^2x(t)}{dt^2}}$$

      \item 

        Taking $x(t)\to0$, we find the homogenous solution form as:

        $$\frac{d^2y(t)}{dt^2}+25y(t)=0$$

        Using the provided equation, we insert into the above:

        $$\left[ K_1\omega_1^2e^{j\omega_1 t}+K_2\omega_2^2e^{j\omega_2t} \right]+25\left[ K_1e^{j\omega_1 t}+K_2e^{j\omega_2t} \right]=0$$

        Dividing by the exponentials, we get:

        $$K_1\omega_1^2+K_2\omega_2^2+25K_1+25K_2=0$$

        By observation, we may see that:

        $$\boxed{\omega_1=\omega_2=\pm5}$$

      \item 

        From part (b), we see that the natural (homogenous) response may be modeled by:

        $$y(t)=Ae^{j5t}+Be^{-j5t}$$

        We may take $A=\frac{1}{2}(a+b)$ and $B=\frac{1}{2}(a-b)$. We then apply Euler's Law to get:

        $$\boxed{y(t)=a\cos(5t)+b\sin(5t)}$$

        We may thus observe that the natural response is sinusoidal in nature.

    \end{itemize}

\end{enumerate}

\end{document}

