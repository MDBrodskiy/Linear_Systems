%%%%%%%%%%%%%%%%%%%%%%%%%%%%%%%%%%%%%%%%%%%%%%%%%%%%%%%%%%%%%%%%%%%%%%%%%%%%%%%%%%%%%%%%%%%%%%%%%%%%%%%%%%%%%%%%%%%%%%%%%%%%%%%%%%%%%%%%%%%%%%%%%%%%%%%%%%%%%%%%%%%
% Written By Michael Brodskiy
% Class: Fundamentals of Linear Systems
% Professor: I. Salama
%%%%%%%%%%%%%%%%%%%%%%%%%%%%%%%%%%%%%%%%%%%%%%%%%%%%%%%%%%%%%%%%%%%%%%%%%%%%%%%%%%%%%%%%%%%%%%%%%%%%%%%%%%%%%%%%%%%%%%%%%%%%%%%%%%%%%%%%%%%%%%%%%%%%%%%%%%%%%%%%%%%

\documentclass[12pt]{article} 
\usepackage{alphalph}
\usepackage[utf8]{inputenc}
\usepackage[russian,english]{babel}
\usepackage{titling}
\usepackage{amsmath}
\usepackage{graphicx}
\usepackage{enumitem}
\usepackage{amssymb}
\usepackage[super]{nth}
\usepackage{everysel}
\usepackage{ragged2e}
\usepackage{geometry}
\usepackage{multicol}
\usepackage{fancyhdr}
\usepackage{cancel}
\usepackage{siunitx}
\usepackage{physics}
\usepackage{tikz}
\usepackage{mathdots}
\usepackage{yhmath}
\usepackage{cancel}
\usepackage{color}
\usepackage{array}
\usepackage{multirow}
\usepackage{gensymb}
\usepackage{tabularx}
\usepackage{extarrows}
\usepackage{booktabs}
\usepackage{lastpage}
\usetikzlibrary{fadings}
\usetikzlibrary{patterns}
\usetikzlibrary{shadows.blur}
\usetikzlibrary{shapes}

\geometry{top=1.0in,bottom=1.0in,left=1.0in,right=1.0in}
\newcommand{\subtitle}[1]{%
  \posttitle{%
    \par\end{center}
    \begin{center}\large#1\end{center}
    \vskip0.5em}%

}
\usepackage{hyperref}
\hypersetup{
colorlinks=true,
linkcolor=blue,
filecolor=magenta,      
urlcolor=blue,
citecolor=blue,
}


\title{Homework 5}
\date{\today}
\author{Michael Brodskiy\\ \small Professor: I. Salama}

\begin{document}

\maketitle

\begin{enumerate}

  \item

    \begin{enumerate}

      \item 

        We may begin by rewriting $S_2$ as:

        $$w[n]=y[n]-\frac{1}{2}y[n-1]$$

        $S_1$ may be rewritten in a similar format to get:

        $$x[n]=w[n]-\frac{1}{4}w[n-1]$$

        Substituting the first equation into the second, we get:

        $$x[n]=y[n]-\frac{1}{2}y[n-1]-\frac{1}{4}\left[ y[n-1]-\frac{1}{2}y[n-2] \right]$$

        This can be simplified to:

        $$\boxed{x[n]=y[n]-\frac{3}{4}y[n-1]+\frac{1}{8}y[n-2]}$$

      \item 

        We may find the impulse response by taking $x[n]\to\delta[n]$. This gives us:

        $$w[n]=\frac{1}{4}w[n-1]+\delta[n]$$

        Taking the $z$ transform, we may write:

        $$W(z)=\frac{1}{\left( 1-\frac{1}{4}z^{-1} \right)},\quad |z|>.25$$
        $$W(z)=\frac{z}{\left( z-\frac{1}{4} \right)},\quad |z|>.25$$
        $$\boxed{w[n]=\left( \frac{1}{4} \right)^{n}u[n]}$$

        Finding the individual impulse response for $S_2$, we get:

        $$Y(z)=\frac{1}{1-\frac{1}{2}z^{-1}},\quad |z|>.5$$
        $$Y(z)=\frac{z}{z-\frac{1}{2}},\quad |z|>.5$$
        $$\boxed{y[n]=\left( \frac{1}{2} \right)^nu[n]}$$

      \item 

        We can once again use the $z$ transform, this time using the equation obtained in (a), to get:

        $$Y(z)=\frac{1}{1-\frac{3}{4}z^{-1}+\frac{1}{8}z^{-2}}$$
        $$Y(z)=\frac{z^2}{z^2-\frac{3}{4}z+\frac{1}{8}}$$
        $$Y(z)=\frac{z^2}{\left( z-\frac{1}{2} \right)\left( z-\frac{1}{4} \right)},\quad |z|>.5$$

        We can use partial fraction decomposition, by rearranging:

        $$\frac{Y(z)}{z}=\frac{z}{\left( z-\frac{1}{2} \right)\left( z-\frac{1}{4} \right)},\quad |z|>.5$$
        $$\frac{Y(z)}{z}=\frac{A}{z-.25}+\frac{B}{z-.5}$$

        We may find: $A=-1$, $B=2$, which gives us:

        $$Y(z)=\frac{-z}{z-.25}+\frac{2z}{z-.5},\quad |z|>.5$$

        We take the inverse transform to get:

        $$\boxed{y[n]=2\left( \frac{1}{2} \right)^nu[n]-\left(\frac{1}{4}\right)u[n]}$$

    \end{enumerate}

  \item

    \begin{enumerate}

      \item 

      \item 

      \item 

    \end{enumerate}

  \item

    \begin{enumerate}

      \item 

        Setting up the Laplace Transform, we get:

        $$X(s)=\int_{2}^{\infty}e^{-(4+s)t}\,dt$$
        $$X(s)=-\frac{e^{-(4+s)t}}{4+s}\Big|_2^{\infty}$$
        $$X(s)=\frac{e^{-2(4+s)}}{s+4}$$
        $$\boxed{X(s)=\frac{e^{-2s-8}}{s+4}}$$

        Since the equation is right-sided, the ROC is to the right of the right-most pole; there is one pole at $s=-4$, so the ROC is $\text{Re}\left\{ s \right\}>-4\longrightarrow \sigma >-4$ (since $s=\sigma+j\omega$)

      \item 

        We may find the Laplace Transform to be:

        $$G(s)=\int_{-\infty}^{-2} Ae^{-(4+s)t}\,dt$$
        $$G(s)=-A\frac{e^{-(4+s)t}}{4+s}\Big|_{-\infty}^{-2}$$
        $$G(s)=-\frac{A}{4+s}\left[e^{2(4+s)}-e^{\infty(4+s)}\right]$$

        We may see that $G(s)$ converge only when $s$ reaches the ROC at $\sigma<-4$

        We can thus drop the term to get

        $$G(s)=-\frac{Ae^{2s+8}}{4+s}$$

        We can check the value of $A$:

        $$-Ae^{2s+8}=e^{-2s-8}$$

        We may see that, though the exponents will never be the same, we may take $A=-1$ to create a similar algebraic form. Thus, we say:

        $$\boxed{A=-1}$$

    \end{enumerate}

  \item

    \begin{enumerate}

      \item We may calculate the Laplace transform for the time-shifted signal as:

        $$X_1(s)=\int_{-\infty}^{\infty}x(t-t_o)e^{-st}\,dt$$

        Taking $t-t_o= n$, we write:

        $$X_1(s)=\int_{-\infty}^{\infty}x(n)e^{-s(n+t_o)}\,dn$$
        $$X_1(s)=e^{-st_o}\underbrace{\int_{-\infty}^{\infty}x(n)e^{-sn}\,dn}_{X(s)}$$

        Which gives us:

        $$\boxed{X_1(s)=e^{-st_o}X(s)}$$

        And we may see that the ROC remains to be $R$.

      \item We may express the signal as:

        $$y(t)=u(t)-2u(t-2)+u(t-4)$$

        Thus, we use our table of known transforms to get:

        $$\boxed{Y(s)=\frac{1}{s}-\frac{2e^{-2t}}{s}+\frac{e^{-4t}}{s}}$$

        We may see that the ROC is: $\boxed{\sigma>0}$

    \end{enumerate}

  \item

    We may rewrite $x(t)$ as:

    $$x(t)=e^t\sin(5t)u(-t)$$

    Which gives us:

    $$X(s)=\int_{-\infty}^{0}e^{-(s-1)t}\sin(5t)\,dt$$
    $$X(s)=-\frac{5}{(s-1)^2+25}$$
    $$\boxed{X(s)=-\frac{5}{s^2-2s+26}}$$

    We may see by the second equation that the region of convergence is left-handed, and occurs at $\boxed{\text{ROC: }\text{Re}\left\{s\right\}-1<0\longrightarrow \sigma <1}$. The poles will occur at the solutions to the quadratic in the denominator:

    $$s^2-2s+26=0$$
    $$\frac{2\pm\sqrt{4-4(1)(26)}}{2}$$
    $$\frac{2\pm10j}{2}$$
    $$\boxed{\text{Poles at: }s=1\pm 5j}$$

  \item

    Using partial fraction decomposition, we may write:

    $$\frac{s-1}{(s+1)(s+3)(s^2+4s+20)}=\frac{A}{s+1}+\frac{B}{s+3}+\frac{Cs+D}{s^2+4s+20}$$

    And then:

    $$(s+3)(s^2+4s+20)A+(s+1)(s^2+4s+20)B+(s+1)(s+3)(Cs+D)=s-1$$
    $$As^3+7As^2+32As+60A+Bs^3+5Bs^2+24Bs+20B+Cs^3+4Cs^2+3Cs+Ds^2+4Ds+3D=s-1$$

    From this, we may derive:

    $$A+B+C=0$$
    $$7A+5B+4C+D=0$$
    $$32A+24B+3C+4D=1$$
    $$60A+20B+3D=-1$$

    Using a solver, we obtain:

    $$\left\{\begin{array}{lll} A&=& -\frac{1}{17}\\B&=&\frac{2}{17} \\C&=&-\frac{1}{17} \\D&=&\frac{1}{17} \end{array}$$

      Now with our coefficients, we take the inverse Laplace transforms to get:

      $$x(t)=-\frac{1}{17}\mathcal{L}^{-1}\left\{ \frac{1}{s+1} \right\}+\frac{2}{17}\mathcal{L}^{-1}\left\{ \frac{1}{s+3} \right\}-\frac{1}{17}\mathcal{L}^{-1}\left\{ \frac{(s+2)}{(s+2)^2+4^2} \right\}+\frac{1}{17}\mathcal{L}^{-1}\left\{ \frac{4}{(s+2)^2+4^2} \right\}$$

      And finally, we get:

      $$\boxed{x(t)=-\frac{e^{-t}}{17}+\frac{2e^{-3t}}{17}-\frac{e^{-2t}\cos(4t)}{17}+\frac{e^{-2t}\sin(4t)}{17}}$$

      For each term, in order, the ROCs may be identified as: $\sigma=-1$, $\sigma=-3$, and $\sigma=-4$. Since all of the signals are causal, we know the ROCs are to the right. Thus, there will be overlap when $\sigma$ is greater than the greatest individual ROC, or $\sigma=-1$. This makes the combined ROC: $\boxed{\sigma > -1}$

      We may observe that \underline{four individual signals} contribute to the Laplace Transform. Furthermore, we can find the zeroes and poles as:

      $$\boxed{\text{Zero: }s-1=0\to s=1}$$
      $$\boxed{\text{Poles: }\left\{\begin{array}{lll} s+1&=&0\\s+3&=&0\\(s^2+4s+20)&=&0\end{array}\to\left\{\begin{array}{lll} s&=&-1\\s&=&-3\\s&=&-2\pm4j\end{array}}$$

          This can be plotted as:

          \begin{figure}[H]
            \centering
            \tikzset{every picture/.style={line width=0.75pt}} %set default line width to 0.75pt        

\begin{tikzpicture}[x=0.75pt,y=0.75pt,yscale=-1,xscale=1]
%uncomment if require: \path (0,474); %set diagram left start at 0, and has height of 474

%Shape: Axis 2D [id:dp6219221046431803] 
\draw  (273,216.2) -- (464,216.2)(292.1,47) -- (292.1,235) (457,211.2) -- (464,216.2) -- (457,221.2) (287.1,54) -- (292.1,47) -- (297.1,54)  ;
%Shape: Axis 2D [id:dp5932371701557269] 
\draw  (311.2,216.2) -- (120.2,216.2)(292.1,385.4) -- (292.1,197.4) (127.2,221.2) -- (120.2,216.2) -- (127.2,211.2) (297.1,378.4) -- (292.1,385.4) -- (287.1,378.4)  ;
%Straight Lines [id:da5927769720733128] 
\draw    (301.1,246.62) -- (282.68,246.62) ;
%Straight Lines [id:da8924142371155793] 
\draw    (301.52,277.04) -- (283.1,277.04) ;
%Straight Lines [id:da6343741647572215] 
\draw    (301.52,307.46) -- (283.1,307.46) ;
%Straight Lines [id:da045291958929904896] 
\draw    (301.52,337.89) -- (283.1,337.89) ;
%Straight Lines [id:da8520490310261968] 
\draw    (301.1,93.62) -- (282.68,93.62) ;
%Straight Lines [id:da09678114913988911] 
\draw    (301.52,124.04) -- (283.1,124.04) ;
%Straight Lines [id:da38186951340761854] 
\draw    (301.52,154.46) -- (283.1,154.46) ;
%Straight Lines [id:da1558443463386583] 
\draw    (301.52,184.89) -- (283.1,184.89) ;
%Straight Lines [id:da5661500775803274] 
\draw    (256.73,225.25) -- (256.73,206.83) ;
%Straight Lines [id:da44782558070120104] 
\draw    (226.31,225.67) -- (226.31,207.25) ;
%Straight Lines [id:da9454565674402448] 
\draw    (195.89,225.67) -- (195.89,207.25) ;
%Straight Lines [id:da23302610272268565] 
\draw    (165.47,225.67) -- (165.47,207.25) ;
%Straight Lines [id:da924562144766639] 
\draw    (418.73,225.25) -- (418.73,206.83) ;
%Straight Lines [id:da8634327340643182] 
\draw    (388.31,225.67) -- (388.31,207.25) ;
%Straight Lines [id:da5319759241322224] 
\draw    (357.89,225.67) -- (357.89,207.25) ;
%Straight Lines [id:da9162378417829431] 
\draw    (327.47,225.67) -- (327.47,207.25) ;
%Shape: Circle [id:dp04210201537722258] 
\draw  [fill={rgb, 255:red, 0; green, 0; blue, 0 }  ,fill opacity=1 ] (325,216) .. controls (325,214.9) and (325.9,214) .. (327,214) .. controls (328.1,214) and (329,214.9) .. (329,216) .. controls (329,217.1) and (328.1,218) .. (327,218) .. controls (325.9,218) and (325,217.1) .. (325,216) -- cycle ;
%Straight Lines [id:da009131086976706282] 
\draw [line width=1.5]    (263.87,209.13) -- (250.13,222.87) ;
%Straight Lines [id:da21142080684247977] 
\draw [line width=1.5]    (263.87,222.87) -- (250.13,209.13) ;

%Straight Lines [id:da26689355190151964] 
\draw [line width=1.5]    (202.87,210.13) -- (189.13,223.87) ;
%Straight Lines [id:da5769224571094517] 
\draw [line width=1.5]    (202.87,223.87) -- (189.13,210.13) ;

%Straight Lines [id:da6919652199776052] 
\draw [line width=1.5]    (233.87,88.13) -- (220.13,101.87) ;
%Straight Lines [id:da8329730824382561] 
\draw [line width=1.5]    (233.87,101.87) -- (220.13,88.13) ;

%Straight Lines [id:da011819217831910422] 
\draw  [dash pattern={on 0.84pt off 2.51pt}]  (292,94) -- (226.1,94.2) ;
%Straight Lines [id:da24635402207491586] 
\draw  [dash pattern={on 0.84pt off 2.51pt}]  (226.1,94.2) -- (227,216) ;
%Straight Lines [id:da44796495939202685] 
\draw [line width=1.5]    (233.87,344) -- (220.13,330.27) ;
%Straight Lines [id:da0628401042555723] 
\draw [line width=1.5]    (233.87,330.27) -- (220.13,344) ;

%Straight Lines [id:da29761276986457086] 
\draw  [dash pattern={on 0.84pt off 2.51pt}]  (292,338.13) -- (226.1,337.93) ;
%Straight Lines [id:da23555416733347623] 
\draw  [dash pattern={on 0.84pt off 2.51pt}]  (226.1,337.93) -- (227,216.13) ;
%Curve Lines [id:da6307660559589354] 
\draw    (378,251) .. controls (340.17,259.73) and (339.98,243.06) .. (334.52,223.79) ;
\draw [shift={(334,222)}, rotate = 73.3] [color={rgb, 255:red, 0; green, 0; blue, 0 }  ][line width=0.75]    (10.93,-3.29) .. controls (6.95,-1.4) and (3.31,-0.3) .. (0,0) .. controls (3.31,0.3) and (6.95,1.4) .. (10.93,3.29)   ;
%Curve Lines [id:da24902666001294926] 
\draw    (210.73,259.25) .. controls (248.56,267.98) and (248.76,251.31) .. (254.21,232.05) ;
\draw [shift={(254.73,230.25)}, rotate = 106.7] [color={rgb, 255:red, 0; green, 0; blue, 0 }  ][line width=0.75]    (10.93,-3.29) .. controls (6.95,-1.4) and (3.31,-0.3) .. (0,0) .. controls (3.31,0.3) and (6.95,1.4) .. (10.93,3.29)   ;

% Text Node
\draw (467,207.4) node [anchor=north west][inner sep=0.75pt]    {$\sigma $};
% Text Node
\draw (283,25.4) node [anchor=north west][inner sep=0.75pt]    {$j\omega $};
% Text Node
\draw (380,254.4) node [anchor=north west][inner sep=0.75pt]    {$\text{Zero: } \sigma =1$};
% Text Node
\draw (218.13,84.73) node [anchor=south east] [inner sep=0.75pt]    {$\text{Pole: } s=-2+4j$};
% Text Node
\draw (218.13,347.4) node [anchor=north east] [inner sep=0.75pt]    {$\text{Pole: } s=-2-4j$};
% Text Node
\draw (208.73,262.65) node [anchor=north east] [inner sep=0.75pt]    {$\text{Pole: } \sigma =-1$};
% Text Node
\draw (193.89,203.85) node [anchor=south east] [inner sep=0.75pt]    {$\text{Pole: } \sigma =-3$};


\end{tikzpicture}

            \caption{Pole-Zero Plot of $X(s)$}
            \label{fig:1}
          \end{figure}

  \item

    \begin{enumerate}

      \item 

        Per the basic Laplace transformation tables, we may write:

        $$\boxed{X(s)=\frac{1}{s+2}-\frac{1}{s-4}}$$

        We may observe that there are two ROCs, $\sigma<4$ and $\sigma>-2$, which gives us overlap in the region:

        $$\boxed{-2<\sigma<4}$$

        This gives us the following plot:

        \begin{figure}[H]
          \centering
          \tikzset{every picture/.style={line width=0.75pt}} %set default line width to 0.75pt        

\begin{tikzpicture}[x=0.75pt,y=0.75pt,yscale=-1,xscale=1]
%uncomment if require: \path (0,474); %set diagram left start at 0, and has height of 474

%Shape: Axis 2D [id:dp6219221046431803] 
\draw  (273,216.2) -- (464,216.2)(292.1,47) -- (292.1,235) (457,211.2) -- (464,216.2) -- (457,221.2) (287.1,54) -- (292.1,47) -- (297.1,54)  ;
%Shape: Axis 2D [id:dp5932371701557269] 
\draw  (311.2,216.2) -- (120.2,216.2)(292.1,385.4) -- (292.1,197.4) (127.2,221.2) -- (120.2,216.2) -- (127.2,211.2) (297.1,378.4) -- (292.1,385.4) -- (287.1,378.4)  ;
%Straight Lines [id:da5927769720733128] 
\draw    (301.1,246.62) -- (282.68,246.62) ;
%Straight Lines [id:da8924142371155793] 
\draw    (301.52,277.04) -- (283.1,277.04) ;
%Straight Lines [id:da6343741647572215] 
\draw    (301.52,307.46) -- (283.1,307.46) ;
%Straight Lines [id:da045291958929904896] 
\draw    (301.52,337.89) -- (283.1,337.89) ;
%Straight Lines [id:da8520490310261968] 
\draw    (301.1,93.62) -- (282.68,93.62) ;
%Straight Lines [id:da09678114913988911] 
\draw    (301.52,124.04) -- (283.1,124.04) ;
%Straight Lines [id:da38186951340761854] 
\draw    (301.52,154.46) -- (283.1,154.46) ;
%Straight Lines [id:da1558443463386583] 
\draw    (301.52,184.89) -- (283.1,184.89) ;
%Straight Lines [id:da5661500775803274] 
\draw    (256.73,225.25) -- (256.73,206.83) ;
%Straight Lines [id:da44782558070120104] 
\draw    (226.31,225.67) -- (226.31,207.25) ;
%Straight Lines [id:da9454565674402448] 
\draw    (195.89,225.67) -- (195.89,207.25) ;
%Straight Lines [id:da23302610272268565] 
\draw    (165.47,225.67) -- (165.47,207.25) ;
%Straight Lines [id:da924562144766639] 
\draw    (418.73,225.25) -- (418.73,206.83) ;
%Straight Lines [id:da8634327340643182] 
\draw    (388.31,225.67) -- (388.31,207.25) ;
%Straight Lines [id:da5319759241322224] 
\draw    (357.89,225.67) -- (357.89,207.25) ;
%Straight Lines [id:da9162378417829431] 
\draw    (327.47,225.67) -- (327.47,207.25) ;
%Straight Lines [id:da009131086976706282] 
\draw [line width=1.5]    (425.87,209.13) -- (412.13,222.87) ;
%Straight Lines [id:da21142080684247977] 
\draw [line width=1.5]    (425.87,222.87) -- (412.13,209.13) ;

%Straight Lines [id:da26689355190151964] 
\draw [line width=1.5]    (232.87,209.13) -- (219.13,222.87) ;
%Straight Lines [id:da5769224571094517] 
\draw [line width=1.5]    (232.87,222.87) -- (219.13,209.13) ;


% Text Node
\draw (467,207.4) node [anchor=north west][inner sep=0.75pt]    {$\sigma $};
% Text Node
\draw (283,25.4) node [anchor=north west][inner sep=0.75pt]    {$j\omega $};
% Text Node
\draw (217.13,205.73) node [anchor=south east] [inner sep=0.75pt]    {$\text{Pole: } \sigma =-2$};
% Text Node
\draw (423.87,205.73) node [anchor=south east] [inner sep=0.75pt]    {$\text{Pole: } \sigma =4$};


\end{tikzpicture}

          \caption{Pole-Zero Plot for 7(a)}
          \label{fig:2}
        \end{figure}

      \item 

        Once again employing the tables, we find:

        $$X(s)=\frac{1}{s+3}+\frac{4}{(s+2)^2+16}$$

        Rearranging to simplify ROC analysis, we get:

        $$\boxed{X(s)=\frac{s^2+8s+32}{(s+3)(s^2+4s+20)}}$$

        From this, we can determine that the zeros are at $s=-4\pm4j$ and there are poles at $-3$ and $-4\pm4j$. Since both are right-sided, we may notice that the ROC occurs to the right of the greatest pole, or $\sigma>-3$

        \begin{figure}[H]
          \centering
          \tikzset{every picture/.style={line width=0.75pt}} %set default line width to 0.75pt        

\begin{tikzpicture}[x=0.75pt,y=0.75pt,yscale=-1,xscale=1]
%uncomment if require: \path (0,474); %set diagram left start at 0, and has height of 474

%Shape: Axis 2D [id:dp6219221046431803] 
\draw  (273,216.2) -- (464,216.2)(292.1,47) -- (292.1,235) (457,211.2) -- (464,216.2) -- (457,221.2) (287.1,54) -- (292.1,47) -- (297.1,54)  ;
%Shape: Axis 2D [id:dp5932371701557269] 
\draw  (311.2,216.2) -- (120.2,216.2)(292.1,385.4) -- (292.1,197.4) (127.2,221.2) -- (120.2,216.2) -- (127.2,211.2) (297.1,378.4) -- (292.1,385.4) -- (287.1,378.4)  ;
%Straight Lines [id:da5927769720733128] 
\draw    (301.1,246.62) -- (282.68,246.62) ;
%Straight Lines [id:da8924142371155793] 
\draw    (301.52,277.04) -- (283.1,277.04) ;
%Straight Lines [id:da6343741647572215] 
\draw    (301.52,307.46) -- (283.1,307.46) ;
%Straight Lines [id:da045291958929904896] 
\draw    (301.52,337.89) -- (283.1,337.89) ;
%Straight Lines [id:da8520490310261968] 
\draw    (301.1,93.62) -- (282.68,93.62) ;
%Straight Lines [id:da09678114913988911] 
\draw    (301.52,124.04) -- (283.1,124.04) ;
%Straight Lines [id:da38186951340761854] 
\draw    (301.52,154.46) -- (283.1,154.46) ;
%Straight Lines [id:da1558443463386583] 
\draw    (301.52,184.89) -- (283.1,184.89) ;
%Straight Lines [id:da5661500775803274] 
\draw    (256.73,225.25) -- (256.73,206.83) ;
%Straight Lines [id:da44782558070120104] 
\draw    (226.31,225.67) -- (226.31,207.25) ;
%Straight Lines [id:da9454565674402448] 
\draw    (195.89,225.67) -- (195.89,207.25) ;
%Straight Lines [id:da23302610272268565] 
\draw    (165.47,225.67) -- (165.47,207.25) ;
%Straight Lines [id:da924562144766639] 
\draw    (418.73,225.25) -- (418.73,206.83) ;
%Straight Lines [id:da8634327340643182] 
\draw    (388.31,225.67) -- (388.31,207.25) ;
%Straight Lines [id:da5319759241322224] 
\draw    (357.89,225.67) -- (357.89,207.25) ;
%Straight Lines [id:da9162378417829431] 
\draw    (327.47,225.67) -- (327.47,207.25) ;
%Straight Lines [id:da26689355190151964] 
\draw [line width=1.5]    (202.87,210.13) -- (189.13,223.87) ;
%Straight Lines [id:da5769224571094517] 
\draw [line width=1.5]    (202.87,223.87) -- (189.13,210.13) ;

%Straight Lines [id:da784776016684483] 
\draw  [dash pattern={on 0.84pt off 2.51pt}]  (226.31,207.25) -- (226.31,338.1) ;
%Straight Lines [id:da21781695643516097] 
\draw  [dash pattern={on 0.84pt off 2.51pt}]  (283.1,337.89) -- (226.31,338.1) ;
%Straight Lines [id:da8723629480754602] 
\draw [line width=1.5]    (232.87,331.13) -- (219.13,344.87) ;
%Straight Lines [id:da7496589826727715] 
\draw [line width=1.5]    (232.87,344.87) -- (219.13,331.13) ;

%Straight Lines [id:da7134281311421604] 
\draw  [dash pattern={on 0.84pt off 2.51pt}]  (226.31,224.87) -- (226.31,94.02) ;
%Straight Lines [id:da7183955359401567] 
\draw  [dash pattern={on 0.84pt off 2.51pt}]  (283.1,94.23) -- (226.31,94.02) ;
%Straight Lines [id:da5345254447393222] 
\draw [line width=1.5]    (232.87,100.99) -- (219.13,87.25) ;
%Straight Lines [id:da9684032918781331] 
\draw [line width=1.5]    (232.87,87.25) -- (219.13,100.99) ;

%Shape: Circle [id:dp4379791563993981] 
\draw  [fill={rgb, 255:red, 0; green, 0; blue, 0 }  ,fill opacity=1 ] (162.47,94.83) .. controls (162.47,93.18) and (163.81,91.83) .. (165.47,91.83) .. controls (167.12,91.83) and (168.47,93.18) .. (168.47,94.83) .. controls (168.47,96.49) and (167.12,97.83) .. (165.47,97.83) .. controls (163.81,97.83) and (162.47,96.49) .. (162.47,94.83) -- cycle ;
%Straight Lines [id:da6159473943552052] 
\draw  [dash pattern={on 0.84pt off 2.51pt}]  (165.47,225.67) -- (165.47,94.83) ;
%Straight Lines [id:da31465228601770034] 
\draw  [dash pattern={on 0.84pt off 2.51pt}]  (222.26,95.04) -- (165.47,94.83) ;
%Shape: Circle [id:dp36236063520445505] 
\draw  [fill={rgb, 255:red, 0; green, 0; blue, 0 }  ,fill opacity=1 ] (162.47,338.67) .. controls (162.47,340.33) and (163.81,341.67) .. (165.47,341.67) .. controls (167.12,341.67) and (168.47,340.33) .. (168.47,338.67) .. controls (168.47,337.02) and (167.12,335.67) .. (165.47,335.67) .. controls (163.81,335.67) and (162.47,337.02) .. (162.47,338.67) -- cycle ;
%Straight Lines [id:da09656901809151264] 
\draw  [dash pattern={on 0.84pt off 2.51pt}]  (165.47,207.83) -- (165.47,338.67) ;
%Straight Lines [id:da5249728498053087] 
\draw  [dash pattern={on 0.84pt off 2.51pt}]  (222.26,338.46) -- (165.47,338.67) ;
%Curve Lines [id:da4525721928863522] 
\draw    (97.47,112.83) .. controls (137.07,83.13) and (118.84,127.92) .. (157.29,99.71) ;
\draw [shift={(158.47,98.83)}, rotate = 143.13] [color={rgb, 255:red, 0; green, 0; blue, 0 }  ][line width=0.75]    (10.93,-3.29) .. controls (6.95,-1.4) and (3.31,-0.3) .. (0,0) .. controls (3.31,0.3) and (6.95,1.4) .. (10.93,3.29)   ;
%Curve Lines [id:da8612798585116854] 
\draw    (97.47,320.83) .. controls (137.07,350.53) and (118.84,305.74) .. (157.29,333.95) ;
\draw [shift={(158.47,334.83)}, rotate = 216.87] [color={rgb, 255:red, 0; green, 0; blue, 0 }  ][line width=0.75]    (10.93,-3.29) .. controls (6.95,-1.4) and (3.31,-0.3) .. (0,0) .. controls (3.31,0.3) and (6.95,1.4) .. (10.93,3.29)   ;

% Text Node
\draw (467,207.4) node [anchor=north west][inner sep=0.75pt]    {$\sigma $};
% Text Node
\draw (283,25.4) node [anchor=north west][inner sep=0.75pt]    {$j\omega $};
% Text Node
\draw (193.89,203.85) node [anchor=south east] [inner sep=0.75pt]    {$\text{Pole: } \sigma =-3$};
% Text Node
\draw (217.13,348.27) node [anchor=north east] [inner sep=0.75pt]    {$\text{Pole: } s=-2-4j$};
% Text Node
\draw (217.13,83.85) node [anchor=south east] [inner sep=0.75pt]    {$\text{Pole: } s=-2+4j$};
% Text Node
\draw (97.47,116.23) node [anchor=north] [inner sep=0.75pt]    {$\text{Zero: } s=-4+4j$};
% Text Node
\draw (97.47,317.43) node [anchor=south] [inner sep=0.75pt]    {$\text{Zero: } s=-4-4j$};


\end{tikzpicture}

          \caption{Pole-Zero Plot for 7(b)}
          \label{fig:3}
        \end{figure}

      \item 

        We may rewrite $x(t)$ as:

        $$x(t)=-te^{2t}u(-t)+te^{-2t}u(t)$$

        Using our known transforms:

        $$X(s)=-\left[ -\frac{d}{ds}\left( \frac{1}{s-2} \right) \right]-\frac{d}{ds}\left( \frac{1}{s+2} \right)$$
        $$\boxed{X(s)=\frac{1}{(s+2)^2}-\frac{1}{(s-2)^2}}$$

        To simplify analysis, we rearrange to get:

        $$X(s)=\frac{(s-2)^2-(s+2)^2}{(s+2)^2(s-2)^2}$$
        $$X(s)=\frac{-8s}{(s+2)^2(s-2)^2}$$

        From this, we observe that there is a zero at $s=0$, and there are poles (both of order 2) $\sigma=\pm 2$. Since both signals are right-handed, the ROC is in: $\sigma>2$. This gives us the following plot:

        \begin{figure}[H]
          \centering
          \tikzset{every picture/.style={line width=0.75pt}} %set default line width to 0.75pt        

\begin{tikzpicture}[x=0.75pt,y=0.75pt,yscale=-1,xscale=1]
%uncomment if require: \path (0,474); %set diagram left start at 0, and has height of 474

%Shape: Axis 2D [id:dp6219221046431803] 
\draw  (273,216.2) -- (464,216.2)(292.1,47) -- (292.1,235) (457,211.2) -- (464,216.2) -- (457,221.2) (287.1,54) -- (292.1,47) -- (297.1,54)  ;
%Shape: Axis 2D [id:dp5932371701557269] 
\draw  (311.2,216.2) -- (120.2,216.2)(292.1,385.4) -- (292.1,197.4) (127.2,221.2) -- (120.2,216.2) -- (127.2,211.2) (297.1,378.4) -- (292.1,385.4) -- (287.1,378.4)  ;
%Straight Lines [id:da5927769720733128] 
\draw    (301.1,246.62) -- (282.68,246.62) ;
%Straight Lines [id:da8924142371155793] 
\draw    (301.52,277.04) -- (283.1,277.04) ;
%Straight Lines [id:da6343741647572215] 
\draw    (301.52,307.46) -- (283.1,307.46) ;
%Straight Lines [id:da045291958929904896] 
\draw    (301.52,337.89) -- (283.1,337.89) ;
%Straight Lines [id:da8520490310261968] 
\draw    (301.1,93.62) -- (282.68,93.62) ;
%Straight Lines [id:da09678114913988911] 
\draw    (301.52,124.04) -- (283.1,124.04) ;
%Straight Lines [id:da38186951340761854] 
\draw    (301.52,154.46) -- (283.1,154.46) ;
%Straight Lines [id:da1558443463386583] 
\draw    (301.52,184.89) -- (283.1,184.89) ;
%Straight Lines [id:da5661500775803274] 
\draw    (256.73,225.25) -- (256.73,206.83) ;
%Straight Lines [id:da44782558070120104] 
\draw    (226.31,225.67) -- (226.31,207.25) ;
%Straight Lines [id:da9454565674402448] 
\draw    (195.89,225.67) -- (195.89,207.25) ;
%Straight Lines [id:da23302610272268565] 
\draw    (165.47,225.67) -- (165.47,207.25) ;
%Straight Lines [id:da924562144766639] 
\draw    (418.73,225.25) -- (418.73,206.83) ;
%Straight Lines [id:da8634327340643182] 
\draw    (388.31,225.67) -- (388.31,207.25) ;
%Straight Lines [id:da5319759241322224] 
\draw    (357.89,225.67) -- (357.89,207.25) ;
%Straight Lines [id:da9162378417829431] 
\draw    (327.47,225.67) -- (327.47,207.25) ;
%Straight Lines [id:da26689355190151964] 
\draw [line width=1.5]    (232.87,210.13) -- (219.13,223.87) ;
%Straight Lines [id:da5769224571094517] 
\draw [line width=1.5]    (232.87,223.87) -- (219.13,210.13) ;

%Straight Lines [id:da9622167571036273] 
\draw [line width=1.5]    (364.87,209.13) -- (351.13,222.87) ;
%Straight Lines [id:da013702222277968024] 
\draw [line width=1.5]    (364.87,222.87) -- (351.13,209.13) ;

%Shape: Circle [id:dp8631696852731696] 
\draw  [fill={rgb, 255:red, 0; green, 0; blue, 0 }  ,fill opacity=1 ] (287.1,216.2) .. controls (287.1,213.44) and (289.34,211.2) .. (292.1,211.2) .. controls (294.86,211.2) and (297.1,213.44) .. (297.1,216.2) .. controls (297.1,218.96) and (294.86,221.2) .. (292.1,221.2) .. controls (289.34,221.2) and (287.1,218.96) .. (287.1,216.2) -- cycle ;
%Curve Lines [id:da4587328406103208] 
\draw    (243,262) .. controls (264.99,236.76) and (269.86,260.27) .. (281.46,227.54) ;
\draw [shift={(282,226)}, rotate = 108.92] [color={rgb, 255:red, 0; green, 0; blue, 0 }  ][line width=0.75]    (10.93,-3.29) .. controls (6.95,-1.4) and (3.31,-0.3) .. (0,0) .. controls (3.31,0.3) and (6.95,1.4) .. (10.93,3.29)   ;

% Text Node
\draw (467,207.4) node [anchor=north west][inner sep=0.75pt]    {$\sigma $};
% Text Node
\draw (283,25.4) node [anchor=north west][inner sep=0.75pt]    {$j\omega $};
% Text Node
\draw (217.13,206.73) node [anchor=south east] [inner sep=0.75pt]    {$\text{Pole: } \sigma =-2$};
% Text Node
\draw (359.89,203.85) node [anchor=south west] [inner sep=0.75pt]    {$\text{Pole: } \sigma =2$};
% Text Node
\draw (241,265.4) node [anchor=north east] [inner sep=0.75pt]    {$\text{Zero:} \ s=0$};


\end{tikzpicture}

          \caption{Pole-Zero Plot for 7(c)}
          \label{fig:4}
        \end{figure}

      \item 

        We may begin by writing:

        $$x(t)=3r(t)-3r(t-1))-3u(t-2)$$

        This gives us the transform as:

        $$\boxed{X(s)=\frac{3}{s^2}-\frac{3}{s^2}e^{s}-\frac{3}{s}e^{2s}}$$

        We may observe that there are no zeros, but there is a pole at $s=0$, which gives an ROC of $s>0$ and the following plot:

        \begin{figure}[H]
          \centering
          \tikzset{every picture/.style={line width=0.75pt}} %set default line width to 0.75pt        

\begin{tikzpicture}[x=0.75pt,y=0.75pt,yscale=-1,xscale=1]
%uncomment if require: \path (0,474); %set diagram left start at 0, and has height of 474

%Shape: Axis 2D [id:dp6219221046431803] 
\draw  (273,216.2) -- (464,216.2)(292.1,47) -- (292.1,235) (457,211.2) -- (464,216.2) -- (457,221.2) (287.1,54) -- (292.1,47) -- (297.1,54)  ;
%Shape: Axis 2D [id:dp5932371701557269] 
\draw  (311.2,216.2) -- (120.2,216.2)(292.1,385.4) -- (292.1,197.4) (127.2,221.2) -- (120.2,216.2) -- (127.2,211.2) (297.1,378.4) -- (292.1,385.4) -- (287.1,378.4)  ;
%Straight Lines [id:da5927769720733128] 
\draw    (301.1,246.62) -- (282.68,246.62) ;
%Straight Lines [id:da8924142371155793] 
\draw    (301.52,277.04) -- (283.1,277.04) ;
%Straight Lines [id:da6343741647572215] 
\draw    (301.52,307.46) -- (283.1,307.46) ;
%Straight Lines [id:da045291958929904896] 
\draw    (301.52,337.89) -- (283.1,337.89) ;
%Straight Lines [id:da8520490310261968] 
\draw    (301.1,93.62) -- (282.68,93.62) ;
%Straight Lines [id:da09678114913988911] 
\draw    (301.52,124.04) -- (283.1,124.04) ;
%Straight Lines [id:da38186951340761854] 
\draw    (301.52,154.46) -- (283.1,154.46) ;
%Straight Lines [id:da1558443463386583] 
\draw    (301.52,184.89) -- (283.1,184.89) ;
%Straight Lines [id:da5661500775803274] 
\draw    (256.73,225.25) -- (256.73,206.83) ;
%Straight Lines [id:da44782558070120104] 
\draw    (226.31,225.67) -- (226.31,207.25) ;
%Straight Lines [id:da9454565674402448] 
\draw    (195.89,225.67) -- (195.89,207.25) ;
%Straight Lines [id:da23302610272268565] 
\draw    (165.47,225.67) -- (165.47,207.25) ;
%Straight Lines [id:da924562144766639] 
\draw    (418.73,225.25) -- (418.73,206.83) ;
%Straight Lines [id:da8634327340643182] 
\draw    (388.31,225.67) -- (388.31,207.25) ;
%Straight Lines [id:da5319759241322224] 
\draw    (357.89,225.67) -- (357.89,207.25) ;
%Straight Lines [id:da9162378417829431] 
\draw    (327.47,225.67) -- (327.47,207.25) ;
%Straight Lines [id:da9622167571036273] 
\draw [line width=1.5]    (298.87,209.13) -- (285.13,222.87) ;
%Straight Lines [id:da013702222277968024] 
\draw [line width=1.5]    (298.87,222.87) -- (285.13,209.13) ;

%Curve Lines [id:da4587328406103208] 
\draw    (243,262) .. controls (264.99,236.76) and (269.86,260.27) .. (281.46,227.54) ;
\draw [shift={(282,226)}, rotate = 108.92] [color={rgb, 255:red, 0; green, 0; blue, 0 }  ][line width=0.75]    (10.93,-3.29) .. controls (6.95,-1.4) and (3.31,-0.3) .. (0,0) .. controls (3.31,0.3) and (6.95,1.4) .. (10.93,3.29)   ;

% Text Node
\draw (467,207.4) node [anchor=north west][inner sep=0.75pt]    {$\sigma $};
% Text Node
\draw (283,25.4) node [anchor=north west][inner sep=0.75pt]    {$j\omega $};
% Text Node
\draw (241,265.4) node [anchor=north east] [inner sep=0.75pt]    {$\text{Pole: } s=0$};


\end{tikzpicture}

          \caption{Pole-Zero Plot for 7(d)}
          \label{fig:5}
        \end{figure}

    \end{enumerate}

  \item Since the function is absolutely integrable, we know that it is bounded by BIBO stability. This means that the ROC contains the $j \omega$ axis, and that the ROC does not have any poles. Knowing this, we can deduce:

    \begin{enumerate}

      \item \underline{The function can not be finite}

        This is due to the fact that, since there is a pole at $s=-4$, we know that the signal contains $x(t)=e^{-4t}$. Since this function extends to $t\to\infty$, we know that the signal can not be finite. Also, note that a finite signal, by definition, does not contain poles.

      \item \underline{The function can not be left-sided}

        Since we know that, due to being absolutely integrable, the function must contain the $j \omega$ axis, the right-most pole must be to the right of this axis. Since $s=-4$ is to the left of the axis, the signal can not be left-sided.

      \item \underline{The function can be right-sided}

        An example of such a signal would be:

        $$\boxed{x(t)=e^{-4t}u(t)}$$

      \item \underline{The function can be two-sided} (given that there may be multiple poles — the problem states that there \textit{is} a pole at $s=-4$, but does not state that it is the only pole)

        Take, for example, the signal (where $n>0$):

        $$x(t)=e^{-4t}u(t)+e^{nt}u(-t)$$

        This signal is two-sided, contains $s=-4$ as a pole, is absolutely integrable (and thus contains the $j\omega$ axis), which makes it a valid signal.

      \item 

        \begin{itemize}

          \item With the absolutely integrable rule change from $x(t)$ to $x(t)e^{-2t}$, nothing would change. This is because it would mean that, instead of the $j\omega$ axis, the ROC would now need to contain the $\sigma=2$ axis. Thus, the properties remain the same since this is to the right of the pole at $s=-4$.

          \item Changing the condition such that $x(t)e^{5t}$ is absolutely integrable would change things. This would mean that the ROC would need to contain the $\sigma=-5$ axis, instead of the $j\omega$ axis, which is left of the pole. Thus, the signal \underline{can now be left-sided, but not right-sided}

        \end{itemize}

      \item 

        We know that, due to the integrable rule, the ROC must be right of the right-most pole (to contain the $j\omega$ axis). This means that the ROC is $\sigma>-2$. From here, we write $X(s)$ as:

        $$X(s)=\frac{N(s-1)}{(s+4)(s+2-j2)}$$

        We can solve for $N$ by using the initial condition:

        $$X(0)=\frac{-N}{4(2-j2)}$$
        $$\frac{-N}{4(2-j2)}=-1$$
        $$N=8-j8$$

        This gives us:

        $$\boxed{X(s)=\frac{(8-j8)(s-1)}{(s+4)(s+2-j2)}}$$

    \end{enumerate}

  \item

    \begin{enumerate}

      \item Using the table, we get:

        $$\boxed{x(t)=\frac{1}{3}\sin(3t)u(t)}$$

        And we see that the ROC is $\boxed{\sigma>0}$

      \item Using the table, we get:

        $$\boxed{x(t)=-\cos(2t)u(-t)}$$

        And we see that the ROC is $\boxed{\sigma<0}$

      \item Using the table, we get:

        $$\boxed{x(t)=-e^{-5t}\cos(4t)u(-t)}$$

        And we see that the ROC is $\boxed{\sigma<-5}$

      \item To simplify analysis, we may rewrite this as:

        $$X(s)=\frac{1}{(s+4)(s+2)}$$

        We use partial fraction decomposition to get:

        $$X(s)=\frac{A}{s+2}+\frac{B}{s+4}$$

        Solving, we find: $A=-.5$ and $B=.5$, which gives us:

        $$X(s)=\frac{.5}{s+4}-\frac{.5}{s+2}$$

        Finally, using the table, we get:

        $$\boxed{x(t)=-\frac{1}{2}\left[ e^{-4t}u(t)+e^{-2t}u(-t) \right]}$$

        And we see the ROC is: $\boxed{-4<\sigma<-2}$

      \item This can be expanded to:

        $$X(s)=\frac{(s+2)^2-5s-10+8}{(s+2)^2}$$

        We then break this into:

        $$X(s)=1-\frac{5}{s+2}+\frac{8}{(s+2)^2}$$

        We now use the table to find:

        $$\boxed{x(t)=\delta(t)-5e^{-2t}u(t)+8te^{-2t}u(t)}$$

        With ROC: $\boxed{\sigma>-2}$

    \end{enumerate}

\end{enumerate}

\end{document}

