%%%%%%%%%%%%%%%%%%%%%%%%%%%%%%%%%%%%%%%%%%%%%%%%%%%%%%%%%%%%%%%%%%%%%%%%%%%%%%%%%%%%%%%%%%%%%%%%%%%%%%%%%%%%%%%%%%%%%%%%%%%%%%%%%%%%%%%%%%%%%%%%%%%%%%%%%%%%%%%%%%%
% Written By Michael Brodskiy
% Class: Fundamentals of Linear Systems
% Professor: I. Salama
%%%%%%%%%%%%%%%%%%%%%%%%%%%%%%%%%%%%%%%%%%%%%%%%%%%%%%%%%%%%%%%%%%%%%%%%%%%%%%%%%%%%%%%%%%%%%%%%%%%%%%%%%%%%%%%%%%%%%%%%%%%%%%%%%%%%%%%%%%%%%%%%%%%%%%%%%%%%%%%%%%%

\documentclass[12pt]{article} 
\usepackage{alphalph}
\usepackage[utf8]{inputenc}
\usepackage[russian,english]{babel}
\usepackage{titling}
\usepackage{amsmath}
\usepackage{graphicx}
\usepackage{enumitem}
\usepackage{amssymb}
\usepackage[super]{nth}
\usepackage{everysel}
\usepackage{ragged2e}
\usepackage{geometry}
\usepackage{multicol}
\usepackage{fancyhdr}
\usepackage{cancel}
\usepackage{siunitx}
\usepackage{physics}
\usepackage{tikz}
\usepackage{mathdots}
\usepackage{yhmath}
\usepackage{cancel}
\usepackage{color}
\usepackage{array}
\usepackage{multirow}
\usepackage{gensymb}
\usepackage{tabularx}
\usepackage{extarrows}
\usepackage{booktabs}
\usepackage{lastpage}
\usetikzlibrary{fadings}
\usetikzlibrary{patterns}
\usetikzlibrary{shadows.blur}
\usetikzlibrary{shapes}

\geometry{top=1.0in,bottom=1.0in,left=1.0in,right=1.0in}
\newcommand{\subtitle}[1]{%
  \posttitle{%
    \par\end{center}
    \begin{center}\large#1\end{center}
    \vskip0.5em}%

}
\usepackage{hyperref}
\hypersetup{
colorlinks=true,
linkcolor=blue,
filecolor=magenta,      
urlcolor=blue,
citecolor=blue,
}


\title{Homework 6}
\date{\today}
\author{Michael Brodskiy\\ \small Professor: I. Salama}

\begin{document}

\maketitle

\begin{enumerate}

  \item

    \begin{enumerate}

      \item Per the rules of Laplace Transforms, we can convolve two signals by the rule that:

        $$y(t)=x_1(t)*x_2(t)\to Y(s)=X_1(s)X_2(s)$$

        As such, we may obtain:

        $$X_1(s)=\frac{1}{s+4}\quad\text{ and }\quad X_2(s)=\frac{1}{s+2}$$

        Now, we account for the shifts. We know that, for $x(t)\to x(t-t_o)$ the transform becomes $X(s)\to e^{st_o}X(s)$. Furthermore, we know that for $x(-t)\to X(-s)$. Thus, we find:

        $$X_1(s)=\frac{e^{-3s}}{-s+4}\quad\text{ and }\quad X_2(s)=\frac{e^{-2s}}{s+2}$$

        Multiplying together, we find:

        $$\boxed{Y(s)=\frac{e^{-5s}}{(4-s)(s+2)},\quad\text{ROC: } -2<\sigma<4}$$

      \item 

    \end{enumerate}

  \item

    First, we know that the poles must be at plus or minus the imaginary value, so the two poles must be at $s=-1\pm 3j$. Thus, we see that $X(s)$ can be expressed as:

    $$X(s)=\frac{k}{(s+1-3j)(s+1+3j)}$$
    $$X(s)=\frac{k}{(s+1)^2+3^2}$$

    We then apply the condition given in statement (5) to get:

    $$2=\frac{k}{(1^2)+(3^2)}$$
    $$k=20$$

    Then, because of statement (4), we know that $s=4$ is NOT in the ROC of $X(s)$. This means that we obtain the transform as:

    $$\boxed{X(s)=\frac{20}{(s+1)^2+3^2},\quad\text{ROC: }\sigma <-1}$$

    Taking the inverse transform, per our Laplace tables, we see:

    $$\boxed{x(t)=-\frac{20}{3}e^{-t}\sin(3t)u(-t)}$$

  \item

    \begin{enumerate}

      \item 

        Using our tables, we may obtain (with $X(s)$ ROC: $\sigma <3$ and $H(s)$ ROC: $\sigma>-2$):

        $$\boxed{X(s)=-\frac{5}{s-3}\quad\text{ and }H(s)=\frac{1}{s+2}}$$

      \item 

        We may write the convolution transform as:

        $$Y(s)=X(s)H(s)$$

        Thus, we get:

        $$Y(s)=\left( -\frac{5}{s-3} \right)\left( \frac{1}{s+2} \right)$$
        $$\boxed{Y(s)=-\frac{5}{(s-3)(s+2)}}$$

      \item 

        We begin by using partial fraction decomposition, which gives us:

        $$Y(s)=\frac{A}{s-3}+\frac{B}{s+2}$$

        From here, we get $A=-1$ and $B=1$, which gives us:

        $$Y(s)=\frac{-1}{s-3}+\frac{1}{s+2}$$

        Using our inverse transforms, we obtain:

        $$\boxed{y(t)=e^{3t}u(-t)+e^{-2t}u(t)}$$

      \item 

        Explicit convolution gives us:

        $$x(t)*h(t)=\int_0^t 5e^{3\tau}u(-\tau)e^{-2(t-\tau)}u(t-\tau)\,d\tau$$
        $$x(t)*h(t)=\int_0^t 5e^{-2t+5\tau}u(-\tau)u(t-\tau)\,d\tau$$

        We see that the function is bounded by:

        $$\tau \leq 0\quad\text{ and }\quad \tau\leq t$$

        From this, we may write:

        $$y(t)=-5e^t\int_0^t e^{5\tau}\,d\tau$$
        $$y(t)=-e^{-2t}\left[ e^{5\tau} \right]\Big|_0^t$$
        $$y(t)=-e^{-2t}\left[ e^{5t}-1 \right]$$

        This confirms:

        $$\boxed{y(t)=e^{3t}u(-t)+e^{-2t}u(t)}$$

    \end{enumerate}

  \item

    \begin{enumerate}

      \item 

        Taking the Laplace transform, we get:

        $$s^2Y(s)-sY(s)-6Y(s)=sX(s)$$
        $$Y(s)[s^2-s-6]=sX(s)$$
        $$\boxed{H(s)=\frac{Y(s)}{X(s)}=\frac{s}{s^2-s-6}}$$

        Thus, we see that there is a zero at $s=0$ and poles at $s=-2,3$. This allows us to plot:

        \begin{figure}[H]
          \centering
          \tikzset{every picture/.style={line width=0.75pt}} %set default line width to 0.75pt        

\begin{tikzpicture}[x=0.75pt,y=0.75pt,yscale=-1,xscale=1]
%uncomment if require: \path (0,413); %set diagram left start at 0, and has height of 413

%Shape: Axis 2D [id:dp07291137441905748] 
\draw  (271,207.4) -- (507,207.4)(294.6,4) -- (294.6,230) (500,202.4) -- (507,207.4) -- (500,212.4) (289.6,11) -- (294.6,4) -- (299.6,11)  ;
%Shape: Axis 2D [id:dp6518246903591405] 
\draw  (318.2,207.4) -- (82.2,207.4)(294.6,410.8) -- (294.6,184.8) (89.2,212.4) -- (82.2,207.4) -- (89.2,202.4) (299.6,403.8) -- (294.6,410.8) -- (289.6,403.8)  ;
%Shape: Grid [id:dp49490922295513473] 
\draw  [draw opacity=0] (294.6,57.4) -- (444.6,57.4) -- (444.6,207.4) -- (294.6,207.4) -- cycle ; \draw   (344.6,57.4) -- (344.6,207.4)(394.6,57.4) -- (394.6,207.4) ; \draw   (294.6,107.4) -- (444.6,107.4)(294.6,157.4) -- (444.6,157.4) ; \draw   (294.6,57.4) -- (444.6,57.4) -- (444.6,207.4) -- (294.6,207.4) -- cycle ;
%Shape: Grid [id:dp22016715431592204] 
\draw  [draw opacity=0] (294.6,207.4) -- (444.6,207.4) -- (444.6,357.4) -- (294.6,357.4) -- cycle ; \draw   (344.6,207.4) -- (344.6,357.4)(394.6,207.4) -- (394.6,357.4) ; \draw   (294.6,257.4) -- (444.6,257.4)(294.6,307.4) -- (444.6,307.4) ; \draw   (294.6,207.4) -- (444.6,207.4) -- (444.6,357.4) -- (294.6,357.4) -- cycle ;
%Shape: Grid [id:dp9168332724400642] 
\draw  [draw opacity=0] (144.6,207.4) -- (294.6,207.4) -- (294.6,357.4) -- (144.6,357.4) -- cycle ; \draw   (194.6,207.4) -- (194.6,357.4)(244.6,207.4) -- (244.6,357.4) ; \draw   (144.6,257.4) -- (294.6,257.4)(144.6,307.4) -- (294.6,307.4) ; \draw   (144.6,207.4) -- (294.6,207.4) -- (294.6,357.4) -- (144.6,357.4) -- cycle ;
%Shape: Grid [id:dp019661334699653366] 
\draw  [draw opacity=0] (144.6,57.4) -- (294.6,57.4) -- (294.6,207.4) -- (144.6,207.4) -- cycle ; \draw   (194.6,57.4) -- (194.6,207.4)(244.6,57.4) -- (244.6,207.4) ; \draw   (144.6,107.4) -- (294.6,107.4)(144.6,157.4) -- (294.6,157.4) ; \draw   (144.6,57.4) -- (294.6,57.4) -- (294.6,207.4) -- (144.6,207.4) -- cycle ;
%Shape: Circle [id:dp8466727816697135] 
\draw  [line width=1.5]  (289.15,207.4) .. controls (289.15,204.39) and (291.59,201.95) .. (294.6,201.95) .. controls (297.61,201.95) and (300.05,204.39) .. (300.05,207.4) .. controls (300.05,210.41) and (297.61,212.85) .. (294.6,212.85) .. controls (291.59,212.85) and (289.15,210.41) .. (289.15,207.4) -- cycle ;
%Straight Lines [id:da019164220491805106] 
\draw [line width=1.5]    (202.22,200.28) -- (187.78,214.72) ;
%Straight Lines [id:da7010872252517704] 
\draw [line width=1.5]    (202.22,214.72) -- (187.78,200.28) ;

%Straight Lines [id:da8787997404879367] 
\draw [line width=1.5]    (452.22,200.28) -- (437.78,214.72) ;
%Straight Lines [id:da40962171693247995] 
\draw [line width=1.5]    (452.22,214.72) -- (437.78,200.28) ;

%Curve Lines [id:da12390900966481411] 
\draw  [dash pattern={on 4.5pt off 4.5pt}]  (117,246) .. controls (156.6,216.3) and (147.2,249.33) .. (185.81,220.88) ;
\draw [shift={(187,220)}, rotate = 143.13] [color={rgb, 255:red, 0; green, 0; blue, 0 }  ][line width=0.75]    (10.93,-3.29) .. controls (6.95,-1.4) and (3.31,-0.3) .. (0,0) .. controls (3.31,0.3) and (6.95,1.4) .. (10.93,3.29)   ;
%Curve Lines [id:da10550563739707763] 
\draw  [dash pattern={on 4.5pt off 4.5pt}]  (528.22,170.28) .. controls (488.62,199.98) and (498.03,166.95) .. (459.41,195.4) ;
\draw [shift={(458.22,196.28)}, rotate = 323.13] [color={rgb, 255:red, 0; green, 0; blue, 0 }  ][line width=0.75]    (10.93,-3.29) .. controls (6.95,-1.4) and (3.31,-0.3) .. (0,0) .. controls (3.31,0.3) and (6.95,1.4) .. (10.93,3.29)   ;

% Text Node
\draw (302,3.4) node [anchor=north west][inner sep=0.75pt]    {$\omega $};
% Text Node
\draw (507,209.4) node [anchor=north west][inner sep=0.75pt]    {$\sigma $};
% Text Node
\draw (26,249) node [anchor=north west][inner sep=0.75pt]   [align=left] {Pole: $\displaystyle \sigma =-2$};
% Text Node
\draw (528.22,167.28) node [anchor=south] [inner sep=0.75pt]   [align=left] {Pole: $\displaystyle \sigma =3$};
% Text Node
\draw (246.6,160.4) node [anchor=north west][inner sep=0.75pt]   [align=left] {Hole:\\$\displaystyle s=0$};


\end{tikzpicture}

          \caption{Pole-Zero Plot}
          \label{fig:1}
        \end{figure}
        
      \item 

        We may begin by using partial fraction decomposition:

        $$\frac{s}{s^2-s-6}\Rightarrow \frac{A}{s-3}+\frac{B}{s+2}$$

        Plugging in our values, we find $A=3/5$ and $B=2/5$, which gives us:

        $$H(s)=\frac{3/5}{s-3}+\frac{2/5}{s+2}$$

        \begin{enumerate}

          \item When the system is stable, we know that the ROC must be bounded. Thus, we know that the ROC is $-2<\sigma<3$. Using our transform table, this gives:

            $$\boxed{h(t)=-\frac{3}{5}e^{3t}u(-t)+\frac{2}{5}e^{-2t}u(t)}$$

          \item When the system is causal, we know that the ROC is right-sided, such that $\sigma>3$. Thus, we see:

            $$\boxed{h(t)=\frac{3}{5}e^{3t}u(t)+\frac{2}{5}e^{-2t}u(t)}$$

          \item When it is neither stable nor causal, the ROC must be left-sided and can not include the j$\omega$ axis. This gives us:

            $$\boxed{h(t)=\frac{3}{5}e^{3t}u(t)-\frac{2}{5}e^{-2t}u(-t)}$$

        \end{enumerate}

    \end{enumerate}

  \item Given that this is the step response, and that it is multiplied by the step function, we know that:

    $$X(s)=\frac{1}{s}$$

    We take the transform of $y(t)$ to get:

    $$Y(s)=\frac{1}{s}-\frac{1}{s+2}-\frac{2}{(s+2)^2}$$

    We know that:

    $$H(s)=\frac{Y(s)}{H(s)}$$

    Thus, we find the transfer function be:

    $$H(s)=1-\frac{s}{s+2}-\frac{2s}{(s+2)^2}$$
    $$H(s)=\frac{4}{(s+2)^2}$$

    Then we can find:

    $$Y_1(s)=\frac{1}{s}-\frac{2}{s+2}+\frac{1}{s+4}$$

    Knowing that this must be equivalent to the transfer function, we see:

    $$\frac{Y_1(s)}{H(s)}=X_1(s)$$

    This gives us:

    $$X_1(s)=\frac{(s+2)^2}{4s}-.5(s+2)+\frac{(s+2)^2}{4(s+4)}$$
    $$X_1(s)=\frac{(s+2)^2}{4s}-.5(s+2)+\frac{(s+2)^2}{4s+16}$$

    We can simplify to get:

    $$X_1(s)=\frac{2s+4}{s^2+4s}$$
    $$X_1(s)=\frac{2}{s+4}+\frac{4}{s(s+4)}$$

    We use partial fraction decomposition for the second term to get:

    $$X_1(s)=\frac{2}{s+4}+\frac{A}{s+4}+\frac{B}{s}$$

    We find $A=-1$ and $B=1$ to get:

    $$X_1(s)=\frac{1}{s+4}+\frac{1}{s}$$

    Taking the inverse, we find:

    $$\boxed{x_1(t)=[e^{-4t}+1]u(t)}$$

  \item

    We may express $x(t)$ as:

    $$x(t)=e^{t}u(-t)+e^{-t}u(t)$$

    This gives us:

    $$X(s)=-\frac{1}{s-1}+\frac{1}{s+1}$$

    We combine the two terms to get:

    $$X(s)=\frac{-2}{s^2-1}$$

    Multiplying by the system function, we get:

    $$Y(s)=\left[ \frac{s+1}{s^2+2s+9} \right]\left[ -\frac{2}{s^2-1} \right]$$

    We continue to simplify:

    $$Y(s)=\left[ \frac{1}{s^2+2s+9} \right]\left[ -\frac{2}{s-1} \right]$$
    $$Y(s)=\frac{-2}{(s^2+2s+9)(s-1)}$$

    We then use partial fraction decomposition to write:

    $$Y(s)=\frac{A}{s-1}+\frac{Bs+C}{s^2+2s+9}$$

    Thus, we get:

    $$A(s^2+2s+9)+(Bs+C)(s-1)=-2$$
    $$As^2+2As+9A+Bs^2-Bs+Cs-C=-2$$

    From this, we can set up the following system:

    $$A+B=0$$
    $$2A-B-C=0$$
    $$9A-C=-2$$

    Solving the system, we see: $A=-\frac{1}{6}$, $B=\frac{1}{6}$, and $C=\frac{1}{2}$, which gives us:

    $$Y(s)=\frac{-1/6}{s-1}+\frac{(1/6)s+(1/2)}{s^2+2s+9}$$

    We break this up further to see:

    $$Y(s)=\frac{-1/6}{s-1}+\frac{(1/6)s+(1/6)}{(s+1)^2+8}+\frac{(1/3)}{(s+1)^2+8}$$
    $$Y(s)=\frac{-1/6}{s-1}+\frac{(1/6)s+(1/6)}{(s+1)^2+8}+\frac{1}{\sqrt{8}}\frac{3\sqrt{8}}{(s+1)^2+8}$$

    We then use the Laplace tables, and the fact that the system is causal, to get:

    $$\boxed{y(t)=-\frac{1}{6}e^{t}u(t)+\frac{1}{6}e^{-t}\cos(\sqrt{8}t)u(t)+\frac{1}{3\sqrt{8}}e^{-t}\sin(\sqrt{8}t)u(t)}$$

  \item

    \begin{enumerate}

      \item 

      \item 

    \end{enumerate}

  \item

    \begin{enumerate}

      \item 

      \item 

      \item 

      \item 

      \item 

      \item 

      \item 

    \end{enumerate}

  \item

    \begin{enumerate}

      \item 

      \item 

      \item 

      \item 

    \end{enumerate}

\end{enumerate}

\end{document}

