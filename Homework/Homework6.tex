%%%%%%%%%%%%%%%%%%%%%%%%%%%%%%%%%%%%%%%%%%%%%%%%%%%%%%%%%%%%%%%%%%%%%%%%%%%%%%%%%%%%%%%%%%%%%%%%%%%%%%%%%%%%%%%%%%%%%%%%%%%%%%%%%%%%%%%%%%%%%%%%%%%%%%%%%%%%%%%%%%%
% Written By Michael Brodskiy
% Class: Fundamentals of Linear Systems
% Professor: I. Salama
%%%%%%%%%%%%%%%%%%%%%%%%%%%%%%%%%%%%%%%%%%%%%%%%%%%%%%%%%%%%%%%%%%%%%%%%%%%%%%%%%%%%%%%%%%%%%%%%%%%%%%%%%%%%%%%%%%%%%%%%%%%%%%%%%%%%%%%%%%%%%%%%%%%%%%%%%%%%%%%%%%%

\documentclass[12pt]{article} 
\usepackage{alphalph}
\usepackage[utf8]{inputenc}
\usepackage[russian,english]{babel}
\usepackage{titling}
\usepackage{amsmath}
\usepackage{graphicx}
\usepackage{enumitem}
\usepackage{amssymb}
\usepackage[super]{nth}
\usepackage{everysel}
\usepackage{ragged2e}
\usepackage{geometry}
\usepackage{multicol}
\usepackage{fancyhdr}
\usepackage{cancel}
\usepackage{siunitx}
\usepackage{physics}
\usepackage{tikz}
\usepackage{mathdots}
\usepackage{yhmath}
\usepackage{cancel}
\usepackage{color}
\usepackage{array}
\usepackage{multirow}
\usepackage{gensymb}
\usepackage{tabularx}
\usepackage{extarrows}
\usepackage{booktabs}
\usepackage{lastpage}
\usetikzlibrary{fadings}
\usetikzlibrary{patterns}
\usetikzlibrary{shadows.blur}
\usetikzlibrary{shapes}

\geometry{top=1.0in,bottom=1.0in,left=1.0in,right=1.0in}
\newcommand{\subtitle}[1]{%
  \posttitle{%
    \par\end{center}
    \begin{center}\large#1\end{center}
    \vskip0.5em}%

}
\usepackage{hyperref}
\hypersetup{
colorlinks=true,
linkcolor=blue,
filecolor=magenta,      
urlcolor=blue,
citecolor=blue,
}


\title{Homework 6}
\date{\today}
\author{Michael Brodskiy\\ \small Professor: I. Salama}

\begin{document}

\maketitle

\begin{enumerate}

  \item

    \begin{enumerate}

      \item Per the rules of Laplace Transforms, we can convolve two signals by the rule that:

        $$y(t)=x_1(t)*x_2(t)\to Y(s)=X_1(s)X_2(s)$$

        As such, we may obtain:

        $$X_1(s)=\frac{1}{s+4}\quad\text{ and }\quad X_2(s)=\frac{1}{s+2}$$

        Now, we account for the shifts. We know that, for $x(t)\to x(t-t_o)$ the transform becomes $X(s)\to e^{st_o}X(s)$. Furthermore, we know that for $x(-t)\to X(-s)$. Thus, we find:

        $$X_1(s)=\frac{e^{-3s}}{-s+4}\quad\text{ and }\quad X_2(s)=\frac{e^{-2s}}{s+2}$$

        Multiplying together, we find:

        $$\boxed{Y(s)=\frac{e^{-5s}}{(4-s)(s+2)},\quad\text{ROC: } -2<\sigma<4}$$

      \item 

    \end{enumerate}

  \item

  \item

    \begin{enumerate}

      \item 

      \item 

      \item 

      \item 

    \end{enumerate}

  \item

    \begin{enumerate}

      \item 

      \item 

        \begin{enumerate}

          \item 

          \item 

          \item 

        \end{enumerate}

    \end{enumerate}

  \item

  \item

  \item

    \begin{enumerate}

      \item 

      \item 

    \end{enumerate}

  \item

    \begin{enumerate}

      \item 

      \item 

      \item 

      \item 

      \item 

      \item 

      \item 

    \end{enumerate}

  \item

    \begin{enumerate}

      \item 

      \item 

      \item 

      \item 

    \end{enumerate}

\end{enumerate}

\end{document}

