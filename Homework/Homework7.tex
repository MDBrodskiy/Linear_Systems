%%%%%%%%%%%%%%%%%%%%%%%%%%%%%%%%%%%%%%%%%%%%%%%%%%%%%%%%%%%%%%%%%%%%%%%%%%%%%%%%%%%%%%%%%%%%%%%%%%%%%%%%%%%%%%%%%%%%%%%%%%%%%%%%%%%%%%%%%%%%%%%%%%%%%%%%%%%%%%%%%%%
% Written By Michael Brodskiy
% Class: Fundamentals of Linear Systems
% Professor: I. Salama
%%%%%%%%%%%%%%%%%%%%%%%%%%%%%%%%%%%%%%%%%%%%%%%%%%%%%%%%%%%%%%%%%%%%%%%%%%%%%%%%%%%%%%%%%%%%%%%%%%%%%%%%%%%%%%%%%%%%%%%%%%%%%%%%%%%%%%%%%%%%%%%%%%%%%%%%%%%%%%%%%%%

\documentclass[12pt]{article} 
\usepackage{alphalph}
\usepackage[utf8]{inputenc}
\usepackage[russian,english]{babel}
\usepackage{titling}
\usepackage{amsmath}
\usepackage{graphicx}
\usepackage{enumitem}
\usepackage{amssymb}
\usepackage[super]{nth}
\usepackage{everysel}
\usepackage{ragged2e}
\usepackage{geometry}
\usepackage{multicol}
\usepackage{fancyhdr}
\usepackage{cancel}
\usepackage{siunitx}
\usepackage{physics}
\usepackage{tikz}
\usepackage{mathdots}
\usepackage{yhmath}
\usepackage{cancel}
\usepackage{color}
\usepackage{array}
\usepackage{multirow}
\usepackage{gensymb}
\usepackage{tabularx}
\usepackage{extarrows}
\usepackage{booktabs}
\usepackage{lastpage}
\usetikzlibrary{fadings}
\usetikzlibrary{patterns}
\usetikzlibrary{shadows.blur}
\usetikzlibrary{shapes}

\geometry{top=1.0in,bottom=1.0in,left=1.0in,right=1.0in}
\newcommand{\subtitle}[1]{%
  \posttitle{%
    \par\end{center}
    \begin{center}\large#1\end{center}
    \vskip0.5em}%

}
\usepackage{hyperref}
\hypersetup{
colorlinks=true,
linkcolor=blue,
filecolor=magenta,      
urlcolor=blue,
citecolor=blue,
}


\title{Homework 7}
\date{\today}
\author{Michael Brodskiy\\ \small Professor: I. Salama}

\begin{document}

\maketitle

\begin{enumerate}

  \item First and foremost, we see that the period is:

    $$T_o=6$$

    We know that the Fourier coefficient formula is given by:

    $$C_n=\frac{1}{T_o}\int_{-T_o/2}^{T_o/2}x(t)e^{-2\pi jnt/T_o}\,dt$$

    We substitute our known values to get:

    $$C_n=\frac{1}{6}\int_{-3}^{3} 2e^{-(\pi jnt)/3}\,dt$$
    $$C_n=\frac{1}{3}\int_{-1}^{1} e^{-(\pi jnt)/3}\,dt$$
    $$C_n=\frac{1}{3}\left(-\frac{3}{\pi jn} e^{-(\pi jnt)/3}\right)\Big|_{-1}^1$$

    We solve this to get:

    $$C_n=-\frac{1}{\pi jn} \left[e^{-(pi jn)/3} - e^{(\pi jn)/3}\right]$$
    $$\boxed{C_n=\frac{2}{n\pi}\sin\left( \frac{n\pi}{3} \right)}$$

    We then substitute to find the values given:

    $$\boxed{\left\{\begin{array}{ll} C_1&= .5513\\ C_2&= .2757\\ C_3&= 0\\ C_{-1}&= .5513\\ C_{-2}&= .2757\\C_{-3}&=0\end{array}}$$

      We then calculate $C_0$ separately:

      $$C_0=\frac{1}{6}\int_{-1}^{1}2\,dt$$
      $$C_0=\frac{1}{6}(2(1)-2(-1))$$
      $$\boxed{C_0=\frac{2}{3}}$$

  \item

    \begin{enumerate}

      \item We may use the transform table to write:

        $$Y(\omega)=\frac{j\omega}{(j\omega)^2+\omega_o^2}$$

        For a transform of $\cos(t)$. Combining this with the shifting property, we may write:

        $$\boxed{X(\omega)=\frac{j\omega + 3}{(j\omega+3)^2+25}}$$

      \item We may write the sinusoid as:

        $$Y(\omega)=\frac{\omega_o}{(j\omega)^2+\omega_o^2}$$

        We then apply the shifting property to get:

        $$X^1(\omega)=\frac{2}{(j\omega+1)^2+4}$$

        Finally, applying the $t$ term, we get:

        $$X(\omega)=j\frac{d}{d\omega}\left[ \frac{2}{(j\omega+1)^2+4} \right]$$
        $$X(\omega)=j\left[ \frac{2(-2j)}{[(j\omega+1)^2+4]^2} \right]$$
        $$\boxed{X(\omega)=\frac{4}{[(j\omega+1)^2+4]^2}}$$

    \end{enumerate}

  \item

    \begin{enumerate}

      \item We first list the relevant properties:

        $$x(t-t_o)\to e^{-j\omega t_o}X(j\omega)$$
        $$x(-t)\to X(-j\omega)$$

        Deconstructing the expression, we may write:

        $$X_1(\omega)=e^{-j\omega(4)}X(-j\omega)+e^{-j\omega(-4)}X(-j\omega)$$
        $$X_1(\omega)=X(-j\omega)[e^{-4j\omega}+e^{4j\omega}]$$

        We then use the property that:

        $$e^{-j\omega}+e^{j\omega}=2\cos(\omega)$$

        To get:

        $$\boxed{X_1(\omega)=2\cos(4\omega)X(-j\omega)}$$

      \item We first list the relevant properties:

        $$x(at)\to \frac{1}{|a|}X\left( \frac{j\omega}{a} \right)$$
        $$x(t-t_o)\to e^{-j\omega t_o}X(j\omega)$$
        $$tx(t)\to j\frac{d}{d\omega}[X(j\omega)]$$

        The expression given to us is of the form:

        $$x(t)=tx(at-t_o)$$

        We rewrite the expression as:

        $$x_2(t)=tx(a[t-t_o/a])$$

        Which lets us determine $a=3$ and $T_o=-2$. As such, we obtain:

        $$X_2(\omega)=j\frac{d}{d\omega}\left[ \frac{e^{\frac{2j\omega}{3}}}{3}X\left(\frac{j\omega}{3}\right) \right]$$
        $$\boxed{X_2(\omega)=\frac{j}{3}\frac{d}{d\omega}\left[ e^{\frac{2j\omega}{3}}X\left(\frac{j\omega}{3}\right) \right]}$$

      \item 

    \end{enumerate}

  \item

    \begin{enumerate}

      \item 

      \item 

      \item 

      \item 

      \item 

    \end{enumerate}

  \item

    \begin{enumerate}

      \item 

      \item 

      \item 

      \item 

    \end{enumerate}

  \item

    \begin{enumerate}

      \item 

      \item 

      \item 

    \end{enumerate}

  \item

    \begin{enumerate}

      \item 

      \item 

    \end{enumerate}

  \item

    \begin{enumerate}

      \item 

      \item 

    \end{enumerate}

\end{enumerate}

\end{document}

