%%%%%%%%%%%%%%%%%%%%%%%%%%%%%%%%%%%%%%%%%%%%%%%%%%%%%%%%%%%%%%%%%%%%%%%%%%%%%%%%%%%%%%%%%%%%%%%%%%%%%%%%%%%%%%%%%%%%%%%%%%%%%%%%%%%%%%%%%%%%%%%%%%%%%%%%%%%%%%%%%%%
% Written By Michael Brodskiy
% Class: Fundamentals of Linear Systems
% Professor: I. Salama
%%%%%%%%%%%%%%%%%%%%%%%%%%%%%%%%%%%%%%%%%%%%%%%%%%%%%%%%%%%%%%%%%%%%%%%%%%%%%%%%%%%%%%%%%%%%%%%%%%%%%%%%%%%%%%%%%%%%%%%%%%%%%%%%%%%%%%%%%%%%%%%%%%%%%%%%%%%%%%%%%%%

\documentclass[12pt]{article} 
\usepackage{alphalph}
\usepackage[utf8]{inputenc}
\usepackage[russian,english]{babel}
\usepackage{titling}
\usepackage{amsmath}
\usepackage{graphicx}
\usepackage{enumitem}
\usepackage{amssymb}
\usepackage[super]{nth}
\usepackage{everysel}
\usepackage{ragged2e}
\usepackage{geometry}
\usepackage{multicol}
\usepackage{fancyhdr}
\usepackage{cancel}
\usepackage{siunitx}
\usepackage{physics}
\usepackage{tikz}
\usepackage{mathdots}
\usepackage{yhmath}
\usepackage{cancel}
\usepackage{color}
\usepackage{array}
\usepackage{multirow}
\usepackage{gensymb}
\usepackage{tabularx}
\usepackage{extarrows}
\usepackage{booktabs}
\usepackage{lastpage}
\usetikzlibrary{fadings}
\usetikzlibrary{patterns}
\usetikzlibrary{shadows.blur}
\usetikzlibrary{shapes}

\geometry{top=1.0in,bottom=1.0in,left=1.0in,right=1.0in}
\newcommand{\subtitle}[1]{%
  \posttitle{%
    \par\end{center}
    \begin{center}\large#1\end{center}
    \vskip0.5em}%

}
\usepackage{hyperref}
\hypersetup{
colorlinks=true,
linkcolor=blue,
filecolor=magenta,      
urlcolor=blue,
citecolor=blue,
}


\title{Homework 8}
\date{\today}
\author{Michael Brodskiy\\ \small Professor: I. Salama}

\begin{document}

\maketitle

\begin{enumerate}

  \item

    \begin{enumerate}

      \item By the duality property, we may write:

        $$\cos(\omega_o t)\leftrightarrow \pi\left[ \delta(\omega-\omega_o)+\delta(\omega+\omega_o) \right]$$
        $$\pi\left[ \delta(t-t_o)+\delta(t+t_o) \right]\leftrightarrow 2\pi\cos(\omega t_o)$$

        In tandem with the time shifting property, we may write:

        $$\frac{1}{2}e^{-\alpha j\omega}\left[ \delta(t-t_o)+\delta(t+t_o) \right]\leftrightarrow \cos(\omega t_o-\alpha)$$

        This gives us:

        $$\boxed{x(t)=\frac{1}{2}e^{-\frac{\pi j \omega}{3}}\left[ \delta(t-4)+\delta(t+4) \right]}$$

      \item We may use the following known transform formulas:

        $$\pi\left[ \delta(\omega-\omega_o)+\delta(\omega+\omega_o) \right]\leftrightarrow \cos(\omega_o t)$$
        $$\frac{\pi}{j}\left[ \delta(\omega-\omega_o)-\delta(\omega+\omega_o) \right]\leftrightarrow \sin(\omega_o t)$$
        $$\frac{1}{(\alpha+j\omega)^2}\leftrightarrow te^{-\alpha t}u(t)$$

        In combination with the time shifting property, we may get:

        $$\boxed{x(t)=\frac{2}{\pi}\cos(5t-3)-4j\pi\sin(2\pi t)+te^{-2t}u(t)}$$

    \end{enumerate}

  \item

    \begin{enumerate}

      \item The signal $x(t)$ may be expressed in several parts: a step up by 1 at $t=-1$ to $t=1$, a slope of 1 between $-1$ and $1$, and a step up of $2$ for $t>1$. Thus, we get:

        $$\boxed{x(t)=r(t+1)-r(t-1)}$$

        This lets us find:

        $$\boxed{\frac{dx}{dt}=u(t+1)-u(t-1)}$$

        And finally:

        $$\boxed{\frac{d^2x}{dt^2}=\delta(t+1)-\delta(t-1)}$$

      \item The DC component can be accounted for using:

        $$x_{DC}=\lim_{T\to\infty}\frac{1}{T}\int_{-\frac{T}{2}}^{\frac{T}{2}} x(t)\,dt$$

        This gives:

        $$x_{DC}=\lim_{T\to\infty} \frac{1}{T}\left[ \int_{-1}^{1}(t+1)\,dt+\int_1^{\frac{T}{2}} 2\,dt \right]$$
        $$x_{DC}=\lim_{T\to\infty} \frac{1}{T}\left[ 2+2\left( \frac{T}{2}-1 \right) \right]$$
        $$x_{DC}=1$$

        Using the duality property, we may write:

        $$1\leftrightarrow 2\pi\delta(\omega)$$

        We then transform the second order derivative to see:

        $$\mathcal{F}\left\{ \frac{d^2x(t)}{dt^2} \right\}=e^{j\omega} -e^{-j\omega}$$
        $$(j\omega)^2X(\omega)=e^{j\omega} -e^{-j\omega}$$

        This gives us:

        $$X(j\omega)=-\frac{2j\sin(\omega)}{\omega^2}$$
          
        We then add in the DC component to get:

        $$\boxed{X(j\omega)=2\pi\delta(\omega)-\frac{2j\sin(\omega)}{\omega^2}}$$

      \item Given that the subtraction of the 1 removes the DC component, we simply get:

        $$\boxed{G(j\omega)=-\frac{2j\sin(\omega)}{\omega^2}}$$

    \end{enumerate}

    \setcounter{enumi}{3}

  \item

    \begin{enumerate}

      \item Per the theorem, we may write:

        $$\int_{-\infty}^{\infty}|x(t)|^2=\frac{1}{2\pi}\int_{-\infty}^{\infty} |X(\omega)|^2\,d\omega$$
        $$2\pi\int_{-\infty}^{\infty}|x(t)|^2=\int_{-\infty}^{\infty} |X(\omega)|^2\,d\omega$$

        This allows us to write:

        $$A=4\pi\int_{0}^{\infty}e^{-4t}\,dt$$
        $$-\pi \left(e^{-4t}\right)\Big|_0^{\infty}$$
        $$-\pi \left(e^{-4\infty}-1\right)$$

        Thus, we obtain:

        $$\boxed{A=\pi}$$

      \item Per our Fourier transform properties, we may write:

        $$tx(t)\to j\frac{d}{d\omega}X(\omega)$$

        For $y(t)=te^{-2|t|}$ this gives us:

        $$Y(j\omega)=j\frac{d}{d\omega}\left[ \frac{4}{(4+\omega^2)} \right]$$

        Differentiating gives us the final answer as:

        $$\boxed{Y(j\omega)=-\frac{8j\omega}{(4+\omega^2)^2}}$$

      \item By the duality property, we know that if $x(t)\leftrightarrow X(j\omega)$, then:

        $$x(t)\leftrightarrow 2\pi X(-j\omega)$$

        As such, we may write:

        $$-\frac{8jt}{(4+t^2)^2}\leftrightarrow 2\pi(-\omega)e^{-2|\omega|}$$
        $$\frac{t}{(4+t^2)^2}\leftrightarrow -j\pi\omega e^{-2|\omega|}$$

        Thus, we see that:

        $$\boxed{\mathcal{F}\left\{ \frac{4t}{(4+t^2)^2} \right\}=-j\pi\omega e^{-2|\omega|}}$$

    \end{enumerate}

  \item We may compute the response of the system using the convolution; the convolution may be more easily computed using the Fourier transform such that:

    $$y(t)=h(t)*x(t)\leftrightarrow Y(j\omega)=H(j\omega)X(j\omega)$$

    We can see that the given response may be written as:

    $$H(j\omega)=\left\{\begin{array}{ll} e^{-3j\omega},&|\omega|\leq4\\0,&\text{otherwise}\end{array}$$

    \begin{enumerate}

      \item We may see that the transform becomes:

        $$X_1(j\omega)=\pi e^{N\pi j\omega}\left[ \delta(\omega-10)+\delta(\omega+10) \right]$$

        We may observe that the response and signal do not have common values for which they are non-zero. This gives us:

        $$Y_1(j\omega)=0$$

        This ultimately means:

        $$\boxed{y_1(t)=0}$$

      \item We may see that the transform becomes:

        $$X_2(j\omega)=5\pi\left[ \delta(\omega-2)+\delta(\omega+2) \right]$$

        We multiply the two together to get:

        $$Y_2(j\omega)=5\pi e^{-3j\omega}\left[  \delta(\omega-2)+\delta(\omega+2)\right]$$

        We see this introduces a delay of three units, which gives us:

        $$\boxed{y_2(t)=5\cos(2(t-3))}$$

      \item We may see that the transform becomes:

        $$X_3(j\omega)=\left\{\begin{array}{ll} 1,&|\omega|\leq1\\0,&\text{otherwise}\end{array}$$

          Multiplying the two together creates a delayed sink function, but with the boundaries of the input:

        $$Y_3(j\omega)=\left\{\begin{array}{ll} e^{-3j\omega},&|\omega|\leq1\\0,&\text{otherwise}\end{array}$$

          Transforming back, we get:

          $$\boxed{y_3(t)=\frac{\sin(t-3)}{\pi(t-3)}}$$

        \item We may observe that the input signals within the passband of the filter simply introduce a delay at the output, thus, we may conclude:

          $$\boxed{y_4(t)=\left( \frac{\sin(t-3)}{\pi(t-3)} \right)^2}$$

    \end{enumerate}

  \item

    \begin{enumerate}

      \item We may apply the Laplace transform to get:

        $$s^2Y(s)+5sY(s)+4Y(s)=3X(s)$$

        Since we know that the response is the output over input, we may write:

        $$H(s)=\frac{3}{s^2+5s+4}$$
        $$H(s)=\frac{3}{(s+4)(s+1)}$$

        We use partial fraction decomposition in order to be able to apply the reverse transform. This gets us:

        $$H(s)=-\frac{1}{s+1}+\frac{1}{s+4}$$

        We take the inverse transform to conclude:

        $$\boxed{h(t)=[-e^{-t}+e^{-4t}]u(t)}$$

      \item Given the input $x(t)$, we may write:

        $$X(s)=\frac{1}{s+1}$$

        We know the response may be written as:

        $$Y(s)=H(s)X(s)$$

        And so we get:

        $$Y(s)=\frac{3}{(s+4)(s+1)^2}$$

        We once again use partial fraction decomposition to write:

        $$Y(s)=\frac{A}{s+1}+\frac{B}{(s+1)^2}+\frac{C}{s+4}$$
        $$Y(s)=\frac{-1}{3(s+1)}+\frac{1}{(s+1)^2}+\frac{1}{3(s+4)}$$

        We take the inverse transform to get:

        $$\boxed{y(t)=\left[-\frac{1}{3}e^{-t}+\frac{1}{3}e^{-4t}+te^{-t}\right]u(t)}$$

    \end{enumerate}

    \setcounter{enumi}{7}

  \item

    \begin{enumerate}

      \item We may take the Laplace transform of both to get:

        $$X(s)=\frac{1}{s+1}+\frac{2}{s+4}$$
        $$Y(s)=\frac{3}{s+1}-\frac{3}{s+3}$$

        We may combine the fractions to get:

        $$X(s)=\frac{3s+6}{(s+1)(s+4)}$$
        $$Y(s)=\frac{6}{(s+1)(s+3)}$$
        
        We know the response may be written as:

        $$H(s)=\frac{Y(s)}{X(s)}$$

        This gives us:

        $$H(s)=\frac{6}{(s+1)(s+3)}\cdot\frac{(s+1)(s+4)}{3s+6}$$
        $$H(s)=\frac{2s+8}{(s+2)(s+3)}$$

        Given that $s=j\omega$, we may write:

        $$H(j\omega)=\frac{2j\omega+8}{(j\omega+2)(j\omega+3)}$$

        We may simplify to get:

        $$\boxed{H(j\omega)=\frac{2j\omega+8}{6+5j\omega-\omega^2}}$$

      \item Using partial fraction decomposition, we may write:

        $$H(s)=\frac{A}{s+2}+\frac{B}{s+3}$$

        We may find that $A=4$ and $B=-2$ to get:

        $$H(s)=\frac{4}{s+2}-\frac{2}{s+3}$$

        Taking the inverse transform, we find:

        $$\boxed{h(t)=[4e^{-2t}-2e^{-3t}]u(t)}$$

      \item From part (a), we know:

        $$\frac{Y(s)}{X(s)}=\frac{2s+8}{(s+2)(s+3)}$$
        $$Y(s)[s^2+5s+6]=X(s)[2s+8]$$

        Taking the inverse transform, we get:

        $$\boxed{\frac{d^2y(t)}{dt}+5\frac{dy(t)}{dt}+6y(t)=2\frac{dx(t)}{dt}+8x(t)}$$

      \item For an inverse system, we know:

        $$H(s)H_i(s)=1$$

        As such, we get:

        $$H_i(s)=\frac{(s+2)(s+3)}{2s+8}$$

        Taking $s\to j\omega$, we get:

        $$\boxed{H_i(j\omega)=\frac{(j\omega+2)(j\omega+3)}{2j\omega+8}}$$

    \end{enumerate}

  \item To plot the signals, we may begin by converting them to their Fourier form. Moving in order, we may begin with $v_1(t)$ to observe:

    $$v_1(t)=x(t)\cdot2\cos(\omega_c t)$$
    $$V_1(j\omega)=2\pi\delta(\omega-\omega_c)+\delta(\omega+\omega-c)X(j\omega)$$

    Since, at $\omega_c$, $X(j\omega)=.5$, This gives us:

    \begin{figure}[H]
      \centering
      \tikzset{every picture/.style={line width=0.75pt}} %set default line width to 0.75pt        

\begin{tikzpicture}[x=0.75pt,y=0.75pt,yscale=-1,xscale=1]
%uncomment if require: \path (0,300); %set diagram left start at 0, and has height of 300

%Shape: Axis 2D [id:dp4364414837639843] 
\draw  (301,209.4) -- (493,209.4)(320.2,69) -- (320.2,225) (486,204.4) -- (493,209.4) -- (486,214.4) (315.2,76) -- (320.2,69) -- (325.2,76)  ;
%Shape: Axis 2D [id:dp20109394216457532] 
\draw  (339.4,209.4) -- (147.4,209.4)(320.2,69) -- (320.2,225) (154.4,204.4) -- (147.4,209.4) -- (154.4,214.4) (325.2,76) -- (320.2,69) -- (315.2,76)  ;
%Straight Lines [id:da7559993359973315] 
\draw    (233,204.29) -- (233,214.71) ;
%Straight Lines [id:da21670318107864883] 
\draw    (407.4,204.8) -- (407.4,215.22) ;
%Straight Lines [id:da3439489556285271] 
\draw    (233,129.22) -- (233,204.29) ;
\draw [shift={(233,126.87)}, rotate = 90] [color={rgb, 255:red, 0; green, 0; blue, 0 }  ][line width=0.75]      (0, 0) circle [x radius= 3.35, y radius= 3.35]   ;
%Straight Lines [id:da9964346805717921] 
\draw    (407.4,129.73) -- (407.4,204.8) ;
\draw [shift={(407.4,127.38)}, rotate = 90] [color={rgb, 255:red, 0; green, 0; blue, 0 }  ][line width=0.75]      (0, 0) circle [x radius= 3.35, y radius= 3.35]   ;
%Straight Lines [id:da19666501065127062] 
\draw  [dash pattern={on 0.84pt off 2.51pt}]  (404.4,127.38) -- (236,126.87) ;

% Text Node
\draw (233,218.11) node [anchor=north] [inner sep=0.75pt]    {$-\omega _{c}$};
% Text Node
\draw (407.4,218.62) node [anchor=north] [inner sep=0.75pt]    {$\omega _{c}$};
% Text Node
\draw (502.4,210.62) node [anchor=north] [inner sep=0.75pt]    {$\omega $};
% Text Node
\draw (320.4,46.62) node [anchor=north] [inner sep=0.75pt]    {$V_{1}( j\omega )$};
% Text Node
\draw (322.2,130.52) node [anchor=north west][inner sep=0.75pt]    {$\pi $};


\end{tikzpicture}

      \caption{Spectrum of $V_1(j\omega)$}
      \label{fig:1}
    \end{figure}

    As a result of the filter, we may see that:
    
    $$W_1(j\omega)=2\pi\left[ \delta(\omega-\omega_c)+\delta(\omega-\omega_c) \right]$$

    Changing $\omega_c$ to $\omega_o$ per the given relationship, we get:

    \begin{figure}[H]
      \centering
      \tikzset{every picture/.style={line width=0.75pt}} %set default line width to 0.75pt        

\begin{tikzpicture}[x=0.75pt,y=0.75pt,yscale=-1,xscale=1]
%uncomment if require: \path (0,300); %set diagram left start at 0, and has height of 300

%Shape: Axis 2D [id:dp4364414837639843] 
\draw  (301,209.4) -- (493,209.4)(320.2,69) -- (320.2,225) (486,204.4) -- (493,209.4) -- (486,214.4) (315.2,76) -- (320.2,69) -- (325.2,76)  ;
%Shape: Axis 2D [id:dp20109394216457532] 
\draw  (339.4,209.4) -- (147.4,209.4)(320.2,69) -- (320.2,225) (154.4,204.4) -- (147.4,209.4) -- (154.4,214.4) (325.2,76) -- (320.2,69) -- (315.2,76)  ;
%Straight Lines [id:da7559993359973315] 
\draw    (233,204.29) -- (233,214.71) ;
%Straight Lines [id:da21670318107864883] 
\draw    (407.4,204.8) -- (407.4,215.22) ;
%Straight Lines [id:da3439489556285271] 
\draw    (233,129.22) -- (233,204.29) ;
\draw [shift={(233,126.87)}, rotate = 90] [color={rgb, 255:red, 0; green, 0; blue, 0 }  ][line width=0.75]      (0, 0) circle [x radius= 3.35, y radius= 3.35]   ;
%Straight Lines [id:da9964346805717921] 
\draw    (407.4,129.73) -- (407.4,204.8) ;
\draw [shift={(407.4,127.38)}, rotate = 90] [color={rgb, 255:red, 0; green, 0; blue, 0 }  ][line width=0.75]      (0, 0) circle [x radius= 3.35, y radius= 3.35]   ;

% Text Node
\draw (233,218.11) node [anchor=north] [inner sep=0.75pt]    {$-\omega _{o}$};
% Text Node
\draw (407.4,218.62) node [anchor=north] [inner sep=0.75pt]    {$\omega _{o}$};
% Text Node
\draw (502.4,210.62) node [anchor=north] [inner sep=0.75pt]    {$\omega $};
% Text Node
\draw (320.4,46.62) node [anchor=north] [inner sep=0.75pt]    {$W_{1}( j\omega )$};


\end{tikzpicture}

      \caption{Spectrum of $W_1(j\omega)$}
      \label{fig:2}
    \end{figure}

    We then check $V_2(j\omega)$ to see that, because of the flipped sign in the transform of the $\sin$ term, the negative corner frequency has negative magnitude:

    \begin{figure}[H]
      \centering
      \tikzset{every picture/.style={line width=0.75pt}} %set default line width to 0.75pt        

\begin{tikzpicture}[x=0.75pt,y=0.75pt,yscale=-1,xscale=1]
%uncomment if require: \path (0,300); %set diagram left start at 0, and has height of 300

%Shape: Axis 2D [id:dp4364414837639843] 
\draw  (301,209.4) -- (493,209.4)(320.2,69) -- (320.2,225) (486,204.4) -- (493,209.4) -- (486,214.4) (315.2,76) -- (320.2,69) -- (325.2,76)  ;
%Shape: Axis 2D [id:dp20109394216457532] 
\draw  (339.4,209.4) -- (147.4,209.4)(320.2,69) -- (320.2,225) (154.4,204.4) -- (147.4,209.4) -- (154.4,214.4) (325.2,76) -- (320.2,69) -- (315.2,76)  ;
%Straight Lines [id:da7559993359973315] 
\draw    (233,204.29) -- (233,214.71) ;
%Straight Lines [id:da21670318107864883] 
\draw    (407.4,204.8) -- (407.4,215.22) ;
%Straight Lines [id:da3439489556285271] 
\draw    (233,289.78) -- (233,214.71) ;
\draw [shift={(233,292.13)}, rotate = 270] [color={rgb, 255:red, 0; green, 0; blue, 0 }  ][line width=0.75]      (0, 0) circle [x radius= 3.35, y radius= 3.35]   ;
%Straight Lines [id:da9964346805717921] 
\draw    (407.4,129.73) -- (407.4,204.8) ;
\draw [shift={(407.4,127.38)}, rotate = 90] [color={rgb, 255:red, 0; green, 0; blue, 0 }  ][line width=0.75]      (0, 0) circle [x radius= 3.35, y radius= 3.35]   ;
%Straight Lines [id:da19666501065127062] 
\draw  [dash pattern={on 0.84pt off 2.51pt}]  (404.4,127.38) -- (320,126.87) ;
%Straight Lines [id:da44127395188037755] 
\draw  [dash pattern={on 0.84pt off 2.51pt}]  (317.4,292.64) -- (233,292.13) ;
%Straight Lines [id:da8182769605122995] 
\draw [line width=0.75]    (319.4,291.93) -- (320.2,209.4) ;
%Straight Lines [id:da4314214379435235] 
\draw [line width=0.75]    (319.4,291.93) -- (320.2,209.4) ;

% Text Node
\draw (235,200.89) node [anchor=south west] [inner sep=0.75pt]    {$-\omega _{c}$};
% Text Node
\draw (407.4,218.62) node [anchor=north] [inner sep=0.75pt]    {$\omega _{c}$};
% Text Node
\draw (502.4,210.62) node [anchor=north] [inner sep=0.75pt]    {$\omega $};
% Text Node
\draw (320.4,46.62) node [anchor=north] [inner sep=0.75pt]    {$V_{2}( j\omega )$};
% Text Node
\draw (322.2,130.52) node [anchor=north west][inner sep=0.75pt]    {$\pi /j$};
% Text Node
\draw (317.4,288.53) node [anchor=south east] [inner sep=0.75pt]    {$-\pi /j$};


\end{tikzpicture}

      \caption{Spectrum of $V_2(j\omega)$}
      \label{fig:3}
    \end{figure}

    Because of the filter, however, we see that the spectrum of $W_2(j\omega)$ remains the same as that of $W_1(j\omega)$:

    \begin{figure}[H]
      \centering
      \tikzset{every picture/.style={line width=0.75pt}} %set default line width to 0.75pt        

\begin{tikzpicture}[x=0.75pt,y=0.75pt,yscale=-1,xscale=1]
%uncomment if require: \path (0,300); %set diagram left start at 0, and has height of 300

%Shape: Axis 2D [id:dp4364414837639843] 
\draw  (301,209.4) -- (493,209.4)(320.2,69) -- (320.2,225) (486,204.4) -- (493,209.4) -- (486,214.4) (315.2,76) -- (320.2,69) -- (325.2,76)  ;
%Shape: Axis 2D [id:dp20109394216457532] 
\draw  (339.4,209.4) -- (147.4,209.4)(320.2,69) -- (320.2,225) (154.4,204.4) -- (147.4,209.4) -- (154.4,214.4) (325.2,76) -- (320.2,69) -- (315.2,76)  ;
%Straight Lines [id:da7559993359973315] 
\draw    (233,204.29) -- (233,214.71) ;
%Straight Lines [id:da21670318107864883] 
\draw    (407.4,204.8) -- (407.4,215.22) ;
%Straight Lines [id:da3439489556285271] 
\draw    (233,129.22) -- (233,204.29) ;
\draw [shift={(233,126.87)}, rotate = 90] [color={rgb, 255:red, 0; green, 0; blue, 0 }  ][line width=0.75]      (0, 0) circle [x radius= 3.35, y radius= 3.35]   ;
%Straight Lines [id:da9964346805717921] 
\draw    (407.4,129.73) -- (407.4,204.8) ;
\draw [shift={(407.4,127.38)}, rotate = 90] [color={rgb, 255:red, 0; green, 0; blue, 0 }  ][line width=0.75]      (0, 0) circle [x radius= 3.35, y radius= 3.35]   ;

% Text Node
\draw (233,218.11) node [anchor=north] [inner sep=0.75pt]    {$-\omega _{o}$};
% Text Node
\draw (407.4,218.62) node [anchor=north] [inner sep=0.75pt]    {$\omega _{o}$};
% Text Node
\draw (502.4,210.62) node [anchor=north] [inner sep=0.75pt]    {$\omega $};
% Text Node
\draw (320.4,46.62) node [anchor=north] [inner sep=0.75pt]    {$W_{2}( j\omega )$};


\end{tikzpicture}

      \caption{Spectrum of $W_2(j\omega)$}
      \label{fig:4}
    \end{figure}

    We see that combining the signals $W_{1,2}$ with the sinusoids results in Fourier responses consisting of rect functions, centered at $\pm\omega_c$, with a width of $2\omega_o$:

    \begin{figure}[H]
      \centering
      \tikzset{every picture/.style={line width=0.75pt}} %set default line width to 0.75pt        

\begin{tikzpicture}[x=0.75pt,y=0.75pt,yscale=-1,xscale=1]
%uncomment if require: \path (0,300); %set diagram left start at 0, and has height of 300

%Shape: Axis 2D [id:dp4364414837639843] 
\draw  (301,209.4) -- (493,209.4)(320.2,69) -- (320.2,225) (486,204.4) -- (493,209.4) -- (486,214.4) (315.2,76) -- (320.2,69) -- (325.2,76)  ;
%Shape: Axis 2D [id:dp20109394216457532] 
\draw  (339.4,209.4) -- (147.4,209.4)(320.2,69) -- (320.2,225) (154.4,204.4) -- (147.4,209.4) -- (154.4,214.4) (325.2,76) -- (320.2,69) -- (315.2,76)  ;
%Straight Lines [id:da7559993359973315] 
\draw    (233,204.29) -- (233,214.71) ;
%Straight Lines [id:da21670318107864883] 
\draw    (407.4,204.8) -- (407.4,215.22) ;
%Straight Lines [id:da3439489556285271] 
\draw    (194.29,131.87) -- (194.29,209.29) ;
%Straight Lines [id:da19666501065127062] 
\draw  [dash pattern={on 0.84pt off 2.51pt}]  (368.69,132.38) -- (271.71,131.87) ;
%Straight Lines [id:da5487354351298724] 
\draw    (271.71,131.87) -- (194.29,131.87) ;
%Straight Lines [id:da1866677761933241] 
\draw    (271.71,131.87) -- (271.71,209.29) ;
%Straight Lines [id:da45800765996237425] 
\draw    (446.11,132.38) -- (368.69,132.38) ;
%Straight Lines [id:da07434960977090421] 
\draw    (368.69,132.38) -- (368.69,209.8) ;
%Straight Lines [id:da2890856974084346] 
\draw    (446.11,132.38) -- (446.11,209.8) ;
%Straight Lines [id:da34528122862725874] 
\draw    (196.29,121.87) -- (269.71,121.87) ;
\draw [shift={(271.71,121.87)}, rotate = 180] [color={rgb, 255:red, 0; green, 0; blue, 0 }  ][line width=0.75]    (10.93,-3.29) .. controls (6.95,-1.4) and (3.31,-0.3) .. (0,0) .. controls (3.31,0.3) and (6.95,1.4) .. (10.93,3.29)   ;
\draw [shift={(194.29,121.87)}, rotate = 0] [color={rgb, 255:red, 0; green, 0; blue, 0 }  ][line width=0.75]    (10.93,-3.29) .. controls (6.95,-1.4) and (3.31,-0.3) .. (0,0) .. controls (3.31,0.3) and (6.95,1.4) .. (10.93,3.29)   ;
%Straight Lines [id:da4595793235204215] 
\draw    (370.69,122.38) -- (444.11,122.38) ;
\draw [shift={(446.11,122.38)}, rotate = 180] [color={rgb, 255:red, 0; green, 0; blue, 0 }  ][line width=0.75]    (10.93,-3.29) .. controls (6.95,-1.4) and (3.31,-0.3) .. (0,0) .. controls (3.31,0.3) and (6.95,1.4) .. (10.93,3.29)   ;
\draw [shift={(368.69,122.38)}, rotate = 0] [color={rgb, 255:red, 0; green, 0; blue, 0 }  ][line width=0.75]    (10.93,-3.29) .. controls (6.95,-1.4) and (3.31,-0.3) .. (0,0) .. controls (3.31,0.3) and (6.95,1.4) .. (10.93,3.29)   ;

% Text Node
\draw (233,218.11) node [anchor=north] [inner sep=0.75pt]    {$-\omega _{c}$};
% Text Node
\draw (407.4,218.62) node [anchor=north] [inner sep=0.75pt]    {$\omega _{c}$};
% Text Node
\draw (502.4,210.62) node [anchor=north] [inner sep=0.75pt]    {$\omega $};
% Text Node
\draw (320.4,46.62) node [anchor=north] [inner sep=0.75pt]    {$Z_{1}( j\omega )$};
% Text Node
\draw (322.2,135.52) node [anchor=north west][inner sep=0.75pt]    {$1/2$};
% Text Node
\draw (233,118.47) node [anchor=south] [inner sep=0.75pt]    {$2\omega _{o}$};
% Text Node
\draw (407.4,118.98) node [anchor=south] [inner sep=0.75pt]    {$2\omega _{o}$};


\end{tikzpicture}

      \caption{Spectrum of $Z_1(j\omega)$}
      \label{fig:5}
    \end{figure}

    \begin{figure}[H]
      \centering
      \tikzset{every picture/.style={line width=0.75pt}} %set default line width to 0.75pt        

\begin{tikzpicture}[x=0.75pt,y=0.75pt,yscale=-1,xscale=1]
%uncomment if require: \path (0,300); %set diagram left start at 0, and has height of 300

%Shape: Axis 2D [id:dp4364414837639843] 
\draw  (301,209.4) -- (493,209.4)(320.2,69) -- (320.2,225) (486,204.4) -- (493,209.4) -- (486,214.4) (315.2,76) -- (320.2,69) -- (325.2,76)  ;
%Shape: Axis 2D [id:dp20109394216457532] 
\draw  (339.4,209.4) -- (147.4,209.4)(320.2,69) -- (320.2,225) (154.4,204.4) -- (147.4,209.4) -- (154.4,214.4) (325.2,76) -- (320.2,69) -- (315.2,76)  ;
%Straight Lines [id:da7559993359973315] 
\draw    (233,204.29) -- (233,214.71) ;
%Straight Lines [id:da21670318107864883] 
\draw    (407.4,204.8) -- (407.4,215.22) ;
%Straight Lines [id:da3439489556285271] 
\draw    (194.29,131.87) -- (194.29,209.29) ;
%Straight Lines [id:da19666501065127062] 
\draw  [dash pattern={on 0.84pt off 2.51pt}]  (368.69,132.38) -- (271.71,131.87) ;
%Straight Lines [id:da5487354351298724] 
\draw    (271.71,131.87) -- (194.29,131.87) ;
%Straight Lines [id:da1866677761933241] 
\draw    (271.71,131.87) -- (271.71,209.29) ;
%Straight Lines [id:da45800765996237425] 
\draw    (446.11,132.38) -- (368.69,132.38) ;
%Straight Lines [id:da07434960977090421] 
\draw    (368.69,132.38) -- (368.69,209.8) ;
%Straight Lines [id:da2890856974084346] 
\draw    (446.11,132.38) -- (446.11,209.8) ;
%Straight Lines [id:da34528122862725874] 
\draw    (196.29,121.87) -- (269.71,121.87) ;
\draw [shift={(271.71,121.87)}, rotate = 180] [color={rgb, 255:red, 0; green, 0; blue, 0 }  ][line width=0.75]    (10.93,-3.29) .. controls (6.95,-1.4) and (3.31,-0.3) .. (0,0) .. controls (3.31,0.3) and (6.95,1.4) .. (10.93,3.29)   ;
\draw [shift={(194.29,121.87)}, rotate = 0] [color={rgb, 255:red, 0; green, 0; blue, 0 }  ][line width=0.75]    (10.93,-3.29) .. controls (6.95,-1.4) and (3.31,-0.3) .. (0,0) .. controls (3.31,0.3) and (6.95,1.4) .. (10.93,3.29)   ;
%Straight Lines [id:da4595793235204215] 
\draw    (370.69,122.38) -- (444.11,122.38) ;
\draw [shift={(446.11,122.38)}, rotate = 180] [color={rgb, 255:red, 0; green, 0; blue, 0 }  ][line width=0.75]    (10.93,-3.29) .. controls (6.95,-1.4) and (3.31,-0.3) .. (0,0) .. controls (3.31,0.3) and (6.95,1.4) .. (10.93,3.29)   ;
\draw [shift={(368.69,122.38)}, rotate = 0] [color={rgb, 255:red, 0; green, 0; blue, 0 }  ][line width=0.75]    (10.93,-3.29) .. controls (6.95,-1.4) and (3.31,-0.3) .. (0,0) .. controls (3.31,0.3) and (6.95,1.4) .. (10.93,3.29)   ;

% Text Node
\draw (233,218.11) node [anchor=north] [inner sep=0.75pt]    {$-\omega _{c}$};
% Text Node
\draw (407.4,218.62) node [anchor=north] [inner sep=0.75pt]    {$\omega _{c}$};
% Text Node
\draw (502.4,210.62) node [anchor=north] [inner sep=0.75pt]    {$\omega $};
% Text Node
\draw (320.4,46.62) node [anchor=north] [inner sep=0.75pt]    {$Z_{2}( j\omega )$};
% Text Node
\draw (322.2,135.52) node [anchor=north west][inner sep=0.75pt]    {$1/2$};
% Text Node
\draw (233,118.47) node [anchor=south] [inner sep=0.75pt]    {$2\omega _{o}$};
% Text Node
\draw (407.4,118.98) node [anchor=south] [inner sep=0.75pt]    {$2\omega _{o}$};


\end{tikzpicture}

      \caption{Spectrum of $Z_2(j\omega)$}
      \label{fig:6}
    \end{figure}

    Summing the signals gives us a height of 1, which indicates that this is a bandpass filter with center frequency $\omega_c$ and bandwidth $2\omega_o$:

    \begin{figure}[H]
      \centering
      \tikzset{every picture/.style={line width=0.75pt}} %set default line width to 0.75pt        

\begin{tikzpicture}[x=0.75pt,y=0.75pt,yscale=-1,xscale=1]
%uncomment if require: \path (0,300); %set diagram left start at 0, and has height of 300

%Shape: Axis 2D [id:dp4364414837639843] 
\draw  (301,209.4) -- (493,209.4)(320.2,69) -- (320.2,225) (486,204.4) -- (493,209.4) -- (486,214.4) (315.2,76) -- (320.2,69) -- (325.2,76)  ;
%Shape: Axis 2D [id:dp20109394216457532] 
\draw  (339.4,209.4) -- (147.4,209.4)(320.2,69) -- (320.2,225) (154.4,204.4) -- (147.4,209.4) -- (154.4,214.4) (325.2,76) -- (320.2,69) -- (315.2,76)  ;
%Straight Lines [id:da7559993359973315] 
\draw    (233,204.29) -- (233,214.71) ;
%Straight Lines [id:da21670318107864883] 
\draw    (407.4,204.8) -- (407.4,215.22) ;
%Straight Lines [id:da3439489556285271] 
\draw    (194.29,131.87) -- (194.29,209.29) ;
%Straight Lines [id:da19666501065127062] 
\draw  [dash pattern={on 0.84pt off 2.51pt}]  (368.69,132.38) -- (271.71,131.87) ;
%Straight Lines [id:da5487354351298724] 
\draw    (271.71,131.87) -- (194.29,131.87) ;
%Straight Lines [id:da1866677761933241] 
\draw    (271.71,131.87) -- (271.71,209.29) ;
%Straight Lines [id:da45800765996237425] 
\draw    (446.11,132.38) -- (368.69,132.38) ;
%Straight Lines [id:da07434960977090421] 
\draw    (368.69,132.38) -- (368.69,209.8) ;
%Straight Lines [id:da2890856974084346] 
\draw    (446.11,132.38) -- (446.11,209.8) ;
%Straight Lines [id:da34528122862725874] 
\draw    (196.29,121.87) -- (269.71,121.87) ;
\draw [shift={(271.71,121.87)}, rotate = 180] [color={rgb, 255:red, 0; green, 0; blue, 0 }  ][line width=0.75]    (10.93,-3.29) .. controls (6.95,-1.4) and (3.31,-0.3) .. (0,0) .. controls (3.31,0.3) and (6.95,1.4) .. (10.93,3.29)   ;
\draw [shift={(194.29,121.87)}, rotate = 0] [color={rgb, 255:red, 0; green, 0; blue, 0 }  ][line width=0.75]    (10.93,-3.29) .. controls (6.95,-1.4) and (3.31,-0.3) .. (0,0) .. controls (3.31,0.3) and (6.95,1.4) .. (10.93,3.29)   ;
%Straight Lines [id:da4595793235204215] 
\draw    (370.69,122.38) -- (444.11,122.38) ;
\draw [shift={(446.11,122.38)}, rotate = 180] [color={rgb, 255:red, 0; green, 0; blue, 0 }  ][line width=0.75]    (10.93,-3.29) .. controls (6.95,-1.4) and (3.31,-0.3) .. (0,0) .. controls (3.31,0.3) and (6.95,1.4) .. (10.93,3.29)   ;
\draw [shift={(368.69,122.38)}, rotate = 0] [color={rgb, 255:red, 0; green, 0; blue, 0 }  ][line width=0.75]    (10.93,-3.29) .. controls (6.95,-1.4) and (3.31,-0.3) .. (0,0) .. controls (3.31,0.3) and (6.95,1.4) .. (10.93,3.29)   ;

% Text Node
\draw (233,218.11) node [anchor=north] [inner sep=0.75pt]    {$-\omega _{c}$};
% Text Node
\draw (407.4,218.62) node [anchor=north] [inner sep=0.75pt]    {$\omega _{c}$};
% Text Node
\draw (502.4,210.62) node [anchor=north] [inner sep=0.75pt]    {$\omega $};
% Text Node
\draw (320.4,46.62) node [anchor=north] [inner sep=0.75pt]    {$Y( j\omega )$};
% Text Node
\draw (322.2,135.52) node [anchor=north west][inner sep=0.75pt]    {$1$};
% Text Node
\draw (233,118.47) node [anchor=south] [inner sep=0.75pt]    {$2\omega _{o}$};
% Text Node
\draw (407.4,118.98) node [anchor=south] [inner sep=0.75pt]    {$2\omega _{o}$};


\end{tikzpicture}

      \caption{Spectrum of $Y(j\omega)$}
      \label{fig:7}
    \end{figure}

  \item We may write the equivalent impedances of all of the components as:

    $$Z_L=.5s\quad\text{ and }Z_C=\frac{1}{2s}$$

    The equivalent impedance becomes:

    $$Z_{eq}=.5s+1+\frac{1}{2s}$$

    The current becomes:

    $$I(s)=\frac{X(s)}{Z_{eq}}$$

    Which means that the voltage across the capacitor, $y(t)$ is:

    $$Y(s)=\frac{\frac{1}{2s}X(s)}{.5s+1+\frac{1}{2s}}$$

    Since the response is output over input, we get:

    $$H(s)=\frac{1}{s^2+2s+1}$$
    $$H(s)=\frac{1}{(s+1)^2}$$

    This gives us the final response as:

    $$\boxed{h(t)=te^{-t}u(t)}$$

\end{enumerate}

\end{document}

