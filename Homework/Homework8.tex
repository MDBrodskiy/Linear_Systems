%%%%%%%%%%%%%%%%%%%%%%%%%%%%%%%%%%%%%%%%%%%%%%%%%%%%%%%%%%%%%%%%%%%%%%%%%%%%%%%%%%%%%%%%%%%%%%%%%%%%%%%%%%%%%%%%%%%%%%%%%%%%%%%%%%%%%%%%%%%%%%%%%%%%%%%%%%%%%%%%%%%
% Written By Michael Brodskiy
% Class: Fundamentals of Linear Systems
% Professor: I. Salama
%%%%%%%%%%%%%%%%%%%%%%%%%%%%%%%%%%%%%%%%%%%%%%%%%%%%%%%%%%%%%%%%%%%%%%%%%%%%%%%%%%%%%%%%%%%%%%%%%%%%%%%%%%%%%%%%%%%%%%%%%%%%%%%%%%%%%%%%%%%%%%%%%%%%%%%%%%%%%%%%%%%

\documentclass[12pt]{article} 
\usepackage{alphalph}
\usepackage[utf8]{inputenc}
\usepackage[russian,english]{babel}
\usepackage{titling}
\usepackage{amsmath}
\usepackage{graphicx}
\usepackage{enumitem}
\usepackage{amssymb}
\usepackage[super]{nth}
\usepackage{everysel}
\usepackage{ragged2e}
\usepackage{geometry}
\usepackage{multicol}
\usepackage{fancyhdr}
\usepackage{cancel}
\usepackage{siunitx}
\usepackage{physics}
\usepackage{tikz}
\usepackage{mathdots}
\usepackage{yhmath}
\usepackage{cancel}
\usepackage{color}
\usepackage{array}
\usepackage{multirow}
\usepackage{gensymb}
\usepackage{tabularx}
\usepackage{extarrows}
\usepackage{booktabs}
\usepackage{lastpage}
\usetikzlibrary{fadings}
\usetikzlibrary{patterns}
\usetikzlibrary{shadows.blur}
\usetikzlibrary{shapes}

\geometry{top=1.0in,bottom=1.0in,left=1.0in,right=1.0in}
\newcommand{\subtitle}[1]{%
  \posttitle{%
    \par\end{center}
    \begin{center}\large#1\end{center}
    \vskip0.5em}%

}
\usepackage{hyperref}
\hypersetup{
colorlinks=true,
linkcolor=blue,
filecolor=magenta,      
urlcolor=blue,
citecolor=blue,
}


\title{Homework 8}
\date{\today}
\author{Michael Brodskiy\\ \small Professor: I. Salama}

\begin{document}

\maketitle

\begin{enumerate}

  \item

    \begin{enumerate}

      \item 

      \item 

    \end{enumerate}

  \item

    \begin{enumerate}

      \item The signal $x(t)$ may be expressed in several parts: a step up by 1 at $t=-1$ to $t=1$, a slope of 1 between $-1$ and $1$, and a step up of $2$ for $t>1$. Thus, we get:

        $$\boxed{x(t)=u(t+1)+r(t+1)+u(t-1)-r(t-1)}$$

        This lets us find:

        $$\boxed{\frac{dx}{dt}=\delta(t+1)+u(t+1)+\delta(t-1)-u(t-1)}$$

        And finally:

        $$\boxed{\frac{d^2x}{dt^2}=\delta(t+1)-\delta(t-1)}$$

      \item 

      \item 

    \end{enumerate}

    \setcounter{enumi}{3}

  \item

    \begin{enumerate}

      \item Per the theorem, we may write:

        $$\int_{-\infty}^{\infty}|x(t)|^2=\frac{1}{2\pi}\int_{-\infty}^{\infty} |X(\omega)|^2\,d\omega$$
        $$2\pi\int_{-\infty}^{\infty}|x(t)|^2=\int_{-\infty}^{\infty} |X(\omega)|^2\,d\omega$$

        This allows us to write:

        $$A=4\pi\int_{0}^{\infty}e^{-4t}\,dt$$
        $$-\pi \left(e^{-4t}\right)\Big|_0^{\infty}$$
        $$-\pi \left(e^{-4\infty}-1\right)$$

        Thus, we obtain:

        $$\boxed{A=\pi}$$

      \item Per our Fourier transform properties, we may write:

        $$tx(t)\to j\frac{d}{d\omega}X(\omega)$$

        For $y(t)=te^{-2|t|}$ this gives us:

        $$Y(j\omega)=j\frac{d}{d\omega}\left[ \frac{4}{(4+\omega^2)} \right]$$

        Differentiating gives us the final answer as:

        $$\boxed{Y(j\omega)=-\frac{8j\omega}{(4+\omega^2)^2}}$$

      \item By the duality property, we know that if $x(t)\leftrightarrow X(j\omega)$, then:

        $$x(t)\leftrightarrow 2\pi X(-j\omega)$$

        As such, we may write:

        $$-\frac{8jt}{(4+t^2)^2}\leftrightarrow 2\pi(-\omega)e^{-2|\omega|}$$
        $$\frac{t}{(4+t^2)^2}\leftrightarrow -j\pi\omega e^{-2|\omega|}$$

        Thus, we see that:

        $$\boxed{\mathcal{F}\left\{ \frac{4t}{(4+t^2)^2} \right\}=-j\pi\omega e^{-2|\omega|}}$$

    \end{enumerate}

  \item We may compute the response of the system using the convolution; the convolution may be more easily computed using the Fourier transform such that:

    $$y(t)=h(t)*x(t)\leftrightarrow Y(j\omega)=H(j\omega)X(j\omega)$$

    We can see that the given response may be written as:

    $$H(j\omega)=\left\{\begin{array}{ll} e^{-3j\omega},&|\omega|\leq4\\0,&\text{otherwise}\end{array}$$

    \begin{enumerate}

      \item We may see that the transform becomes:

        $$X_1(j\omega)=\pi e^{N\pi j\omega}\left[ \delta(\omega-10)+\delta(\omega+10) \right]$$

        We may observe that the response and signal do not have common values for which they are non-zero. This gives us:

        $$Y_1(j\omega)=0$$

        This ultimately means:

        $$\boxed{y_1(t)=0}$$

      \item We may see that the transform becomes:

        $$X_2(j\omega)=5\pi\left[ \delta(\omega-2)+\delta(\omega+2) \right]$$

        We multiply the two together to get:

        $$Y_2(j\omega)=5\pi e^{-3j\omega}\left[  \delta(\omega-2)+\delta(\omega+2)\right]$$

        We see this introduces a delay of three units, which gives us:

        $$\boxed{y_2(t)=5\cos(2(t-3))}$$

      \item We may see that the transform becomes:

        $$X_3(j\omega)=\left\{\begin{array}{ll} 1,&|\omega|\leq1\\0,&\text{otherwise}\end{array}$$

          Multiplying the two together creates a delayed sink function, but with the boundaries of the input:

        $$Y_3(j\omega)=\left\{\begin{array}{ll} e^{-3j\omega},&|\omega|\leq1\\0,&\text{otherwise}\end{array}$$

          Transforming back, we get:

          $$\boxed{y_3(t)=\frac{\sin(t-3)}{\pi(t-3)}}$$

        \item We may observe that the input signals within the passband of the filter simply introduce a delay at the output, thus, we may conclude:

          $$\boxed{y_4(t)=\left( \frac{\sin(t-3)}{\pi(t-3)} \right)^2}$$

    \end{enumerate}

  \item

    \begin{enumerate}

      \item 

      \item 

    \end{enumerate}

    \setcounter{enumi}{7}

  \item

    \begin{enumerate}

      \item 

      \item 

      \item 

      \item 

    \end{enumerate}

  \item

  \item

\end{enumerate}

\end{document}

