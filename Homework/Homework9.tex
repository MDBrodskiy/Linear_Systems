%%%%%%%%%%%%%%%%%%%%%%%%%%%%%%%%%%%%%%%%%%%%%%%%%%%%%%%%%%%%%%%%%%%%%%%%%%%%%%%%%%%%%%%%%%%%%%%%%%%%%%%%%%%%%%%%%%%%%%%%%%%%%%%%%%%%%%%%%%%%%%%%%%%%%%%%%%%%%%%%%%%
% Written By Michael Brodskiy
% Class: Fundamentals of Linear Systems
% Professor: I. Salama
%%%%%%%%%%%%%%%%%%%%%%%%%%%%%%%%%%%%%%%%%%%%%%%%%%%%%%%%%%%%%%%%%%%%%%%%%%%%%%%%%%%%%%%%%%%%%%%%%%%%%%%%%%%%%%%%%%%%%%%%%%%%%%%%%%%%%%%%%%%%%%%%%%%%%%%%%%%%%%%%%%%

\documentclass[12pt]{article} 
\usepackage{alphalph}
\usepackage[utf8]{inputenc}
\usepackage[russian,english]{babel}
\usepackage{titling}
\usepackage{amsmath}
\usepackage{graphicx}
\usepackage{enumitem}
\usepackage{amssymb}
\usepackage[super]{nth}
\usepackage{everysel}
\usepackage{ragged2e}
\usepackage{geometry}
\usepackage{multicol}
\usepackage{fancyhdr}
\usepackage{cancel}
\usepackage{siunitx}
\usepackage{physics}
\usepackage{tikz}
\usepackage{mathdots}
\usepackage{yhmath}
\usepackage{cancel}
\usepackage{color}
\usepackage{array}
\usepackage{multirow}
\usepackage{gensymb}
\usepackage{tabularx}
\usepackage{extarrows}
\usepackage{booktabs}
\usepackage{lastpage}
\usetikzlibrary{fadings}
\usetikzlibrary{patterns}
\usetikzlibrary{shadows.blur}
\usetikzlibrary{shapes}

\geometry{top=1.0in,bottom=1.0in,left=1.0in,right=1.0in}
\newcommand{\subtitle}[1]{%
  \posttitle{%
    \par\end{center}
    \begin{center}\large#1\end{center}
    \vskip0.5em}%

}
\usepackage{hyperref}
\hypersetup{
colorlinks=true,
linkcolor=blue,
filecolor=magenta,      
urlcolor=blue,
citecolor=blue,
}


\title{Homework 9}
\date{\today}
\author{Michael Brodskiy\\ \small Professor: I. Salama}

\begin{document}

\maketitle

\begin{enumerate}

  \item From the fundamental period, we may get:

    $$\omega_o=\frac{2\pi}{N}$$
    $$\omega_o=\frac{2\pi}{5}$$

    We can express the signal as:

    $$x[n]=\sum_{-\infty}^{\infty} a_ke^{jk\omega_o n}$$

    We know:

    $$a_o=4,\,a_2=2je^{-\frac{j\pi}{4}}=a_{-2}^{*},\,a_4=e^{-\frac{j\pi}{8}}=c_{-4}^{*}$$

    This allows us to write:

    $$x[n]=4+a_2e^{2j\omega_o n}+a_{-2}e^{-2j\omega_on}+a_{4}e^{4j\omega_o n}+a_{-4}e^{-4j\omega_o n}$$

    Plugging in known values gets us:

    $$x[n]=4+2je^{-\frac{j\pi}{4}}e^{2j\omega_o n}-2je^{\frac{j\pi}{4}}e^{-2j\omega_on}+e^{-\frac{j\pi}{8}}e^{4j\omega_o n}-e^{\frac{j\pi}{8}}e^{-4j\omega_o n}$$

    Per our exponential formulas, we may get:

    $$x[n]=4-4\sin\left( 2\omega_on-\frac{\pi}{4} \right)+2\cos\left( 4\omega_on-\frac{\pi}{8} \right)$$

    We then use:

    $$\cos(\theta)=\sin\left( \theta+\frac{\pi}{2} \right)$$

    To get:

    $$x[n]=4-4\sin\left( 2\omega_on-\frac{\pi}{4} \right)+2\sin\left( 4\omega_on+\frac{3\pi}{8} \right)$$

    Inserting the fundamental frequency gets us:

    $$\boxed{x[n]=4-4\sin\left( \frac{4\pi}{5}n-\frac{\pi}{4} \right)+2\sin\left(\frac{8\pi}{5}n+\frac{3\pi}{8} \right)}$$

  \item

    \begin{enumerate}

      \item Using our trigonometric identities, we may expand to get:

        $$\cos\left( \frac{2\pi}{3}n \right)\cos\left( \frac{2\pi}{4}n \right)=\frac{1}{2}\left[ \cos\left( \frac{7\pi}{6}n \right)+\cos\left( \frac{\pi}{6}n \right) \right]$$

        We then chance the sinusoids to exponentials to get:

        $$\frac{1}{2}\left[ \cos\left( \frac{7\pi}{6}n \right)+\cos\left( \frac{\pi}{6}n \right) \right]=\frac{1}{4}\left[ e^{\frac{7j\pi}{6}n}+e^{-\frac{7j\pi}{6}n}+e^{\frac{j\pi}{6}n}+e^{\frac{-j\pi}{6}n} \right]$$

        Thus, we write:

        $$x[n]=\frac{1}{4}e^{\frac{7j\pi}{6}n}+\frac{1}{4}e^{-\frac{7j\pi}{6}n}+\frac{1}{4}e^{\frac{j\pi}{6}n}+\frac{1}{4}e^{\frac{-j\pi}{6}n}$$

        From this, and the fact that the signal is real, we may write the coefficients as:

        $$\boxed{c_n=a_n=\left\{\begin{array}{ll} a_{-1}&= 1/4\\a_{1}&= 1/4\\a_{-7}&=1/4\\a_{7}&= 1/4\end{array}} $$

      \item We can use the following formula:

        $$c_k=\frac{1}{N_o}\sum_{n=0}^{N-1}x[n]e^{-\frac{2jk\pi}{N_o}n}$$

        We may substitute in known values:

        $$c_k=\frac{1}{3}\sum_{n=0}^{2}\left( 1-\sin\left( \frac{\pi}{3}n \right) \right)e^{-\frac{2jk\pi}{3}n}$$

        We evaluate for all values to get:

        $$c_k=\frac{1}{3}\left[1+\left( 1-\frac{\sqrt{3}}{2} \right)e^{-\frac{2jk\pi}{3}}+\left( 1-\frac{\sqrt{3}}{2} \right)e^{-\frac{4jk\pi}{3}}\right]$$

        We can simplify to get:

        $$\boxed{c_k=\frac{1}{3}+\frac{1}{3}\left( 1-\frac{\sqrt{3}}{2} \right)\left[e^{-\frac{2jk\pi}{3}}+e^{-\frac{4jk\pi}{3}}\right]}$$

    \end{enumerate}

  \item

    \begin{enumerate}

      \item To simplify the Fourier transform, we may express $x[n]$ in terms of delta functions, which gives us:

        $$x[n]=\delta[n+2]+\delta[n+1]+\cdots+\delta[n-4]$$

        Using our known transforms, we may write:

        $$\boxed{X_1(e^{j\Omega})=e^{2j\Omega}+e^{j\Omega}+1+e^{-j\Omega}+e^{-2j\Omega}+e^{-3j\Omega}+e^{-4j\Omega}}$$

      \item Per the time-shifting property, we may simply shift each signal by the frequency of the added exponential to get:

        $$\boxed{X_2(e^{j\Omega})=e^{2j\left(\Omega-\frac{\pi}{4}\right)}+e^{j\left(\Omega-\frac{\pi}{4}\right)}+1+e^{-j\left(\Omega-\frac{\pi}{4}\right)}+e^{-2j\left(\Omega-\frac{\pi}{4}\right)}+e^{-3j\left(\Omega-\frac{\pi}{4}\right)}+e^{-4j\left(\Omega-\frac{\pi}{4}\right)}}$$

      \item Combining several of our known Fourier transform properties, we may write:

        $$\boxed{X_3(e^{j\Omega})=\frac{e^{2j\Omega}}{1-\frac{1}{8}e^{j\Omega}}}$$

      \item We can take:

        $$\cos\left( \frac{\pi}{3}n \right)\to \pi\left[ \delta\left( \Omega-\frac{\pi}{3} \right)+\delta\left( \Omega+\frac{\pi}{3} \right) \right]$$

        And:

        $$\left( \frac{1}{2} \right)^nu[n]\to\frac{1}{1-\frac{1}{2}e^{-j\Omega}}$$

        We combine the two to get:

        $$\boxed{X_4(e^{j\Omega}=\frac{\pi\delta\left( \Omega-\frac{\pi}{3} \right)}{1-\frac{1}{2}e^{-j\left( \Omega-\frac{\pi}{3} \right)}}+\frac{\pi\delta\left( \Omega+\frac{\pi}{3} \right)}{1-\frac{1}{2}e^{-j\left( \Omega+\frac{\pi}{3} \right)}}}$$

      \item We take the known transforms for both sinusoids to write:

        $$2\sin\left( \frac{\pi}{3}n \right)\to \frac{2\pi}{j}\left[ \delta\left( \Omega-\frac{\pi}{3} \right)-\delta\left( \Omega+\frac{\pi}{3} \right) \right]$$
        $$4\cos\left( \frac{2\pi}{5}n \right)\to 4\pi\left[ \delta\left( \Omega-\frac{2\pi}{5} \right)+\delta\left( \Omega+\frac{2\pi}{5} \right) \right]$$

        We sum the two to get:

        $$\boxed{X_5(e^{j\Omega})=\frac{2\pi}{j}\left[ \delta\left( \Omega-\frac{\pi}{3} \right)-\delta\left( \Omega+\frac{\pi}{3} \right) \right]+4\pi\left[ \delta\left( \Omega-\frac{2\pi}{5} \right)+\delta\left( \Omega+\frac{2\pi}{5} \right) \right]}$$

    \end{enumerate}

  \item

    \begin{enumerate}

      \item Based on our formulas, we may write:

        $$x[n]=\frac{1}{2\pi}\left[\int_{\frac{\pi}{3}}^{\frac{2\pi}{3}}e^{jn\Omega}\,d\Omega\right]$$

        From here, we simply evaluate:

        $$x[n]=\frac{1}{2jn\pi}\left[e^{jn\Omega}\Big|_{-\frac{2\pi}{3}}^{-\frac{\pi}{3}}+e^{jn\Omega}\Big|_{\frac{\pi}{3}}^{\frac{2\pi}{3}}\right]$$
        $$x[n]=\frac{1}{2jn\pi}\left[e^{-\frac{jn\pi}{3}}-e^{-\frac{2jn\pi}{3}} +e^{-\frac{2jn\pi}{3}}-e^{\frac{jn\pi}{3}} \right]$$

        Using our formulas, we get:

        $$\boxed{x[n]=\frac{1}{n\pi}\left[\sin\left( \frac{2\pi n}{3}\Omega \right)-\sin\left( \frac{\pi n}{3} \Omega\right)\right]}$$

      \item We may observe that this is simply a collection of delta signals:

        $$\boxed{x_2[n]=5\delta[n]+4\delta[n-1]+2\delta[n-3]-5\delta[n-4]+\delta[n-7]}$$

      \item We may rewrite this as:

        $$X_3(e^{j\Omega})=2\left[ 1+\cos(6\Omega) \right]+4\left[ 1-\cos(4\Omega) \right]$$
        $$X_3(e^{j\Omega})=6+2\cos(6\Omega) -4\cos(4\Omega) \right]$$

        We may once again rewrite this as:

        $$X_3(e^{j\Omega})=6+e^{6j\Omega}+e^{-6j\Omega}-2e^{4j\Omega}-2e^{-4j\Omega}$$

        This can easily be converted to delta signals:

        $$\boxed{x_3[n]=6\delta[n]+\delta[n-6]+\delta[n+6]-2\delta[n-4]-2\delta[n+4]}$$

      \item Skipped (4 Needed for Full Credit)

      \item We may rewrite this as:

      $$\frac{1-\frac{1}{3}e^{-j\Omega}}{1-\frac{3}{4}e^{-j\Omega}+\frac{1}{8}e^{-2j\Omega}}=\frac{1-\frac{1}{3}e^{-j\Omega}}{\left( 1-\frac{1}{2}e^{-j\Omega} \right)\left( 1-\frac{1}{4}{e^{-j\Omega}} \right)}$$

        Using partial fraction decomposition, we may write this as:

        $$X_5(e^{j\Omega})=\frac{2/3}{1-\frac{1}{2}e^{-j\Omega}}+\frac{1/3}{1-\frac{1}{4}e^{-j\Omega}}$$

        From here, we can use our transform formulas to obtain:

        $$\boxed{x_5[n]=\frac{2}{3}\left( \frac{1}{2} \right)^{n}u[n]+\frac{1}{3}\left( \frac{1}{4} \right)^{n}u[n]}$$

    \end{enumerate}

  \item

    \begin{enumerate}

      \item Taking the transform, we get:

        $$Y(e^{j\Omega})+\frac{1}{6}e^{-j\Omega}Y(e^{j\Omega})-\frac{1}{6}e^{-2j\Omega}Y(e^{j\Omega})=X(e^{j\Omega})-e^{-j\Omega}X(e^{j\Omega})$$

        We know that the transfer function may be defined as the output divided by the input. This gives us:

        $$\boxed{H(e^{j\Omega})=\frac{1-e^{-j\Omega}}{1+\frac{1}{6}e^{-j\Omega}-\frac{1}{6}e^{-2j\Omega}}}$$

      \item We may factor the denominator in the above to get:

        $$H(e^{j\Omega})=\frac{1-e^{-j\Omega}}{-\frac{1}{6}(e^{-j\Omega}-3)(e^{-j\Omega}+2)}$$

        Using partial fraction decomposition, we get:

        $$H(e^{j\Omega})=\frac{18/5}{e^{-j\Omega}+2}-\frac{12/5}{e^{-j\Omega}-3}$$

        We can rearrange this to a form that is easier to convert:

        $$H(e^{j\Omega})=\frac{18/5}{1+\frac{1}{2}e^{-j\Omega}}-\frac{12/5}{1-\frac{1}{3}e^{-j\Omega}}$$

        This allows to take the inverse transform to obtain:

        $$\boxed{h[n]=\frac{18}{5}\left( -\frac{1}{2} \right)^nu[n]-\frac{12}{5}\left( \frac{1}{3} \right)^nu[n]}$$

      \item We can multiply the response in the frequency domain by the input in the frequency domain:

        $$X_1(e^{j\Omega})=\frac{1}{1-\frac{1}{4}e^{-j\Omega}}$$

        This gives us:

        $$y_1(e^{j\Omega})=\frac{18/5}{\left( 1+\frac{1}{2}e^{-j\Omega} \right)\left(1-\frac{1}{4}e^{-j\Omega}\right)}-\frac{12/5}{\left( 1-\frac{1}{3}e^{-j\Omega} \right)\left(1-\frac{1}{4}e^{-j\Omega}\right)}$$

        We decompose to get:

        $$y_1(e^{j\Omega})=\frac{12/5}{1+\frac{1}{2}e^{-j\Omega}}+\frac{6/5}{1-\frac{1}{4}e^{-j\Omega}}-\frac{48/5}{1-\frac{1}{3}e^{-j\Omega}}+\frac{36/5}{1-\frac{1}{4}e^{-j\Omega}}$$
        $$y_1(e^{j\Omega})=\frac{12/5}{1+\frac{1}{2}e^{-j\Omega}}+\frac{42/5}{1-\frac{1}{4}e^{-j\Omega}}-\frac{48/5}{1-\frac{1}{3}e^{-j\Omega}}$$

        And then we take the inverse:

        $$\boxed{y_1[n]=\left[\frac{12}{5}\left( -\frac{1}{2} \right)^n+\frac{42}{5}\left( \frac{1}{4} \right)^n-\frac{48}{5}\left( \frac{1}{3} \right)^n\right]u[n]}$$

      \item We use the time-shifting property to write:

        $$h_2(e^{j\Omega})=e^{-\frac{j \pi}{2}\Omega}\left[\frac{18/5}{1+\frac{1}{2}e^{-j\Omega}}-\frac{12/5}{1-\frac{1}{3}e^{-j\Omega}}\right]$$

        This gives us:

        $$\boxed{y_2[n]=\frac{18}{5}\left( -\frac{1}{2} \right)^{n-\frac{\pi}{2}}u\left[ n-\frac{\pi}{2} \right]-\frac{12}{5}\left( \frac{1}{3} \right)^{n-\frac{\pi}{2}}u\left[ n-\frac{\pi}{2} \right]}$$

      \item We know that this sinusoid may be written as:

        $$2\cos\left( \frac{\pi}{2}n \right)\to e^{-\frac{j\pi n}{2}}+e^{\frac{j\pi n}{2}}$$

        Thus, we may write this input as:

        $$x_3[n]=\frac{1}{2}x_2[n]+\frac{e^{-j\pi n}}{2}x_2[n]$$

        This gives us:

        $$\boxed{y_3[n]=\left[\frac{9}{5}\left( -\frac{1}{2} \right)^{n-\frac{\pi}{2}}-\frac{6}{5}\left( \frac{1}{3} \right)^{n-\frac{\pi}{2}}\right]u\left[ n-\frac{\pi}{2} \right]+}$$
        $$\boxed{\left[\frac{9}{5}\left( -\frac{1}{2} \right)^{n-\frac{3\pi}{2}}-\frac{6}{5}\left( \frac{1}{3} \right)^{n-\frac{3\pi}{2}}\right]u\left[ n-\frac{3\pi}{2} \right]}$$

    \end{enumerate}

  \item

    \begin{enumerate}

      \item First, we transform the response to get:

        $$X(e^{j\Omega})=\frac{1}{1-\frac{1}{2}e^{-j\Omega}}$$

        The response can then be written as:

        $$Y(e^{j\Omega})=j\frac{d}{d\Omega}\left[ \frac{1}{1-\frac{1}{3}e^{-j\Omega}} \right]$$
        $$Y(e^{j\Omega})= \frac{\frac{1}{3}e^{-j\Omega}}{\left[1-\frac{1}{3}e^{-j\Omega}\right]^2} \right]$$

        We divide the response by the input to get:

        $$\boxed{H(e^{j\Omega})=\frac{\frac{1}{3}e^{-j\Omega}-\frac{1}{6}e^{-2j\Omega}}{\left[1-\frac{1}{3}e^{-j\Omega}\right]^2}}$$

      \item We expand the response from above to get:

        $$\frac{Y(e^{j\Omega})}{X(e^{j\Omega})}=\frac{\frac{1}{3}e^{-j\Omega}-\frac{1}{6}e^{-2j\Omega}}{1-\frac{2}{3}e^{-j\Omega}+\frac{1}{9}e^{-2j\Omega}}$$

        From this, we get:

        $$\boxed{y[n]-\frac{2}{3}y[n-1]+\frac{1}{9}y[n-2]=\frac{1}{3}x[n-1]-\frac{1}{6}x[n-2]}$$

    \end{enumerate}

  \item

    \begin{enumerate}

      \item The overall response can be found by multiplying the two individual responses:

        $$H(e^{j\Omega})=H_1(e^{j\Omega})H_2(e^{j\Omega})$$
        $$H(e^{j\Omega})=\left( \frac{1-\frac{1}{2}e^{-j\Omega}}{1+\frac{1}{3}e^{-j\Omega}} \right)\left( \frac{1}{\left[ 1-\frac{1}{3}e^{-j\Omega} \right]^2}  \right)$$

        This gives us:

        $$\boxed{H(e^{j\Omega})=\frac{1-\frac{1}{2}e^{-j\Omega}}{\left( 1+\frac{1}{3}e^{-j\Omega} \right)\left( 1-\frac{1}{3}e^{-j\Omega} \right)^2}}$$

      \item 
        
        We can multiply out the denominator to get:

        $$H(e^{j\Omega})=\frac{1-\frac{1}{2}e^{-j\Omega}}{\left( 1-\frac{1}{9}e^{-2j\Omega} \right)\left( 1-\frac{1}{3}e^{-j\Omega} \right)}$$
        $$H(e^{j\Omega})=\frac{1-\frac{1}{2}e^{-j\Omega}}{1-\frac{1}{3}e^{-j\Omega} -\frac{1}{9}e^{-2j\Omega}+\frac{1}{27}e^{-3j\Omega}}$$

        This gives us the difference equation as:

        $$\boxed{y[n]-\frac{1}{3}y[n-1]-\frac{1}{9}y[n-2]+\frac{1}{27}y[n-3]=x[n]-\frac{1}{2}x[n-1]}$$

      \item We can take the inverse of our result from (a) by first breaking it into partial fractions:

        $$H(e^{j\Omega})=\frac{15/8}{3+e^{-j\Omega}}+\frac{15/8}{3-e^{-j\Omega}}-\frac{9/4}{(-3+e^{-j\Omega})^2}$$
        $$H(e^{j\Omega})=\frac{5/8}{1+\frac{1}{3}e^{-j\Omega}}+\frac{5/8}{1-\frac{1}{3}e^{-j\Omega}}-\frac{1/4}{(1-\frac{1}{3}e^{-j\Omega})^2}$$

        Taking the inverse, we get:

        $$\boxed{h[n]=\left[\frac{5}{8}\left( -\frac{1}{3} \right)^{n}+\frac{5}{8}\left( \frac{1}{3} \right)^{n}-\frac{n}{4}\left( \frac{1}{3} \right)^n\right]u[n]}$$

      \item Taking the inverse of the first system, we get:

        $$H_1^{inv}(e^{j\Omega})=\frac{1+\frac{1}{3}e^{-j\Omega}}{1-\frac{1}{2}e^{-j\Omega}}$$

        This can be separated into:

        $$\boxed{H_1^{inv}(e^{j\Omega})=\frac{1}{1-\frac{1}{2}e^{-j\Omega}}-\frac{\frac{1}{3}e^{-j\Omega}}{1-\frac{1}{2}e^{-j\Omega}}}$$

        This gives us:

        $$\boxed{h_1^{inv}[n]=\left( \frac{1}{2} \right)^nu[n]-\frac{1}{3}\left( \frac{1}{2} \right)^{n-1}u[n-1]}$$

    \end{enumerate}

\end{enumerate}

\end{document}

