%%%%%%%%%%%%%%%%%%%%%%%%%%%%%%%%%%%%%%%%%%%%%%%%%%%%%%%%%%%%%%%%%%%%%%%%%%%%%%%%%%%%%%%%%%%%%%%%%%%%%%%%%%%%%%%%%%%%%%%%%%%%%%%%%%%%%%%%%%%%%%%%%%%%%%%%%%%%%%%%%%%
% Written By Michael Brodskiy
% Class: Fundamentals of Linear Systems
% Professor: I. Salama
%%%%%%%%%%%%%%%%%%%%%%%%%%%%%%%%%%%%%%%%%%%%%%%%%%%%%%%%%%%%%%%%%%%%%%%%%%%%%%%%%%%%%%%%%%%%%%%%%%%%%%%%%%%%%%%%%%%%%%%%%%%%%%%%%%%%%%%%%%%%%%%%%%%%%%%%%%%%%%%%%%%

\include{Includes.tex}

\title{Lecture 2 — Introduction to Signals and Systems}
\date{\today}
\author{Michael Brodskiy\\ \small Professor: I. Salama}

\begin{document}

\maketitle

\begin{itemize}

  \item Signal Power and Energy

    \begin{itemize}

      \item Definition

        \begin{itemize}

          \item Consider signal $x(t)$ representing the voltage or current in a unit resistance. The signal power is defined as $p(t)=|x(t)|^2$

          \item It is a common terminology to refer to $|x(t)|^2$ or $|x[n]|^2$ as the signal power even if the signal does not represent voltage or current

        \end{itemize}

      \item Total energy in a finite duration interval

        \begin{itemize}

          \item The total energy in an interval $T=t_2-t_1$ is given by:

            $$\text{Continuous Time } \to E=\int_{t_1}^{t_2} \underbrace{|x(t)|^2}_{p(t)}\,dt$$
            $$\text{Discrete Time } \to E=\Delta T\sum_{n=n_1}^{n_2}\underbrace{|x[n]|^2}_{p(t)}\,\text{ where }T=(n_2-n_1+1)\Delta T$$

        \end{itemize}

      \item The average power in a finite duration interval

        $$P_{avg}=\frac{E}{t_2-t_1}=\frac{1}{t_1-t_2}\int_{t_1}^{t_2}|x(t)|^2\,dt$$
        \begin{center}
          or
        \end{center}
        $$P_{avg}=\frac{1}{n_2-n_1+1}\sum_{n=n_1}^{n_2} |x[n]|^2$$

    \end{itemize}

  \item Power and Energy over an infinite time interval

    \begin{itemize}

      \item Energy

        $$\text{Continuous Time } \to E_{\infty}=\lim_{T\to\infty}\int_{-T}^{T} \underbrace{|x(t)|^2}_{p(t)}\,dt$$
        $$\text{Discrete Time } \to E_{\infty}=\lim_{N\to\infty}\cancel{\Delta T}\sum_{n=-N}^{N}\underbrace{|x[n]|^2}_{p(t)}$$

      \item Power

        $$P_{\infty}=\lim_{T\to\infty}\frac{E_{\infty}}{2T}=\lim_{T\to\infty}\frac{1}{2T}\int_{-T}^{T} \underbrace{|x(t)|^2}_{p(t)}\,dt$$
        \begin{center}
          or
        \end{center}
        $$P_{\infty}=\lim_{N\to\infty}\frac{E_{\infty}}{2N+1}=\lim_{T\to\infty}\frac{1}{2N+1}\sum_{n=-N}^{N}\underbrace{|x[n]|^2}_{p(t)}$$

    \end{itemize}

  \item Energy Signals versus Power Signals

    \begin{itemize}

      \item The energy or power of a signal quantifies the magnitude of the signal. For this measure to be meaningful, it must be finite. This requirement leads to the following classification of signals:

        \begin{itemize}

          \item Energy

            \begin{itemize}

              \item Signals with finite total energy ($E_{\infty}<\infty$)

              \item They have zero average power

                $$P_{\infty}=\lim_{T\to\infty}\frac{E_{\infty}}{2T}=0$$
                $$P_{\infty}=\lim_{N\to\infty}\frac{E_{\infty}}{2N+1}=0$$

            \end{itemize}

          \item Power

            \begin{itemize}

              \item Signals with finite average power ($P_{\infty} < \infty$)

              \item They have infinite energy

                $$E_{\infty}=\lim_{T\to\infty}2T(P_{\infty})\to\infty$$
                $$E_{\infty}=\lim_{N\to\infty}(2N+1)(P_{\infty})\to\infty$$

            \end{itemize}

          \item Any finite signal is automatically an energy signal (think: some value in range, 0 otherwise)

        \end{itemize}

    \end{itemize}

  \item Periodic Signals

    \begin{itemize}

      \item Periodic signals are classified as power signals because they possess an infinite amount of energy

      \item The average power of a periodic signal can be determined by averaging its power over one period:

        $$P_{\infty}=P_{avg}=\frac{1}{T_o}\int_{-T_o/2}^{T_o/2}|x(t)|^2\,dt$$

    \end{itemize}

  \item Signals with neither finite power nor energy

    \begin{itemize}

      \item Some signals have neither finite power nor energy

      \item An example is a ramp signal, where $x(t)=t,\,t\geq 0$

      \item Neither the energy nor the power can be defined for such signals

    \end{itemize}

  \item Transformation of the independent variable

    \begin{itemize}

      \item In this section, we will explore key elementary signal transformations that involve basic modifications of the independent variable for both discrete and continuous-time signals. These transformations include:

        \begin{itemize}

          \item Time shifting

          \item Time scaling

          \item Time reversal

          \item Combined operations

        \end{itemize}

    \end{itemize}

  \item Time Shifting

    \begin{itemize}

      \item Given a signal $x(t)$, a shift could be written as $y(t)=x(t-t_o)$. This would mean that:

        $$x(t-t_o)=x(t_{old})\text{ at } t=t_{old}+t_o$$

      \item For shift $x(t-t_o)$ the signal is shifted to the right, and for shift $x(t+t_o)$ the signal is shifted to the left

      \item For discrete time, a shift has the same effect

    \end{itemize}

  \item Time Reversal

    \begin{itemize}

      \item Time reversal is performed using a $180^{\circ}$ rotation around the vertical axis. If $x(t)$ represents an audio recording, $x(-t)$ is the audio recording played backward

      \item For example $x(t)=t$, a linear line with slope one, would become $x(-t)=-t$ upon reversal, with slope negative one (reflected about vertical axis)

    \end{itemize}

  \item Time Scaling

    \begin{itemize}

      \item Continuous Time Signals

        \begin{itemize}

          \item $x(\alpha t)$ leads to linear compression if $\alpha>1$ and linear stretching if $0\leq \alpha\leq 1$

          \item If $x(t)$ is an audio recording, the expression $x(2t)$ represents the recording played at twice the speed, and $x\left( \frac{1}{2}t \right)$ is the recording played at half the speed

          \item In the time scaling operation, $t=0$ serves as a fixed anchor point and remains unchanged, since $x(t)=x(\alpha t)$ at $t=0$

          \item The concepts of compression and expansions differ slightly for discrete time signals

        \end{itemize}

      \item Discrete Time Signals

        \begin{itemize}

          \item Zero remains as an anchor point

          \item Take only equivalent values at integer $n$ values, and assume $y[n]=0$ for non-integer $n$ values

          \item For $x\left[ \frac{n}{L} \right]$, which increases the sampling rate, (resample) the sequence by a factor of $L$, a process known as up-sampling

          \item $L-1$ zeros are inserted between each consecutive data points

          \item Usually followed by a low pass filter to interpolate

          \item Compression is much simpler, as a compression would lose information, but will only use integer $n$ values

          \item First, a low pass filter is used, and the values are down-sampled

          \item For $x[Mn]$, the signal is decimated by a factor of $M$, which keeps only very $M$-th sample

        \end{itemize}

    \end{itemize}

  \item Combined Operations

    \begin{itemize}

      \item For $y(t)=x(\alpha t + \beta)=x(t_{old}$ we can write:

          $$t_{old}=\alpha t + \beta$$
          $$t=\frac{1}{\alpha}\left( t_{old}-\beta \right)$$

          \begin{itemize}

            \item This means the signal was scaled by $1/\alpha$ and shifted to the left $\beta/\alpha$

          \end{itemize}

    \end{itemize}

  \item Euler Formula

    \begin{itemize}

      \item $Ae^{j(\omega_o t+\phi)}=A\left\{ \cos(\omega_o t+\phi)+j\sin(\omega_o t+\phi) \right\}$

      \item $Ae^{-j(\omega_o t+\phi)}=A\left\{ \cos(\omega_o t+\phi)-j\sin(\omega_o t+\phi) \right\}$

      \item $A\cos(\omega_o t+\phi)=\dfrac{A}{2}\left\{ e^{j\omega_o t+\phi)}+e^{-j(\omega_o t+\phi)} \right\}=\text{Re}\left\{ Ae^{j(\omega_o t+\phi)} \right\}$

      \item $A\sin(\omega_o t+\phi)=\dfrac{A}{2j}\left\{ e^{j\omega_o t+\phi)}-e^{-j(\omega_o t+\phi)} \right\}=\text{Im}\left\{ Ae^{j(\omega_o t+\phi)} \right\}$

    \end{itemize}

\end{itemize}

\end{document}

