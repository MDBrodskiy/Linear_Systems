%%%%%%%%%%%%%%%%%%%%%%%%%%%%%%%%%%%%%%%%%%%%%%%%%%%%%%%%%%%%%%%%%%%%%%%%%%%%%%%%%%%%%%%%%%%%%%%%%%%%%%%%%%%%%%%%%%%%%%%%%%%%%%%%%%%%%%%%%%%%%%%%%%%%%%%%%%%%%%%%%%%
% Written By Michael Brodskiy
% Class: Fundamentals of Linear Systems
% Professor: I. Salama
%%%%%%%%%%%%%%%%%%%%%%%%%%%%%%%%%%%%%%%%%%%%%%%%%%%%%%%%%%%%%%%%%%%%%%%%%%%%%%%%%%%%%%%%%%%%%%%%%%%%%%%%%%%%%%%%%%%%%%%%%%%%%%%%%%%%%%%%%%%%%%%%%%%%%%%%%%%%%%%%%%%

\documentclass[12pt]{article} 
\usepackage{alphalph}
\usepackage[utf8]{inputenc}
\usepackage[russian,english]{babel}
\usepackage{titling}
\usepackage{amsmath}
\usepackage{graphicx}
\usepackage{enumitem}
\usepackage{amssymb}
\usepackage[super]{nth}
\usepackage{everysel}
\usepackage{ragged2e}
\usepackage{geometry}
\usepackage{multicol}
\usepackage{fancyhdr}
\usepackage{cancel}
\usepackage{siunitx}
\usepackage{physics}
\usepackage{tikz}
\usepackage{mathdots}
\usepackage{yhmath}
\usepackage{cancel}
\usepackage{color}
\usepackage{array}
\usepackage{multirow}
\usepackage{gensymb}
\usepackage{tabularx}
\usepackage{extarrows}
\usepackage{booktabs}
\usepackage{lastpage}
\usetikzlibrary{fadings}
\usetikzlibrary{patterns}
\usetikzlibrary{shadows.blur}
\usetikzlibrary{shapes}

\geometry{top=1.0in,bottom=1.0in,left=1.0in,right=1.0in}
\newcommand{\subtitle}[1]{%
  \posttitle{%
    \par\end{center}
    \begin{center}\large#1\end{center}
    \vskip0.5em}%

}
\usepackage{hyperref}
\hypersetup{
colorlinks=true,
linkcolor=blue,
filecolor=magenta,      
urlcolor=blue,
citecolor=blue,
}


\title{Lecture 2 — Introduction to Signals and Systems}
\date{\today}
\author{Michael Brodskiy\\ \small Professor: I. Salama}

\begin{document}

\maketitle

\begin{itemize}

  \item Signal Power and Energy

    \begin{itemize}

      \item Definition

        \begin{itemize}

          \item Consider signal $x(t)$ representing the voltage or current in a unit resistance. The signal power is defined as $p(t)=|x(t)|^2$

          \item It is a common terminology to refer to $|x(t)|^2$ or $|x[n]|^2$ as the signal power even if the signal does not represent voltage or current

        \end{itemize}

      \item Total energy in a finite duration interval

        \begin{itemize}

          \item The total energy in an interval $T=t_2-t_1$ is given by:

            $$\text{Continuous Time } \to E=\int_{t_1}^{t_2} \underbrace{|x(t)|^2}_{p(t)}\,dt$$
            $$\text{Discrete Time } \to E=\Delta T\sum_{n=n_1}^{n_2}\underbrace{|x[n]|^2}_{p(t)}\,\text{ where }T=(n_2-n_1+1)\Delta T$$

        \end{itemize}

      \item The average power in a finite duration interval

        $$P_{avg}=\frac{E}{t_2-t_1}=\frac{1}{t_1-t_2}\int_{t_1}^{t_2}|x(t)|^2\,dt$$
        \begin{center}
          or
        \end{center}
        $$P_{avg}=\frac{1}{n_2-n_1+1}\sum_{n=n_1}^{n_2} |x[n]|^2$$

    \end{itemize}

  \item Power and Energy over an infinite time interval

    \begin{itemize}

      \item Energy

        $$\text{Continuous Time } \to E_{\infty}=\lim_{T\to\infty}\int_{-T}^{T} \underbrace{|x(t)|^2}_{p(t)}\,dt$$
        $$\text{Discrete Time } \to E_{\infty}=\lim_{N\to\infty}\cancel{\Delta T}\sum_{n=-N}^{N}\underbrace{|x[n]|^2}_{p(t)}$$

      \item Power

        $$P_{\infty}=\lim_{T\to\infty}\frac{E_{\infty}}{2T}=\lim_{T\to\infty}\frac{1}{2T}\int_{-T}^{T} \underbrace{|x(t)|^2}_{p(t)}\,dt$$
        \begin{center}
          or
        \end{center}
        $$P_{\infty}=\lim_{N\to\infty}\frac{E_{\infty}}{2N+1}=\lim_{T\to\infty}\frac{1}{2N+1}\sum_{n=-N}^{N}\underbrace{|x[n]|^2}_{p(t)}$$

    \end{itemize}

  \item Energy Signals versus Power Signals

    \begin{itemize}

      \item The energy or power of a signal quantifies the magnitude of the signal. For this measure to be meaningful, it must be finite. This requirement leads to the following classification of signals:

        \begin{itemize}

          \item Energy

            \begin{itemize}

              \item Signals with finite total energy ($E_{\infty}<\infty$)

              \item They have zero average power

                $$P_{\infty}=\lim_{T\to\infty}\frac{E_{\infty}}{2T}=0$$
                $$P_{\infty}=\lim_{N\to\infty}\frac{E_{\infty}}{2N+1}=0$$

            \end{itemize}

          \item Power

            \begin{itemize}

              \item Signals with finite average power ($P_{\infty} < \infty$)

              \item They have infinite energy

                $$E_{\infty}=\lim_{T\to\infty}2T(P_{\infty})\to\infty$$
                $$E_{\infty}=\lim_{N\to\infty}(2N+1)(P_{\infty})\to\infty$$

            \end{itemize}

          \item Any finite signal is automatically an energy signal (think: some value in range, 0 otherwise)

        \end{itemize}

    \end{itemize}

  \item Periodic Signals

    \begin{itemize}

      \item Periodic signals are classified as power signals because they possess an infinite amount of energy

      \item The average power of a periodic signal can be determined by averaging its power over one period:

        $$P_{\infty}=P_{avg}=\frac{1}{T_o}\int_{-T_o/2}^{T_o/2}|x(t)|^2\,dt$$

    \end{itemize}

  \item Signals with neither finite power nor energy

    \begin{itemize}

      \item Some signals have neither finite power nor energy

      \item An example is a ramp signal, where $x(t)=t,\,t\geq 0$

      \item Neither the energy nor the power can be defined for such signals

    \end{itemize}

  \item Transformation of the independent variable

    \begin{itemize}

      \item In this section, we will explore key elementary signal transformations that involve basic modifications of the independent variable for both discrete and continuous-time signals. These transformations include:

        \begin{itemize}

          \item Time shift

        \end{itemize}

    \end{itemize}

\end{itemize}

\end{document}

