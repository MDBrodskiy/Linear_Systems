%%%%%%%%%%%%%%%%%%%%%%%%%%%%%%%%%%%%%%%%%%%%%%%%%%%%%%%%%%%%%%%%%%%%%%%%%%%%%%%%%%%%%%%%%%%%%%%%%%%%%%%%%%%%%%%%%%%%%%%%%%%%%%%%%%%%%%%%%%%%%%%%%%%%%%%%%%%%%%%%%%%
% Written By Michael Brodskiy
% Class: Fundamentals of Linear Systems
% Professor: I. Salama
%%%%%%%%%%%%%%%%%%%%%%%%%%%%%%%%%%%%%%%%%%%%%%%%%%%%%%%%%%%%%%%%%%%%%%%%%%%%%%%%%%%%%%%%%%%%%%%%%%%%%%%%%%%%%%%%%%%%%%%%%%%%%%%%%%%%%%%%%%%%%%%%%%%%%%%%%%%%%%%%%%%

\documentclass[12pt]{article} 
\usepackage{alphalph}
\usepackage[utf8]{inputenc}
\usepackage[russian,english]{babel}
\usepackage{titling}
\usepackage{amsmath}
\usepackage{graphicx}
\usepackage{enumitem}
\usepackage{amssymb}
\usepackage[super]{nth}
\usepackage{everysel}
\usepackage{ragged2e}
\usepackage{geometry}
\usepackage{multicol}
\usepackage{fancyhdr}
\usepackage{cancel}
\usepackage{siunitx}
\usepackage{physics}
\usepackage{tikz}
\usepackage{mathdots}
\usepackage{yhmath}
\usepackage{cancel}
\usepackage{color}
\usepackage{array}
\usepackage{multirow}
\usepackage{gensymb}
\usepackage{tabularx}
\usepackage{extarrows}
\usepackage{booktabs}
\usepackage{lastpage}
\usetikzlibrary{fadings}
\usetikzlibrary{patterns}
\usetikzlibrary{shadows.blur}
\usetikzlibrary{shapes}

\geometry{top=1.0in,bottom=1.0in,left=1.0in,right=1.0in}
\newcommand{\subtitle}[1]{%
  \posttitle{%
    \par\end{center}
    \begin{center}\large#1\end{center}
    \vskip0.5em}%

}
\usepackage{hyperref}
\hypersetup{
colorlinks=true,
linkcolor=blue,
filecolor=magenta,      
urlcolor=blue,
citecolor=blue,
}


\title{Lecture 3 — The Unit Step and Unit Impulse Functions}
\date{\today}
\author{Michael Brodskiy\\ \small Professor: I. Salama}

\begin{document}

\maketitle

\begin{itemize}

  \item The Unit Step Function

    \begin{itemize}

      \item The step function is discontinuous at $t=0$

        $$u(t)=\left\{\begin{array}{l l}0,&t<0\\1,&t\geq0\end{array}$$

    \end{itemize}

  \item The Dirac Delta (Unit Impulse) Function

    \begin{itemize}

      \item As $\epsilon$ reduces to zero, the derivative to the unit step function reduces to a Dirac delta function

        $$\frac{d}{dt}[u(t)]=\lim_{\epsilon\to0}\left\{\begin{array}{ll}\dfrac{1}{\epsilon},&-\dfrac{\epsilon}{2}\leq t<\dfrac{\epsilon}{2}\\0,&\text{otherwise}\end{array}$$

        \item The Dirac delta function has an infinite amplitude and unit area, and a scaled impulse $k\delta(t)$ has an area equal to $k$:

          $$u(t)=\int_{-\infty}^t\delta(\tau)\,d\tau$$

        \item Substitute $x=\tau-t\to u(t)$:

          $$u(t)=\int_{-\infty}^t\delta(\tau)\,d\tau=\int_0^{\infty}\delta(t-x)\,dx$$

        \item Useful Properties:

          $$\int_{-\infty}^{\infty}\delta(t-t_o)\,dt=1$$

          $$\int_{-\infty}^{\infty}A\delta(t-t_o)\,dt=A$$

          $$g(t)\delta(t-t_o)=g(t_o)\delta(t-t_o)$$

          $$\int_{-\infty}^{\infty}g(t)\delta(t-t_o)\,dt=g(t_o)\text{ (sifting property)}$$

          \begin{itemize}

            \item The derivative property ($x(t)$ is continuous and $t_1<t_o<t_2$):

              $$\int_{t_1}^{t_2}x(t)\dot{\delta}(t-t_o)\,dt=-\dot{x}(t_o)$$
              $$\int_{t_1}^{t_2}x(t)\delta^n(t-t_o)\,dt=(-1)^nx^n(t_o)$$

              \begin{itemize}

                \item Where $\delta^n(t-t_o)$ is the n-th derivative of $\delta(t-t_o)$ and $x^n$ is the n-th derivative of $x(t)$

              \end{itemize}

          \end{itemize}

    \end{itemize}

  \item The Unit Ramp Function
    
    $$r(t)=\int_{-\infty}^tu(\tau)\,d\tau=\left\{\begin{array}{ll}t,&t\geq0\\0,&\text{otherwise}\end{array}$$

\end{itemize}

\end{document}

