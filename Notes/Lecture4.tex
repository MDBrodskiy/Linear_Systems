%%%%%%%%%%%%%%%%%%%%%%%%%%%%%%%%%%%%%%%%%%%%%%%%%%%%%%%%%%%%%%%%%%%%%%%%%%%%%%%%%%%%%%%%%%%%%%%%%%%%%%%%%%%%%%%%%%%%%%%%%%%%%%%%%%%%%%%%%%%%%%%%%%%%%%%%%%%%%%%%%%%
% Written By Michael Brodskiy
% Class: Fundamentals of Linear Systems
% Professor: I. Salama
%%%%%%%%%%%%%%%%%%%%%%%%%%%%%%%%%%%%%%%%%%%%%%%%%%%%%%%%%%%%%%%%%%%%%%%%%%%%%%%%%%%%%%%%%%%%%%%%%%%%%%%%%%%%%%%%%%%%%%%%%%%%%%%%%%%%%%%%%%%%%%%%%%%%%%%%%%%%%%%%%%%

\include{Includes.tex}

\title{Lecture 4 — Classifications/Interconnections of Systems}
\date{\today}
\author{Michael Brodskiy\\ \small Professor: I. Salama}

\begin{document}

\maketitle

\begin{itemize}

  \item System Representation

    \begin{itemize}

      \item A system takes a signal as an input and transforms it into an output

      \item This is written as $x(t)$ passed through transformation function $T\left\{ \cdots \right\}$ makes $y(t)$

    \end{itemize}

  \item Linear Systems and the Principle of Superposition

    \begin{itemize}

      \item A homogenous system has zero output for zero input (if $x(t)$ transforms to $y(t)$, then $ax(t)\to ay(t)$)

      \item Additive: $x_1(t)$ causes response $y_1(t)$ and $x_2(t)$ causes response $y_2(t)$, then $x_1(t)+x_2(t)$ causes $y_1(t)+y_2(t)$

      \item A linear system is both homogenous and additive (the superposition principle applies)

    \end{itemize}

  \item Linearity

    \begin{itemize}

      \item The system with an input-output relationship $y(t)=t^2x(t)$ is linear

      \item We can prove linearity by saying:

        $$x_1(t)\to y_1(t)=T\left\{ x_1(t) \right\}=t^2x_1(t)\quad\text{ and }\quad x_2(t)\to y_2(t)=T\left\{ x_2(t) \right\}=t^2x_2(t)$$

      \item and then proving:

        $$T\left\{ a_1x_1(t)+a_2x_2(t) \right\}=t^2(a_1x_1(t)+a_2x_2(t))=a_1t_1(t)+a_2y_2(t)$$


      \item The system with an input-output relationship $y(t)=x^2(t)$ is non-linear

    \end{itemize}

  \item Incrementally Linear Systems

    \begin{itemize}

      \item The system described by $y[n]=2x[n]+4$ is non-linear

      \item The system has a non-zero output for a zero input

      \item The system is sometimes described as incrementally linear, meaning that the difference between two output responses is a linear function of the difference between their corresponding inputs

        $$y_2[n]-y_1[n]=2(x_2[n]-x_2[n])$$

      \item A system described by a linear constant coefficient differential or difference equation is incrementally linear

        $$\frac{d}{dt}v_c(t)+\frac{1}{RC}v_c(t)=\frac{1}{RC}v_s(t)$$

      \item The system's response can be split into two parts: the zero-input response, due to the initial conditions, and the zero-state response

      \item In general, a system described by a linear constant coefficient differential or difference equation is incrementally linear; it has a response equal to the sum of a zero-input response and the zero-state response

    \end{itemize}

  \item Time Invariance

    \begin{itemize}

      \item A system is time invariant if the behavior/parameters of the system are fixed over time; the nature of the response is not expected to change if the experiment is performed now or a few days later

      \item A network initially at rest and composed of RLC components and other commonly used active components with fixed parameters is time invariant

      \item A system described by a linear constant coefficient differential/difference equation is time invariant if initially at rest. If not initially at rest or if the coefficients of the differential/difference become time-dependent, the system becomes time varying

    \end{itemize}

  \item Dynamic Versus Instantaneous System

    \begin{itemize}

      \item Instantaneous or Memoryless System

        \begin{itemize}

          \item The output of the system at any instants depends only on the current instant of the input; the history of the input or system response do not affect the current output

          \item Resistive networks are memoryless

        \end{itemize}

      \item Dynamic Systems or Systems with Memory

        \begin{itemize}

          \item A system is said to be dynamic if the response is determined by the input signal over the past interval of time and/or system response over the past interval of time

        \end{itemize}

    \end{itemize}

  \item Causal Versus Non-Causal Systems

    \begin{itemize}

      \item Causal Systems

        \begin{itemize}

          \item The output of a causal system at any instant depends on the current and previous values of the input; a causal system is non-anticipative

          \item All memoryless systems are causal

        \end{itemize}

      \item Non-Causal Systems

        \begin{itemize}

          \item A system is said to be non-causal if the response depends on future values of the input

        \end{itemize}

    \end{itemize}

  \item Stability

    \begin{itemize}

      \item The input is bounded if there exists a positive finite value $B_{x}$ such that:

        $$|x(t)|<B_x<\infty\,\quad\forall t$$
        $$|x[n]|<B_x<\infty\,\quad\forall n$$

      \item A system is stable if for every bounded input, the output is also bounded; \textit{i}.\textit{e}. there exists a positive finite value, $B_y$ such that:

        $$|y(t)|<B_y<\infty\,\quad\forall t$$
        $$|y[n]|<B_y<\infty\,\quad\forall n$$

      \item This type of stability is referred to as bounded input/bounded output (BIBO) stability

      \item Stability of physical systems usually results from a mechanism that dissipates energy, such as a resistor or losst transmission line in an electrical system, and friction in a mechanical system

      \item Consider a constant applied force, $f(t)=F$, applied to a vehicle initially at rest; the vehicle will continue to accelerate until the frictional force, which is proportional to the velocity, balances the applied force ($\rho v=F$)

      \item The velocity of the vehicle is therefore bounded, and the maximum velocity is given by $v=F/\rho$

    \end{itemize}

  \item Invertibility and Inverse Systems

    \begin{itemize}

      \item Consider a system $S$ which processes an input signal, $x(t)/x[n]$ to produce an output $y(t)/y[n]$; if an inverse system $S_i$ exists tha can reproduce the original input signal $x(t)/x[n]$ using the output, $y(t)/y[n]$, the system is said to be invertible

      \item Cascading a system with its inverse system results in an identity system, producing an output identical to the input

      \item Data transmitted over a communication channel can be distorted due to the channel's non-ideal frequency response; an inverse system for the channel, known as an equalizer, can be used to compensate for this distortion

      \item To demonstrate that a system is invertible, one must find the inverse system; conversely, to show that a system is non-invertible, it's necessary to identify two different inputs that result in the same output

    \end{itemize}

\end{itemize}

\end{document}

