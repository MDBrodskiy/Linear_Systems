%%%%%%%%%%%%%%%%%%%%%%%%%%%%%%%%%%%%%%%%%%%%%%%%%%%%%%%%%%%%%%%%%%%%%%%%%%%%%%%%%%%%%%%%%%%%%%%%%%%%%%%%%%%%%%%%%%%%%%%%%%%%%%%%%%%%%%%%%%%%%%%%%%%%%%%%%%%%%%%%%%%
% Written By Michael Brodskiy
% Class: Fundamentals of Linear Systems
% Professor: I. Salama
%%%%%%%%%%%%%%%%%%%%%%%%%%%%%%%%%%%%%%%%%%%%%%%%%%%%%%%%%%%%%%%%%%%%%%%%%%%%%%%%%%%%%%%%%%%%%%%%%%%%%%%%%%%%%%%%%%%%%%%%%%%%%%%%%%%%%%%%%%%%%%%%%%%%%%%%%%%%%%%%%%%

\include{Includes.tex}

\title{Lecture 4 — Classifications/Interconnections of Systems}
\date{\today}
\author{Michael Brodskiy\\ \small Professor: I. Salama}

\begin{document}

\maketitle

\begin{itemize}

  \item System Representation

    \begin{itemize}

      \item A system takes a signal as an input and transforms it into an output

      \item This is written as $x(t)$ passed through transformation function $T\left\{ \cdots \right\}$ makes $y(t)$

    \end{itemize}

  \item Linear Systems and the Principle of Superposition

    \begin{itemize}

      \item A homogenous system has zero output for zero input (if $x(t)$ transforms to $y(t)$, then $ax(t)\to ay(t)$)

      \item Additive: $x_1(t)$ causes response $y_1(t)$ and $x_2(t)$ causes response $y_2(t)$, then $x_1(t)+x_2(t)$ causes $y_1(t)+y_2(t)$

      \item A linear system is both homogenous and additive (the superposition principle applies)

    \end{itemize}

  \item Linearity

    \begin{itemize}

      \item The system with an input-output relationship $y(t)=t^2x(t)$ is linear

      \item We can prove linearity by saying:

        $$x_1(t)\to y_1(t)=T\left\{ x_1(t) \right\}=t^2x_1(t)\quad\text{ and }\quad x_2(t)\to y_2(t)=T\left\{ x_2(t) \right\}=t^2x_2(t)$$

      \item and then proving:

        $$T\left\{ a_1x_1(t)+a_2x_2(t) \right\}=t^2(a_1x_1(t)+a_2x_2(t))=a_1t_1(t)+a_2y_2(t)$$


      \item The system with an input-output relationship $y(t)=x^2(t)$ is non-linear

    \end{itemize}

\end{itemize}

\end{document}

