%%%%%%%%%%%%%%%%%%%%%%%%%%%%%%%%%%%%%%%%%%%%%%%%%%%%%%%%%%%%%%%%%%%%%%%%%%%%%%%%%%%%%%%%%%%%%%%%%%%%%%%%%%%%%%%%%%%%%%%%%%%%%%%%%%%%%%%%%%%%%%%%%%%%%%%%%%%%%%%%%%%
% Written By Michael Brodskiy
% Class: Fundamentals of Linear Systems
% Professor: I. Salama
%%%%%%%%%%%%%%%%%%%%%%%%%%%%%%%%%%%%%%%%%%%%%%%%%%%%%%%%%%%%%%%%%%%%%%%%%%%%%%%%%%%%%%%%%%%%%%%%%%%%%%%%%%%%%%%%%%%%%%%%%%%%%%%%%%%%%%%%%%%%%%%%%%%%%%%%%%%%%%%%%%%

\include{Includes.tex}

\title{Lecture 5 — Linear Time Invariant Systems}
\date{\today}
\author{Michael Brodskiy\\ \small Professor: I. Salama}

\begin{document}

\maketitle

\begin{itemize}

  \item The Impulse Response

    \begin{itemize}

      \item Linear algebra concepts apply to LTI system analysis; if the response to basis input signals are known, any general input can be expressed as a linear combination of these basis signals, allowing the output to be determined

      \item The system is fully characterized by its responses to the basis signals

      \item In the time domain representation, we use delayed unit impulse functions as basis signals

      \item The impulse response, $h(t)/h[n]$, of a linear time invariant system is defined as the response of the system to a unit impulse input, $x(t)=\delta(t)/x[n]=\delta[n]$

      \item The response of a linear time invariant system with impulse response $h[n]$ to a general input $x[n]$ can be obtained using the convolution sum; the convolution sum sifts through the sequence values, $x[k]$. respresented by an impulse located at $n=k$, and finds the corresponding response given by $h[n-k]$

        $$x[n]=\sum_{k=-\infty}^{\infty}x[k]\delta[n-k]\Longrightarrow y[n]=\sum_{k=-\infty}^{\infty}x[k]h[n-k]$$

    \end{itemize}

  \item The Convolution Sum

    \begin{itemize}

      \item The convolution sum may also be expressed as:

        $$y[n]=x[n]*h[n]=h[n]*x[n]=\sum_{k=-\infty}^{\infty}x[n-k]h[k]$$

    \end{itemize}

\end{itemize}

\end{document}

