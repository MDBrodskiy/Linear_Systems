%%%%%%%%%%%%%%%%%%%%%%%%%%%%%%%%%%%%%%%%%%%%%%%%%%%%%%%%%%%%%%%%%%%%%%%%%%%%%%%%%%%%%%%%%%%%%%%%%%%%%%%%%%%%%%%%%%%%%%%%%%%%%%%%%%%%%%%%%%%%%%%%%%%%%%%%%%%%%%%%%%%
% Written By Michael Brodskiy
% Class: Fundamentals of Linear Systems
% Professor: I. Salama
%%%%%%%%%%%%%%%%%%%%%%%%%%%%%%%%%%%%%%%%%%%%%%%%%%%%%%%%%%%%%%%%%%%%%%%%%%%%%%%%%%%%%%%%%%%%%%%%%%%%%%%%%%%%%%%%%%%%%%%%%%%%%%%%%%%%%%%%%%%%%%%%%%%%%%%%%%%%%%%%%%%

\include{Includes.tex}

\title{Lecture 6 — Properties of Linear Time Invariant Systems}
\date{\today}
\author{Michael Brodskiy\\ \small Professor: I. Salama}

\begin{document}

\maketitle

\begin{itemize}

  \item Changing the order of cascaded LTI systems does not change the overall response (commutative)

  \item You can determine the overall response by first applying the input to the first system, computing its output, and then using that output as the input to the second system; alternatively, you can find the impulse response of the equivalent system, $h_{eq}=h_1*h_2$, and use it to find the overall response, $y=x*h_{eq}$ (associative)

  \item Two systems in parallel with a single input can be added together to find the output, $y=x*(h_1+h_2)$ (distributive)

  \item If an invertible system is cascaded with its inverse system, the output will be the same as the input; the system formed by cascading an invertible system with its inverse is referred to as the identity system ($y(t)/y[n]=x(t)*h(t)*h_i(t)/x[n]*h[n]*h_i[n]\to y(t)/y[n]=x(t)/x[n]$)

    \begin{itemize}

      \item Known as the identity LTI system

    \end{itemize}
    
  \item An \nth{N} order LCCDE differential equation is an implicit description of the output in terms of the input

    $$a_N\frac{d^Ny(t)}{dt^N}+a_{N-1}\frac{d^{N-1}y(t)}{dt^{N-1}}+\cdots+a_1\frac{dy(t)}{dt}+a_0y(t)=b_M\frac{d^Mx(t)}{dt^M}+\cdots+b_ox(t)$$

  \item Since the derivative operation is irreversible, to solve for $y(t)$, further information about the conditions of $y(t)$ and its derivatives at the start of the interval is needed. These conditions are referred to as the initial our auxiliary conditions

  \item The term linear in LCC does not imply that the system is linear, but rather indicates that the equation is represented by linear combinations of the derivative of both input and output

  \item The complete solution may be written as:

    $$y(t)=y_p(t)+y_h(t)$$

    \begin{itemize}

      \item Where the $p$ component is the particular solution

        \begin{itemize}

          \item Also known as the forced solution

          \item Has the same form of the applied input

          \item Represents the steady-state response of the system

        \end{itemize}

      \item And the $h$ component is the homogenous solution

    \end{itemize}

  \item The homogenous solution may be found using:

    $$\sum_{n=0}^Na_n\frac{d^ny(t)}{dt^n}=0$$

  \item The initial rest conditions imply that the system is causal, which means that if the input $x(t)=0$ for $t<t_o$, then the output can not exist before $t=t_o$. This also enforces linearity and time invariance

\end{itemize}

\end{document}

