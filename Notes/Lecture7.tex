%%%%%%%%%%%%%%%%%%%%%%%%%%%%%%%%%%%%%%%%%%%%%%%%%%%%%%%%%%%%%%%%%%%%%%%%%%%%%%%%%%%%%%%%%%%%%%%%%%%%%%%%%%%%%%%%%%%%%%%%%%%%%%%%%%%%%%%%%%%%%%%%%%%%%%%%%%%%%%%%%%%
% Written By Michael Brodskiy
% Class: Fundamentals of Linear Systems
% Professor: I. Salama
%%%%%%%%%%%%%%%%%%%%%%%%%%%%%%%%%%%%%%%%%%%%%%%%%%%%%%%%%%%%%%%%%%%%%%%%%%%%%%%%%%%%%%%%%%%%%%%%%%%%%%%%%%%%%%%%%%%%%%%%%%%%%%%%%%%%%%%%%%%%%%%%%%%%%%%%%%%%%%%%%%%

\include{Includes.tex}

\title{Lecture 7 — The Laplace Transform}
\date{\today}
\author{Michael Brodskiy\\ \small Professor: I. Salama}

\begin{document}

\maketitle

\begin{itemize}

  \item For an input $x(t)=e^{st}$ passed through $h(t)$, we get the Laplace transform determined as:

    $$y(t)=H(s)e^{st}$$

    \begin{itemize}

      \item With $s$ in some form $s=\sigma+j\omega$
        
    \end{itemize}

  \item $H(s)$ can be expressed as:

    $$H(s)=\int_{-\infty}^{\infty} e^{-s\tau}h(\tau)\,d\tau$$

    \begin{itemize}

      \item Taking $s=j\omega$, this is equivalent to writing:

        $$H(j\omega)=\int_{-\infty}^{\infty} e^{-j\omega\tau}h(\tau)\,d\tau$$

      \item This is known as the continuous time Fourier transform; note that this is a special case of the Laplace Transform

    \end{itemize}

  \item The Laplace Transform is defined as:

    $$X(s)=\int_{-\infty}^{\infty}x(t)e^{-st}\,dt$$

    \begin{itemize}

      \item Example transform:

        $$x(t)=e^{-at}u(t)\longleftrightarrow X(s)=\frac{1}{s+a}$$

    \end{itemize}

\end{itemize}

\end{document}

