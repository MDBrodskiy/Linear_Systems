%%%%%%%%%%%%%%%%%%%%%%%%%%%%%%%%%%%%%%%%%%%%%%%%%%%%%%%%%%%%%%%%%%%%%%%%%%%%%%%%%%%%%%%%%%%%%%%%%%%%%%%%%%%%%%%%%%%%%%%%%%%%%%%%%%%%%%%%%%%%%%%%%%%%%%%%%%%%%%%%%%%
% Written By Michael Brodskiy
% Class: Fundamentals of Linear Systems
% Professor: I. Salama
%%%%%%%%%%%%%%%%%%%%%%%%%%%%%%%%%%%%%%%%%%%%%%%%%%%%%%%%%%%%%%%%%%%%%%%%%%%%%%%%%%%%%%%%%%%%%%%%%%%%%%%%%%%%%%%%%%%%%%%%%%%%%%%%%%%%%%%%%%%%%%%%%%%%%%%%%%%%%%%%%%%

\include{Includes.tex}

\title{Lecture 8 — The Fourier Transform}
\date{\today}
\author{Michael Brodskiy\\ \small Professor: I. Salama}

\begin{document}

\maketitle

\begin{itemize}

  \item A periodic signal, $x(t)$, with period $T_o$ can be expressed as a sum of complex exponentials at the fundamental frequency and its harmonics. The analysis equation may be written as:

    $$X(\omega)=\int_{-\infty}^{\infty} x(t)e^{-j\omega t}\,dt$$

  \item The synthesis equation may be written as:

    $$x(t)=\frac{1}{2\pi}\int_{-\infty}^{\infty}X(\omega)e^{j\omega t}\,d\omega$$

  \item The Fourier Transform exists only if the $j\omega$ axis lies within the ROC of the Laplace Transform

  \item Fourier Transforms are governed by the Dirichlet conditions:

    \begin{enumerate}

      \item $x(t)$ is absolutely integrable

      \item $x(t)$ has a finite number of maxima and minima over any finite interval

      \item $x(t)$ has a finite number of finite discontinuities over any finite interval

      \item Note: Periodic signals do not satisfy these conditions but are considered to have Fourier transforms if impulse functions are included in the Fourier representation

    \end{enumerate}

  \item Properties of Fourier Transforms

    \begin{itemize}

      \item Let $X(\omega)=A(\omega)+jB(\omega)$

      \item $X(-\omega)=A(-\omega)+jB(-\omega)=X^*(\omega)=A(\omega)-jB(\omega)$

      \item $A(-\omega)=A(\omega)$, real part is an even function

      \item $B(-\omega)=-B(\omega)$, imaginary part is an odd function

      \item $|X(\omega)|=\sqrt{A^2(\omega)+B^2(\omega)}$ is an even function

      \item $\angle X(\omega)=\tan^{-1}\left( \frac{B(\omega)}{A(\omega)} \right)$ is an odd function

      \item For a real signal: $x(t)=x^*(t)\to X(\omega)=X^*(-\omega)$ and $X(-\omega)=X^*(\omega)$

    \end{itemize}

\end{itemize}

\end{document}

